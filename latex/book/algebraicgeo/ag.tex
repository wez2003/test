\documentclass[12pt,a4paper]{book}
%宏包
\usepackage{amsmath}
\usepackage{amssymb}
\usepackage{amsthm}
\usepackage{mathrsfs}
\usepackage{geometry}
\usepackage{natbib}%bibtex
\usepackage[dvipsnames]{xcolor}
\usepackage{tcolorbox}
\usepackage{enumerate}
\usepackage{tikz}
\usepackage{tikz-cd}
\usepackage{quiver}
\usepackage{float}
\usepackage{caption}
\usepackage[colorlinks,linkcolor=blue]{hyperref}
\usepackage{enumerate}
\usepackage{tabularx}%控制列宽
\usepackage{xr}%跨文件引用
\usepackage{verbatim}%多行注释
\externaldocument{D:/latex/book/algebra/algebra}
\externaldocument{D:/latex/book/number theory/number}
%页面设置
\linespread{1.2}
\geometry{a4paper,left=2cm,right=2cm,top=2.5cm,bottom=2cm}
%\geometry{a4paper,left=2cm,right=2cm,top=2.5cm,bottom=2cm}

%环境和宏指令
\newenvironment{prooff}{{\noindent\it\textcolor{cyan!40!black}{Proof}:}\,}{\par}
\newenvironment{proofff}{{\noindent\it\textcolor{cyan!40!black}{Proof of the lemma}:}\,}{\qed \par}
\newcommand{\bbrace}[1]{\left\{ #1 \right\} }
\newcommand{\bb}[1]{\mathbb{#1}}
\newcommand{\p}{^{\prime}}
\renewcommand{\mod}[1]{(\text{mod}\,#1)}
\newcommand{\blue}[1]{\textcolor{blue}{#1}}
\newcommand{\spec}[1]{\text{Spec}({#1})}
\newcommand{\rarr}[1]{\xrightarrow{#1}}
\newcommand{\larr}[1]{\xleftarrow{#1}}
\newcommand{\emptyy}{\underline{\quad}}
\newenvironment{enu}{\begin{enumerate}[(1)]}{\end{enumerate}}
%ctrl+点击文本返回代码  选中代码 ctrl+alt+j 为代码查找文本




%定理环境
\theoremstyle{definition}
\newtheorem{defn}{Definition}[section]
\newtheorem{coro}[defn]{Corollary}
\newtheorem{theo}[defn]{Theorem}
\newtheorem{exer}[defn]{Exercise}
\newtheorem{rema}[defn]{Remark}
\newtheorem{lem}[defn]{Lemma}
\newtheorem{prop}[defn]{Proposition}
\newtheorem{nota}[defn]{Notation}
\newtheorem{exam}[defn]{Example}



\begin{document}
\title{Algebraic Geometry}
\author{Erzhuo Wang}
\date{\today}
\maketitle % 标题页
\tableofcontents
\begin{comment}
\chapter{Classical Theory}
\section{Affine Case}
\begin{defn}
    We will interpret the elements of $\bb{A}$ as functions from the affine $n$-space to $k$, by defining $f(P)=f\left(a_1, \ldots, a_n\right)$, where $f \in A$ and $P \in \bb{A}^n$. Thus if $f \in A$ is a polynomial, we can talk about the set of zeros of $f$, namely $Z(f)=\left\{P \in \bb{A}^n \mid f(P)=0\right\}$. More generally, if $T$ is any subset of $\bb{A}$, we define the zero set of $T$ to be the common zeros of all the elements of $T$, namely
    $$
        Z(T)=\left\{P \in \bb{A}^n \mid f(P)=0 \text { for all } f \in T\right\} .
    $$
    A subset $Y$ of $\bb{A}^n$ is an algebraic set if there exists a subset $T \subseteq A$ such that $Y=Z(T)$.
\end{defn}
\begin{prop}
    The union of two algebraic sets is an algebraic set. The intersection of any family of algebraic sets is an algebraic set. The empty set and the whole space are algebraic sets.
\end{prop}
\begin{prooff}
    If $Y_1=Z\left(T_1\right)$ and $Y_2=Z\left(T_2\right)$, then $Y_1 \cup Y_2=Z\left(T_1 T_2\right)$, where $T_1 T_2$ denotes the set of all products of an element of $T_1$ by an element of $T_2$. Indeed, if $P \in Y_1 \cup Y_2$, then either $P \in Y_1$ or $P \in Y_2$, so $P$ is a zero of every polynomial in $T_1 T_2$. Conversely, if $P \in Z\left(T_1 T_2\right)$, and $P \notin Y_1$ say, then there is an $f \in T_1$ such that $f(P) \neq 0$. Now for any $g \in T_2,(f g)(P)=0$ implies that $g(P)=0$, so that $P \in Y_2$.


    If $Y_\alpha=Z\left(T_\alpha\right)$ is any family of algebraic sets, then $\bigcap Y_\alpha=Z\left(\bigcup T_\alpha\right)$, so $\bigcap Y_\alpha$ is also an algebraic set. Finally, the empty set $\varnothing=Z(1)$, and the whole space $\bb{A}^n=Z(0)$.
\end{prooff}
\begin{defn}
    We define the Zariski topology on $\bb{A}^n$ by taking the open subsets to be the complements of the algebraic sets. This is a topology, because according to the proposition, the intersection of two open sets is open, and the union of any family of open sets is open. Furthermore, the empty set and the whole space are both open.
\end{defn}
\begin{defn}
    For any subset $Y \subseteq \bb{A}^n$, let us define the ideal of $Y$ in $\bb{A}$ by
    $$
        I(Y)=\{f \in A \mid f(P)=0 \text { for all } P \in Y\} .
    $$
\end{defn}
\begin{prop}
    \begin{enu}
        \item If $T_1 \subseteq T_2$ are subsets of $\bb{A}$, then $Z\left(T_1\right) \supseteq Z\left(T_2\right)$.
        \item If $Y_1 \subseteq Y_2$ are subsets of $\bb{A}^n$, then $I\left(Y_1\right) \supseteq I\left(Y_2\right)$.
        \item For any two subsets $Y_1, Y_2$ of $\bb{A}^n$, we have $I\left(Y_1 \cup Y_2\right)=I\left(Y_1\right) \cap I\left(Y_2\right)$.
        \item For any subset $Y \subseteq \bb{A}^n, Z(I(Y))=\bar{Y}$, the closure of $Y$.
        \item an algebraic set $Y$ is irreducible if and only if $I(Y)$ is a prime ideal.
    \end{enu}
    \label{proposition:spec of algebraic set}
\end{prop}
\begin{prooff}
    (4): We note that $Y \subseteq Z(I(Y))$, which is a closed set, so clearly $\bar{Y} \subseteq Z(I(Y))$. On the other hand, let $W$ be any closed set containing $Y$. Then $W=Z(\mathfrak{a})$ for some ideal $\mathfrak{a}$. So $Z(\mathfrak{a}) \supseteq Y$, and by (b), $I Z(\mathfrak{a}) \subseteq I(Y)$. But certainly $\mathfrak{a} \subseteq I Z(\mathfrak{a})$, so by (a) we have $W=Z(\mathfrak{a}) \supseteq Z I(Y)$. Thus $Z I(Y)=\bar{Y}$

    (5): If $Y$ is irreducible, we show that $I(Y)$ is prime. Indeed, if $f_g \in I(Y)$, then $Y \subseteq Z(f g)=Z(f) \cup Z(g)$. Thus $Y=$ $(Y \cap Z(f)) \cup(Y \cap Z(g))$, both being closed subsets of $Y$. Since $Y$ is irreducible, we have either $Y=Y \cap Z(f)$, in which case $Y \subseteq Z(f)$, or $Y \subseteq$ $Z(g)$. Hence either $f \in I(Y)$ or $g \in I(Y)$.
    Conversely, if $Y=Y_1\cap Y_2$, $Y_1,Y_2 \subsetneqq Y$ are closed subset of $A^n$, then by (4), $I(Y_1)\supsetneqq  I(Y)$. Hence take $f\in I(Y_1)$ such that $f\notin I(Y)$. Similarly, we can take $g\in I(Y_2)$ such that $g\notin I(Y)$, then $fg\in I(Y_1\cup Y_2)=I(Y)$. A contradiction!
\end{prooff}

\blue{From now on, we assumne $k$ is algebracally closed.}

\begin{theo}[Hilbert's Nullstellensatz]
    If $k$ is algebracally closed, then
    $$
        I(V(A))=\sqrt{A} .
    $$
\end{theo}
\begin{theo}
    If $k$ is algebraically closed, then there is a one-to-one inclusion-reversing correspondence between algebraic sets(irreducible algebraic sets, points) in $\bb{A}^n$ and radical ideals(prime ideals, maximal ideals)  in $k[x_1,\dots,x_n]$, given by $Y \mapsto I(Y)$ and $\mathfrak{a} \mapsto Z(\mathfrak{a})$.
\end{theo}
\begin{theo}
    Let $\mathfrak{a} \subseteq k\left[T_1, \ldots, T_n\right]$ be a radical ideal, i.e., $\mathfrak{a}=\operatorname{rad}(\mathfrak{a})$. Then $\mathfrak{a}$ is 
    the intersection of a finite number of prime ideals that do not contain each other. The set of these prime ideals is uniquely determined by $\mathfrak{a}$.

    Equivalently, every algebraic set $V$ can be written uniquely as union of irreducible algebraic sets $V_1,\dots,V_r$ such that $V_i\nsubseteq V_j$.
\end{theo}
\begin{defn}
    Let $X \subseteq \mathbb{A}^m(k)$ and $Y \subseteq \mathbb{A}^n(k)$ be affine algebraic sets. $A$ morphism $X \rightarrow Y$ of affine algebraic sets is a map $f: X \rightarrow Y$ of the underlying sets such that there exist polynomials $f_1, \ldots, f_n \in k\left[T_1, \ldots, T_m\right]$ with $f(x)=\left(f_1(x), \ldots, f_n(x)\right)$ for all $x \in X$.
\end{defn}
\begin{prop}
    Let $X$ be an affine algebraic set. The affine coordinate ring $\Gamma(X)$ is a reduced finitely generated $k$-algebra. Moreover, $X$ is irreducible if and only if $\Gamma(X)$ is an integral domain.
\end{prop}
\begin{defn}
    $X$ be an algebraic set with subspace topology of $\mathbb{A}^m(k)$. Then topology is the same as the topology defined by closed subsets 
    of the form 
    \begin{equation*}
        V(\mathfrak{a})=\{x \in X :\forall f \in \mathfrak{a}: f(x)=0\}=V\left(\pi^{-1}(\mathfrak{a})\right) \cap X
    \end{equation*}
    where $\mathfrak{a}$ is an ideal of $\Gamma(X)$.
\end{defn}
\begin{defn}[principal open subsets]
    For $f \in \Gamma(X)$ we set
    $$
    D(f):=\{x \in X ; f(x) \neq 0\}=X \backslash V(f)
    $$
    These are open subsets of $X$, called principal open subsets. The open sets $D(f), f \in \Gamma(X)$, form a basis of the subspace topology of $X$.
\end{defn}
\begin{prop}
      Let $f: X \rightarrow Y$ be a morphism of affine algebraic sets. The map
    $$
    \Gamma(f): \operatorname{Hom}\left(Y, \mathbb{A}^1(k)\right) \rightarrow \operatorname{Hom}\left(X, \mathbb{A}^1(k)\right), \quad g \mapsto g \circ f
    $$
    defines a homomorphism of $k$-algebras. We obtain a functor
    $$
    \Gamma: \text { (affine algebraic sets) }{ }^{\text {opp }} \rightarrow \text { (reduced finitely generated } k \text {-algebras). }
    $$
    The functor $\Gamma$ induces an equivalence of categories. By restriction one obtains an equivalence of categories
    $$
    \Gamma:(\text { irreducible affine algebraic sets })^{\mathrm{opp}} \rightarrow(\text { integral finitely generated } k \text {-algebras }) \text {. }
    $$
\end{prop}
\begin{prooff}
    We show that $\Gamma$ is fully faithful, i.e., that for affine algebraic sets $X \subseteq \mathbb{A}^m(k)$, $Y \subseteq \mathbb{A}^n(k)$ the map $\Gamma: \operatorname{Hom}(X, Y) \rightarrow \operatorname{Hom}(\Gamma(Y), \Gamma(X))$ is bijective. 
    We define an inverse map. If $\varphi: \Gamma(Y) \rightarrow \Gamma(X)$ is given, define 
    $$
    \bar{\varphi}:X\rightarrow Y, x\mapsto (\varphi(y_1+I(Y))(x),\dots,\varphi(y_n+I(Y))(x))
    $$
    and obtain the desired inverse map.
\end{prooff}
\begin{prop}
    Let $f: X \rightarrow Y$ be a morphism of affine algebraic sets and let $\Gamma(f): \Gamma(Y) \rightarrow \Gamma(X)$ be the corresponding homomorphism of the affine coordinate rings. Then $\Gamma(f)^{-1}\left(\mathfrak{m}_x\right)=\mathfrak{m}_{f(x)}$ for all $x \in X$ since $g(f(x))=\Gamma(f)(g)(x)$ for $g \in \Gamma(Y)=\operatorname{Hom}\left(Y, \mathbb{A}^1(k)\right)$.
\end{prop}   
\begin{defn}
    A space with functions over $K$ is a topological space $X$ together with a family $\mathscr{O}_X$ of $K$-subalgebras $\mathscr{O}_X(U) \subseteq \operatorname{Map}(U, K)$ for every open subset $U \subseteq X$ that satisfy the following properties:
\begin{enu}
\item If $U^{\prime} \subseteq U \subseteq X$ are open and $f \in \mathscr{O}_X(U)$, the restriction $f_{\mid U^{\prime}} \in \operatorname{Map}\left(U^{\prime}, K\right)$ is an element of $\mathscr{O}_X\left(U^{\prime}\right)$.
\item (Axiom of Gluing) Given open subsets $U_i \subseteq X, i \in I$, and $f_i \in \mathscr{O}_X\left(U_i\right), i \in I$, with
$$
f_{i \mid U_i \cap U_j}=f_{j \mid U_i \cap U_j} \quad \text { for all } i, j \in I \text {, }
$$
the unique function $f: \bigcup_i U_i \rightarrow K$ with $f_{\mid U_i}=f_i$ for all $i \in I$ lies in $\mathscr{O}_X\left(\bigcup_i U_i\right)$. 
\end{enu}
\end{defn}
\begin{defn}
    A morphism $g:\left(X, \mathscr{O}_X\right) \rightarrow\left(Y, \mathscr{O}_Y\right)$ of spaces with functions is a continuous map $g: X \rightarrow Y$ such that for all open subsets $V \subseteq Y$ and functions $f \in \mathscr{O}_Y(V)$ the function $f \circ g_{\mid g^{-1}(V)}: g^{-1}(V) \rightarrow K$ lies in $\mathscr{O}_X\left(g^{-1}(V)\right)$.
\end{defn} 
Let $X \subseteq \mathbb{A}^n(k)$ be an irreducible affine algebraic set. 
It is endowed with the Zariski topology and we want to define for every open subset $U \subseteq X$ a $k$-algebra of functions $\mathscr{O}_X(U)$ such that $\left(X, \mathscr{O}_X\right)$ is a space with functions.                                                                                                                                                                                                                                                                                                                                                                                                                                                                                                                                                                                                                                                                                                                                                                                                                                                                                                                                                                                                                                                                                                                                                                                                                                                                                                                                                                                                                                                                                                                                                                                                                                                                                                                                                                                                                                                                                                                                                                                                                                                                                                                                                                                                                                                                                                                                                                                                                                                                                                                                                                                                                                                                                                                                                                                                                                                                                                                                                                                                                                                                                                                                                                                                                                                                                                                                                                                                                                                                                                                                                                                                                                                                                                                                                                                                                                                                                                                                                                                                                                                                                                                                                                                                                                                                                                                                                                                                                                                                                                                                                                                                                                                                                                                                                                                                                                                                                                                                                                                                                                                                                                                                                                                                                                                                                                                                                                                                                                                                                                                                                                                                                                                                                                                                                                                                                                                                                                                                                                                                                                                                                                                                                                                                                                                                                                                                                                                                                                                                                                                                                                                                                                                                                                                                                                                                                                                                                                                                                                                                                                                                                                                                                                                                                                                                                                                                                                                                                                                                                                                                                                                                                                                                                                                                                                                                                                                                                                                                                                                                                                                                                                                                                                                                                                                                                                                                                                                                                                                                                                                                                                                                                                                                                                                                                                                                                                                                                                                                                                                                                                                                                                                                                                                                                                                                                                                                                                                                                                                                                                                                                                                                                                                                                                                                                                                                                                                                                                                                                                                                                                                                                                                                                                                                                                                                                                                                                                                                                                                                                                         
% \begin{defn}
%     The field of fractions $K(X):=\operatorname{Frac}(\Gamma(X))$ is called the function field of $X$.
% \end{defn}
\begin{lem} 
    Let $X$ be an irreducible affine algebraic set and let $\frac{f_1}{g_1}$ and $\frac{f_2}{g_2}$ be elements of $K(X)\left(f_1, f_2, g_1, g_2 \in \Gamma(X)\right)$, such that there exists a non-empty open subset $U \subseteq D\left(g_1 g_2\right)$ with:
    $$
    \forall x \in U: \frac{f_1(x)}{g_1(x)}=\frac{f_2(x)}{g_2(x)}
    $$
    Then 
    $$
    \frac{f_1}{g_1}=\frac{f_2}{g_2}
    $$ in $K(X)$.
\end{lem}
\begin{prooff}
    Notice that $U $ dense in $X$.
\end{prooff}
\begin{defn}
    Let $X$ be an \blue{irreducible} affine algebraic set and let $\emptyset \neq U \subseteq X$ be open. We denote by $\mathfrak{m}_x$ the maximal ideal of $\Gamma(X)$ corresponding to $x \in X$ and by $\Gamma(X)_{\mathfrak{m}_x}$ the localization of the affine coordinate ring with respect to $\mathfrak{m}_x$. We define
    $$
    \mathscr{O}_X(U)=\bigcap_{x \in U} \Gamma(X)_{\mathfrak{m}_x} \subset \text{Frac}(\Gamma(X))
    $$
    We let $\mathscr{O}_X(\emptyset)$ be a singleton.、

    To consider $\left(X, \mathscr{O}_X\right)$ as space with functions, 
    we first have to explain how to identify elements 
    $f \in \mathscr{O}_X(U)$ with functions $U \rightarrow k$. 
    Given $x \in U$ the element $f$ is by definition in $\Gamma(X)_{\mathfrak{m}_x}$ and we may write $f=\frac{g}{h}$ with $g, h \in \Gamma(X), h \notin \mathfrak{m}_x$. But then $h(x) \neq 0$ and we may set $f(x):=g(x)/h(x) \in k$. The value $f(x)$ is well defined and this construction defines an injective map $\mathscr{O}_X(U) \rightarrow \operatorname{Map}(U, k)$.
\end{defn}
\begin{prop}
    Let $\left(X, \mathscr{O}_X\right)$ be the space with functions associated to the irreducible affine algebraic set $X$ and let $f \in \Gamma(X)$. Then there is an equality
    $$
    \mathscr{O}_X(D(f))=\Gamma(X)_f.
    $$
    In particular $\mathscr{O}_X(X)=\Gamma(X)($ taking $f=1)$.
\end{prop}
\begin{coro}
    A function $f: U \rightarrow k\in \mathscr{O}_X(U)$ iff for all $x\in U$, there is an open neighborhood $V$ with $x \in V \subseteq U$, and $g, h \in \Gamma(X)$, such that 
    $h$ is nowhere zero on $V$, and $f=g / h$ on $V$.
    \label{Corollary:regular function,space with functions}
\end{coro}
\begin{theo}
    Let $X, Y$ be irreducible affine algebraic sets and $f: X \rightarrow Y$ a map. The following assertions are equivalent.
\begin{enu}   
    \item The map $f$ is a morphism of affine algebraic sets.
    \item If $g \in \Gamma(Y)$, then $g \circ f \in \Gamma(X)$.
    \item The map $f$ is a morphism of spaces with functions, i.e., $f$ is continuous and if $U \subseteq Y$ open and $g \in \mathscr{O}_Y(U)$, then $g \circ f_{\mid f^{-1}(U)} \in \mathscr{O}_X\left(f^{-1}(U)\right)$.
\end{enu}
    Proof. The equivalence of (1) and (2) has already been proved. 
    Moreover, it is clear that (2) is implied by (3) by 
    taking $U=Y$. 
    
Let $\varphi: \Gamma(Y) \rightarrow \Gamma(X)$ be the homomorphism $h \mapsto h \circ f$. For $g \in \Gamma(Y)$ we have
    $$
    f^{-1}(D(g))=\{x \in X ; g(f(x)) \neq 0\}=D(\varphi(g))
    $$
    As the principal open subsets form a basis of the topology, this shows that $f$ is continuous. The homomorphism $\varphi$ induces a homomorphism of the localizations $\Gamma(Y)_g \rightarrow \Gamma(X)_{\varphi(g)}$. 
    By definition of $\varphi$ this is the map $\mathscr{O}_Y(D(g)) \rightarrow \mathscr{O}_X(D(\varphi(g))), \quad h \mapsto h \circ f$. This shows the claim if $U$ is principal open. As we can obtain functions on arbitrary open subsets of $Y$ by gluing functions on principal open subsets, this proves (3).

\end{theo}
Altogether we obtain
$X \mapsto\left(X, \mathscr{O}_X\right)$
 defines a fully faithful functor (Irreducible affine algebraic sets) $\rightarrow$ (Spaces with functions over $k)$.
\begin{defn}
    We call a space with functions $\left(X, \mathscr{O}_X\right)$ connected, if the underlying topological space $X$ is connected.
\begin{enu} 
    \item An affine variety is a space with functions that is isomorphic to a space with functions associated to an irreducible affine algebraic set.
    \item A prevariety is a connected space with functions $\left(X, \mathscr{O}_X\right)$ with the property that there exists a finite open covering $X=\bigcup_{i=1}^n U_i$ such that the space with functions $\left(U_i, \mathscr{O}_{X \mid U_i}\right)$ is an affine variety for all $i=1, \ldots, n$.
    \item A morphism of prevarieties is a morphism of spaces with functions.
\end{enu}
\end{defn}
\begin{prop}
    Let $X \subseteq \mathbb{A}^n(k)$ be any subspace. Then $X$ is noetherian.
\end{prop}
\begin{prooff}
    By Hilbert Basis Theorem, $k[x_1,\dots,x_n]$ is noetherian. Hence $\bb{A}^n(k)$
    is noetherian.
\end{prooff}
\begin{prop}
    Let $\left(X, \mathscr{O}_X\right)$ be a prevariety. The topological space $X$ is noetherian (in particular quasi-compact) and irreducible.
\end{prop}
\begin{prooff}
    By Algebra\,\ref{proposition: covered by connected open subset}.
\end{prooff}
\begin{theo}[principal open subset of affine variety is affine]
    Let $X$ be an affine variety, $0\neq f \in \Gamma(X)=\mathscr{O}_X(X)$, and let $D(f) \subseteq X$ be the corresponding principal open subset. 
    Let $\Gamma(X)_f$ be the localization of $\Gamma(X)$ by $f$ and let $\left(Y, \mathscr{O}_Y\right)$ be the affine variety corresponding to this integral finitely generated $k$-algebra. 
    Then $\left(D(f), \mathscr{O}_{X \mid D(f)}\right)$ and $\left(Y, \mathscr{O}_Y\right)$ are isomorphic spaces with functions. In particular, $\left(D(f), \mathscr{O}_{X \mid D(f)}\right)$ is an affine variety. 
\end{theo}
\begin{prooff}
    Consider $ Y=\left\{\left(x, x_{n+1}\right) \in X \times \mathbb{A}^1(k) ; x_{n+1} f(x)=1\right\}$. We will show that 
    $$(Y,\mathscr{O}_Y)\simeq \left(D(f), \mathscr{O}_{X}(D(f))\right)$$ as spaces with functions.
    
    It's suffice to check the projection $(x,x_{n+1})\mapsto x$ and 
    $x\mapsto (x, 1/f(x))\in Y$ are morphisms between spaces of functions. 
    And we only need to check the case when open subsets of $X$ or $Y$ are principal since 
    we can apply Axiom of Gluing. 
\end{prooff}
\begin{coro}[open subspace of prevariety is prevariety]
    Let $\left(X, \mathscr{O}_X\right)$ be a prevariety and let $U \subseteq X$ be a non-empty open subset. Then $\left(U, \mathscr{O}_{X \mid U}\right)$ is a prevariety and the inclusion $U \rightarrow X$ 
    is a morphism of prevarieties. Moreover, if $X$ is a prevariety, open affine subsets of $X$
    form a basis of topology of $X$.
\end{coro}
% \begin{defn}[Function spacce of prevariety]
    
% \end{defn}
\begin{defn}[closed subprevarieties]
    Let $X$ be a prevariety and let $Z \subseteq X$ be an irreducible closed subset. We want to define on $Z$ the structure of a prevariety. For this we have to define functions on open subsets $U$ of $Z$. We define:
    $$
    \mathscr{O}_Z^{\prime}(U)=\left\{f \in \operatorname{Map}(U, k) ; \forall x \in U: \exists x \in V \subseteq X \text { open, } g \in \mathscr{O}_X(V): f_{\mid U \cap V}=g_{\mid U \cap V}\right\}
    $$
\end{defn}
\begin{prop}
    $\left(Z, \mathscr{O}_Z^{\prime}\right)$ is a space with functions and that $\mathscr{O}_X^{\prime}=\mathscr{O}_X$.
\end{prop}
\begin{prooff}
    We want to show that given open subsets $U_i \subseteq Z$ and $f_i \in \mathcal{O}_Z^{\prime}\left(U_i\right), i \in I$, with
    $$
    f_{i \mid U_t \cap U_j}=f_{j \mid U_t \cap U_j}
    $$
    for all $i, j \in I$ the unique function $f: \bigcup_i U_i \rightarrow k$ with $f_{\mid U_i}=f_i$ lies in $\mathcal{O}_Z^{\prime}\left(\bigcup_i U_i\right)$.
    Note that the $U_i$ are assumed to be open in $Z$ (not necessarily in $X!$ ) and set $U=\bigcup_i U_i$. To check $f \in \mathcal{O}_Z^{\prime}(U)$, assume $x \in U$
    then $x \in U_i$ for some $i \in I$. As $f_i \in \mathcal{O}_Z^{\prime}\left(U_i\right)$ there is $V$ open in $X$ and $g\in \mathcal{O}_X\p(V)$ such that 
    \begin{equation*}
        \left.f\right|_{U_i\cap V}=\left.g\right|_{U_i\cap V}
    \end{equation*}
    Notice that $U_i=W_i\cap Z$ for some $W_i$ open in $X$ and 
    \begin{equation*}
        U_i\cap V= W_i\cap Z\cap V= W_i\cap Z\cap V \cap U=(W_i\cap V) \cap U
    \end{equation*}
    Hence, $f\in \mathcal{O}\p_Z(U)$.
\end{prooff}
\begin{prop}
Let $X \subseteq \mathbb{A}^n(k)$ be an irreducible affine algebraic set and let $Z \subseteq X$ be an irreducible closed subset. Then the space with functions $\left(Z, \mathscr{O}_Z\right)$ associated to the affine algebraic set $Z$ and the above defined space with functions $\left(Z, \mathscr{O}_Z^{\prime}\right)$ coincide.
\label{proposition: closed subprevariety of affine variety}
\end{prop}
\begin{prooff}
    Step 1: $Z$ is closed in $\bb{A}^n(k)$ since $Z=\bar{Z}\cap X$ and $X$ is closed in $\bb{A}^n(k)$.

    Step 2: $\mathscr{O}\p_Z(U)\subset \mathscr{O}_Z(U)$: By Corollary\,\ref{Corollary:regular function,space with functions}.

    Step 3: $\mathscr{O}\p_Z(U)\supset \mathscr{O}_Z(U)$: Let $f \in \mathscr{O}_Z(U)$. For $x \in U$ there exists $h \in \Gamma(Z)$ with $x \in D(h) \subseteq U$. 
    The restriction $\left.f\right|_{D(h)}\in \mathscr{O}_Z(D(h))=\Gamma(Z)_h$ has the form $f=\dfrac{g}{h^n}, n \geq 0, g \in \Gamma(Z)$. 
    We lift $g$ and $h$ to elements in $\tilde{g}, \tilde{h} \in \Gamma(X)$, set $V:=D(\tilde{h}) \subseteq X$, 
    and obtain $\dfrac{\tilde{g}}{\tilde{h}^n} \in \mathscr{O}_X(D(\tilde{h}))$ and 
    $$\left.f\right|_{U\cap V}=\left.\frac{\tilde{g}}{\tilde{h}^n}\right|_{U\cap V}$$.
\end{prooff}
\begin{prop}
    Let $X$ be a prevariety and let $Z \subseteq X$ be an irreducible closed subset. Let $\mathscr{O}_Z$ be the system of functions defined above. Then $\left(Z, \mathscr{O}_Z\right)$ is a prevariety. The inclusion $Z \hookrightarrow X$ is a morphism of prevarieties.
\end{prop}
\begin{prooff}
    Step 1: If
    $$
          X=\bigcup_{i=1}^n U_i
    $$
    where $U_i$ be affine open subet of $X$, $Z\cap U_i$ is closed irreducible subset of $U_i$. This is because, 
    for $W_1,W_2$ be open subets of $X$ and $Y_j=W_j\cap Z\cap U_i\neq \varnothing,j=1,2$, since $Z$ is irreducible, the intersection  
    of $Y_1$ and $Y_2$ is non-empty. Hence, $Z\cap U_i$ is irreducible.

    Step 2:  Since $X$ can be covered by affine open subset, the proposition follows from Proposition\,\ref{proposition: closed subprevariety of affine variety}.
\end{prooff}
\section{Projective Case}
\begin{lem}
    Let $\mathcal{F}$ be the subset of $K\left(X_0, \ldots, X_n\right)$ that consists of those elements $\frac{f}{g}$, where $f, g \in K\left[X_0, \ldots, X_n\right]$ are homogeneous polynomials of the same degree. It is easy to check that $\mathcal{F}$ is a subfield of $K\left(X_0, \ldots, X_n\right)$ and 
    $$
    \Phi_i: \mathcal{F} \xrightarrow{\sim} K\left(T_0, \ldots, \widehat{T}_i, \ldots, T_n\right), \quad \frac{f}{g} \mapsto \frac{\Phi_i(f)}{\Phi_i(g)}
    $$
    is an isomorphism.
\end{lem}
\begin{defn}[projective space]
    As a set we define for every field $k$ (not necessarily algebraically closed)
    $$
    \mathbb{P}^n(k)=\left\{\text { lines through the origin in } k^{n+1}\right\}=\left(k^{n+1} \backslash\{0\}\right) / k^{\times} .
    $$
    Here a line through the origin is per definition a 1-dimensional $k$-subspace and we denote by $\left(k^{n+1} \backslash\{0\}\right) / k^{\times}$the set of equivalence classes in $k^{n+1} \backslash\{0\}$ with respect to the equivalence relation
    $$
    \left(x_0, \ldots x_n\right) \sim\left(x_0^{\prime}, \ldots, x_n^{\prime}\right) \Leftrightarrow \exists \lambda \in k^{\times}: \forall i: x_i=\lambda x_i^{\prime}
    $$
    For $0 \leq i \leq n$ we set
    $$
    U_i:=\left\{\left(x_0: \ldots: x_n\right) \in \mathbb{P}^n(k) ; x_i \neq 0\right\} \subset \mathbb{P}^n(k)
    $$
    This subset is well-defined and the union of the $U_i$ for $0 \leq i \leq n$ is all of $\mathbb{P}^n(k)$. There are bijections
$$
U_i \xrightarrow{\sim} \mathbb{A}^n(k), \quad\left(x_0: \ldots: x_n\right) \mapsto\left(\frac{x_0}{x_i}, \ldots, \frac{\widehat{x_i}}{x_i}, \ldots \frac{x_n}{x_i}\right)
$$
Via this bijection we will endow $U_i$ with the structure of a space with functions, isomorphic to $\left(\mathbb{A}^n(k), \mathscr{O}_{\mathbb{A}^n(k)}\right)$, which we denote by $\left(U_i, \mathscr{O}_{U_i}\right)$.
    
Topology on $\mathbb{P}^n(k)$: We define the topology on $\mathbb{P}^n(k)$ by calling a subset $U \subseteq \mathbb{P}^n(k)$ open if $U \cap U_i$ is open in $U_i$ for all $i$. This defines a topology on $\mathbb{P}^n(k)$ and with this definition, $\left(U_i\right)_{0 \leq i \leq n}$ is an open covering of $\mathbb{P}^n(k)$. 

$\mathbb{P}^n(k)$ is connected: If $V_1\cup V_2=\mathbb{P}^n(k)$, $V_1\cap V_2=\varnothing$ and $V_1,V_2$ open, we obtain 
\begin{equation*}
    (V_1\cap U_i)\cup (V_2\cap U_i)=U_i, i=0,\dots,n.
\end{equation*} 
Hence one of them contains $U_i$ for all $i=0,\dots,n$. A contradiction!

Subspace topology on $U_i$ is $U_i$ itself: 

Spaces with functions: We still have to define functions on open subsets $U \subseteq \mathbb{P}^n(k)$. We set
$$
\mathscr{O}_{\mathbb{P}^n(k)}(U)=\left\{f \in \operatorname{Map}(U, k) ; \forall i \in\{0, \ldots, n\}: f_{\mid U \cap U_i} \in \mathscr{O}_{U_i}\left(U \cap U_i\right)\right\} .
$$
By Corollary\,\ref{Corollary:regular function,space with functions}, we have 
\begin{align*}
     \mathscr{O}_{\mathbb{P}^n(k)}(U)=\{f: U \rightarrow k : &\forall x \in U \exists x \in V \subseteq U \text { open and } \\ 
     &g, h \in k\left[X_0, \ldots, X_n\right] \text{ homogeneous of the same degree}  \\
     &\text{such that } h(v) \neq 0 \text{ and } f(v)=\frac{g(v)}{h(v)}, \forall v \in V \}.
\end{align*}

\end{defn}
\begin{prop}
    Let $i \in\{0, \ldots, n\}$. The bijection $U_i \xrightarrow{\sim} \mathbb{A}^n(k)$ induces an isomorphism
    $$
    \left(U_i, \mathscr{O}_{\mathbb{P}^n(k)} \mid U_i\right) \xrightarrow{\sim} \mathbb{A}^n(k) .
    $$
    of spaces with functions. The space with functions $\left(\mathbb{P}^n(k), \mathscr{O}_{\mathbb{P}^n(k)}\right)$ is a prevariety.
\end{prop}
\begin{prooff}
    If $U$ open in $U_i$, it suffice to show $\mathscr{O}_{\mathbb{P}^n(k)}(U)=\mathscr{O}_{U_i}(U)$.
\end{prooff}
\begin{prop}
    The only global functions on $\mathbb{P}^n(k)$ 
    are the constant functions, i.e., $$\mathscr{O}_{\mathbb{P}^n(k)}\left(\mathbb{P}^n(k)\right)=k$$. In particular, $\mathbb{P}^n(k)$ is not an affine variety for $n \geq 1$.
\end{prop}
\begin{defn}
    A prevariety is called a projective variety if is isomorphic to a closed 
    subprevariety of a projective space $\mathbb{P}^n(k)$.
\end{defn}
\begin{defn}[affine cone]
    Affine algebraic sets $X \subseteq \mathbb{A}^{n+1}(k)$ are called affine cones if for all $x \in X$ we have $\lambda x \in X$ for all $\lambda \in k^{\times}$.
\end{defn}
\begin{lem}
    A ideal $I$ in $k[x_0,\dots,x_n]$ generated by homogeneous elements iff for each $g\in I$, its 
    homogeneous components are in $I$.
\end{lem}
\begin{prop}
    Define for homogeneous polynomials $f_1, \ldots, f_m \in k\left[X_0, \ldots, X_n\right]$ (not necessarily of the same degree) the vanishing set
    $$
    V_{+}\left(f_1, \ldots, f_m\right)=\left\{\left(x_0: \ldots: x_n\right) \in \mathbb{P}^n(k) ; \forall j: f_j\left(x_0, \ldots, x_n\right)=0\right\}
    $$
    Since 
    $$
    V_{+}\left(f_1, \ldots, f_m\right) \cap U_i=V\left(\Phi_i\left(f_1\right), \ldots, \Phi_i\left(f_m\right)\right)
    $$
    we have $V_{+}\left(f_1, \ldots, f_m\right)^c\cap U_i$ is open for all $i=0,\dots,n$. Hence 
    $V_{+}\left(f_1, \ldots, f_m\right)$ is closed.
\end{prop}
\begin{prop}
    Let $X \subseteq \mathbb{A}^{n+1}(k)$ be an affine algebraic set such that $X \neq\{0\}$. $f$ is defined to be   
    $$
    f: \mathbb{A}^{n+1}(k)-\{0\} \rightarrow \mathbb{P}^n(k), \quad\left(x_0, \ldots, x_n\right) \mapsto\left(x_0: \cdots: x_n\right)
    $$
    which is a morphism of prevarieties. Then the following assertions are equivalent:
    \begin{enu} 
    \item $X$ is an affine cone.
    \item $I(X)$ is generated by homogeneous polynomials.
    \item There exists a closed subset $Z \subseteq \mathbb{P}^n(k)$ such that $X=\overline{f^{-1}(Z)}=C(Z)$.
    \end{enu}
\end{prop}
\begin{prooff}
    Assume $X$ is non-empty. Hence there's $0\neq x\in X$. 

    (3) implies (1): $f^{-1}(Z)=\overline{f^{-1}(Z)}\cap(\mathbb{A}^{n+1}(k)-\bbrace{0})=X\cap (\mathbb{A}^{n+1}(k)-\bbrace{0})$
    
    (1) implies (2): To show that $I(X)$ is generated by homogeneous elements, let $g \in I(X)$ and write $g=\sum_d g_d$, where $g_d$ is homogeneous of degree $d$. As $X$ is an affine cone, we have $g(\lambda x)=0$ for all $x=\left(x_0, \ldots, x_n\right) \in X$ and $\lambda \in k^{\times}$. If there existed $g_d \notin I(X)$, 
    we would find $x \in X$ such that $g_d(x) \neq 0$. Then $\sum_d g_d(x) T^d$ is not the zero polynomial and 
    there exists a $\lambda \in k^{\times}$ with
    $$
    0 \neq \sum_d g_d(x) \lambda^d=\sum_d g_d(\lambda x)=g(\lambda x)=0
    $$
    Contradiction!

    (2) implies (3): If $I(X)$ generated by $(f_1,\dots,f_m)$, where $f_i$ be homogeneous polynomials with degree $\ge 1$(otherwise, $X=\varnothing$), 
    then $0\in X=V(f_1,\dots,f_2)$. Hence $X\cap \mathbb{A}^{n+1}(k)-\{0\}=f^{-1}(V_{+}(f_1,\dots,f_n))$. It suffices to show $f^{-1}(V_{+}(f_1,\dots,f_2))$
    is not closed. 
    
    Take $0\neq x\in X$. If there's $g$ such that $0\in D(g)$ such that $D(g)\cap f^{-1}(V_{+}(f_1,\dots,f_2))=\varnothing$.
    For infinite many $\lambda\in k^*$, we have 
    $$
     \sum_d g_d(x) \lambda^d=\sum_d g_d(\lambda x)=g(\lambda x)=0
    $$
    Hence $g_d(x)=0$ for all $d\ge 0$, which contradicts to the constant term 
    $g_0(x)\neq 0$.
\end{prooff}
\begin{prop}
    A closed subset in $\bb{P}^n(k)$ is of the form $V_{+}(f_1,\dots,f_m)$ where $f_i$ are homogeneous polynomial.
\end{prop}
\begin{prooff}
    If $Z$ closed in $\bb{P}(k)$, $\overline{f^{-1}(Z)}=f^{-1}(Z)+0 =V(f_1,\dots,f_n)$. Hence, $Z=V_{+}(f_1,\dots,f_n)$. 

\end{prooff}
\begin{defn}[Linear Subspace]
    For $m \geq-1$ let $\varphi: k^{m+1} \rightarrow k^{n+1}$ be an injective homomorphism of $k$-vector spaces. It maps one-dimensional subspaces of $k^{m+1}$ to one-dimensional subspaces of $k^{n+1}$ and we obtain an injective morphism $\iota: \mathbb{P}^m(k) \rightarrow \mathbb{P}^n(k)$ of prevarieties. This is in fact an isomorphism of $\mathbb{P}^m(k)$ onto a closed subprevariety of $\mathbb{P}^n(k)$ : If $A=\left(a_{i j}\right) \in M_{\ell \times(n+1)}(k)$ is a matrix such that $\operatorname{Ker} A=\operatorname{im} \varphi$, then $\iota$ defines an isomorphism of $\mathbb{P}^m(k)$ with $V_{+}\left(f_1, \ldots, f_{\ell}\right)$, where $f_i=\sum_{j=0}^n a_{i j} X_j \in k\left[X_0, \ldots, X_n\right]$.
    Closed subprevarieties of this form are called linear subspaces of $\mathbb{P}^n(k)$ of dimension $m$.
\end{defn}
\begin{defn}
    The only linear subspace of dimension -1 is the empty set, the linear subspaces of dimension 0 are the points. Linear subspaces in $\mathbb{P}^n(k)$ of dimension 1 (resp. 2, resp. $n-1$ ) are called lines (resp. planes, resp. hyperplanes).
\end{defn}
\begin{prop} 
    For every two points $p \neq q \in \mathbb{P}^n(k)$ there exists a unique line in $\mathbb{P}^n(k)$ that contains $p$ and $q$. This is clear, because two different one-dimensional subspaces of $k^{n+1}$ are contained in a unique two-dimensional subspace. We denote this line by $\overline{p q}$.
\end{prop}
\begin{prop}
    Two different lines in $\mathbb{P}^2(k)$ always intersect in a unique point: Lines in $\mathbb{P}^2(k)$ correspond to two-dimensional subspaces in $k^3$ and any two different two-dimensional subspaces in $k^3$ meet in a unique one-dimensional subspace which corresponds to a point in $\mathbb{P}^2(k)$.
\end{prop}
\begin{defn}
    A prevariety is called quasi-projective variety if it is isomorphic to an open subvariety of a projective variety.
\end{defn}
\begin{prop}
    Quasi-projective varieties are those that are of the form $\left(Y, \mathscr{O}_Y\right)$, where $Y \subseteq \mathbb{P}^n(k)$ is a locally closed subspace and where $\mathscr{O}_Y=\left.\mathscr{O}_{X}\right|_Y$ for a closed subvariety $X$ of $\mathbb{P}^n(k)$ such that $Y$ is open in $X$. The structure of a prevariety depends only on $Y$ and not on the choice of $X$
\end{prop}
\begin{prop}[morphism from quasi-projective variety to $\bb{P}^m(k)$ ]
    Let $Y \subseteq \mathbb{P}^n(k)$ be a quasi-projective variety.
\begin{enu} 
\item Let $f_0, \ldots, f_m \in k\left[X_0, \ldots, X_n\right]$ be homogeneous polynomials of the same degree such that for all $y=\left(y_0: \ldots: y_n\right) \in Y$ there exists an index $j$ such that $f_j(y) \neq 0$. Then
$$
h: Y \rightarrow \mathbb{P}^m(k), \quad y \mapsto\left(f_0(y): \ldots: f_m(y)\right)
$$
is a morphism of prevarieties. Another family $g_0, \ldots, g_m \in k\left[X_0, \ldots, X_n\right]$ as above defines the same morphism $h$ if and only if $f_i(y) g_j(y)=f_j(y) g_i(y)$ for all $y \in Y$ and all $i, j \in\{0, \ldots, m\}$.

\item Conversely, let $h: Y \rightarrow \mathbb{P}^m(k)$ be a morphism of prevarieties. Then there exists for every $y \in Y$ an open neighborhood $U$ of $y$ in $Y$ such that $h_{\mid U}$ is of the above form.
\end{enu}
\end{prop}
\begin{prooff}
    (2): Consider functions $f_j: U_i\subset \mathbb{P}^m(k)\rightarrow k: [x_0,\dots,x_n]\mapsto x_j/x_i$
    which are in $\mathscr{O}_{\mathbb{P}^m(k)}(U_i)$. 

\end{prooff}
\begin{defn}
    Let $H \subset \mathbb{P}^n(k)$ be a hyperplane and let $p \in \mathbb{P}^n(k) \backslash H$ be a point. Let $X \subseteq H$ be a closed subvariety. We define the cone $\overline{X, p}$ of $X$ over $p$ by
    $$
    \overline{X, p}=\bigcup_{q \in X} \overline{q p}
    $$
    This is a closed subvariety of $\mathbb{P}^n(k)$ : Indeed, after a change of coordinates we may assume $H=V_{+}\left(X_n\right)$ and $p=(0: \ldots: 0: 1)$. Then we have
    $$
    X=V_{+}\left(f_1, \ldots, f_m\right) \subseteq \mathbb{P}^{n-1}(k)=H \quad \text { for } f_i \in k\left[X_0, \ldots, X_{n-1}\right]
    $$
\end{defn}
\begin{prop}[projective version of Hilbert Nullstellensatz]
    Let $\mathfrak{a} \subseteq k\left[X_0, \ldots, X_n\right]$ be a homogeneous ideal (Exercise 1.20) and set

    $$
    V_{+}(\mathfrak{a}):=\left\{x \in \mathbb{P}^n(k) ; f(x)=0 \text { for all homogeneous } f \in \mathfrak{a}\right\}
    $$
    
    
    Show that the maps $\mathfrak{a} \mapsto V_{+}(\mathfrak{a})$ and $Z \mapsto C(Z)$ define bijections between the following sets.
\begin{enu} 
    \item The set of homogeneous radical ideals $\mathfrak{a} \subseteq k\left[X_0, \ldots, X_n\right]$ with $\mathfrak{a} \neq\left(X_0, \ldots, X_n\right)$.
    \item The set of closed subspaces $Z$ of $\mathbb{P}^n(k)$.
    \item The set of closed affine cones $C \subseteq \mathbb{A}^{n+1}(k)$ such that $C \neq\{0\}$. 
    If $Z \subseteq \mathbb{P}^n(k)$ is a closed subset we denote by $I_{+}(Z)$ the corresponding homogeneous ideal. Show that $I_{+}(Z)=I(C(Z))$ and deduce that the following assertions are equivalent.
\end{enu}
\begin{enu} 
    \item $Z$ is irreducible.
    \item $I_{+}(Z)$ is a prime ideal.
    \item $C(Z)$ is irreducible.  
\end{enu}
% https://q.uiver.app/#q=WzAsMyxbMCwwLCJcXHRleHR7SG9tb2dlbm91cyByYWFkaWNhbCBpZGVhbH1cXG5lcSAoWF8wLFxcZG90cyxYX24pIl0sWzEsMSwiXFx0ZXh0e2Nsb3NlZCBzdWJzZXQgb2YgfVxcbWF0aGJie1B9Xm4oaykiXSxbMCwyLCJcXHRleHR7YWZmaW5lIGNvbmVzfVxcbmVxIFxcbGVmdFxcezBcXHJpZ2h0XFx9Il0sWzAsMSwiVl97K30iLDAseyJjdXJ2ZSI6LTF9XSxbMSwwLCJJX3srfSIsMCx7ImN1cnZlIjotMn1dLFsxLDIsIkM9XFxvdmVybGluZXtmXnstMX19Il0sWzIsMCwiSSJdXQ==
\[\begin{tikzcd}
	{\text{homogenous radical ideal}\neq (X_0,\dots,X_n)} \\
	& {\text{closed subset of }\mathbb{P}^n(k)} \\
	{\text{affine cones}\neq \left\{0\right\}}
	\arrow["{V_{+}}", curve={height=-6pt}, from=1-1, to=2-2]
	\arrow["{I_{+}}", curve={height=-12pt}, from=2-2, to=1-1]
	\arrow["{C=\overline{f^{-1}}}", from=2-2, to=3-1]
	\arrow["I", from=3-1, to=1-1]
\end{tikzcd}\] 
\end{prop}
\begin{prop}
    Let $Y, Z$ be linear subspaces of $\mathbb{P}^n(k)$. Show that $Y \cap Z$ is again a linear subspace of dimension $\geq \operatorname{dim}(Y)+\operatorname{dim}(Z)-n$. Deduce that $Y \cap Z$ is always non-empty if $\operatorname{dim}(Y)+\operatorname{dim}(Z) \geq n$.
\end{prop}
\begin{prooff}
    Assume $\varphi_1:k^{s+1}\rightarrow k^{n+1}$ and $\varphi_2:k^{t+1}\rightarrow k^{n+1}$ be injective linear maps. Then consider 
    $\text{id}: \text{im}\varphi_1\cap\text{im}\varphi_2\rightarrow k^{n+1}$.
\end{prooff}
\begin{defn}[nilpotent cone]
    We identify the space $M_2(k)$ of $2 \times 2$-matrices over $k$ with $\mathbb{A}^4(k)$ (with coordinates $a, b, c, d)$. A matrix $A=\left(\begin{array}{ll}a & b \\ c & d\end{array}\right) \in M_2(k)$ is nilpotent if $A^2=0$ or, equivalently, if its determinant and trace are zero. Thus if
    $$
    \mathfrak{a}:=\left(a^2+b c, d^2+b c,(a+d) b,(a+d) c\right), \quad \mathfrak{b}=(a d-b c, a+d)
    $$
    we have that
    $$
    V(\mathfrak{a})=V(\mathfrak{b})=\left\{A \in M_2(k) ; A \text { nilpotent }\right\}
    $$
    Then $\operatorname{rad} \mathfrak{a}=\mathfrak{b}$( since $k[a,b,c,d]/(\mathfrak{b})=k[b,c,d]/(d^2+bc)$ ) and that $0\neq V(\mathfrak{b})$ is an irreducible closed affine cone in $M_2(k)=\mathbb{A}^4(k)$ (the so-called nilpotent cone).
\end{defn}
\begin{defn}[function field]
    Let $X$ be a prevariety. Define on the set of all pairs $(U, f)$, where $\emptyset \neq U \subseteq X$ open and $f \in \Gamma(U)$, an equivalence relation by setting $(U, f) \sim(V, g)$ 
    if there exists $\emptyset \neq W \subseteq U \cap V$ open such that $f_{\mid W}=g_{\mid W}$. 
    Denote the set of equivalence classes by $K(X)$.
\end{defn}
\begin{prop}
    $K(X)$ is a field.
\end{prop}
\begin{prooff} 
For $f\in \mathscr{O}_X(U)$, if $f\neq 0$, take $x\in U $ such that $f(x)\neq 0$. 
Assume 
$$
X=\bigcup_{i=1}^n U_i
$$ 
where $U_i$ be affine open subsets. Assume $x\in U_i$, 
then for some open subset $V$ of $U\cap U_i$, $\left.f\right|_V=g/h$ with $g,h\neq 0 $ in $V$. 
Hence $(U,f)$ is invertible.
\end{prooff}
\begin{prop}
    $X$ be a prevariety, $U$ be an affine open subset, then $\mathscr{O}_X(U)$ is an integral $k$-algebra. Since $U$ is irreducible, 
    the map 
    \begin{equation*}
        \mathscr{O}_X(U)\rightarrow K(X), f\mapsto (U,f)
    \end{equation*}
    is injective. Then by Corollary\,\ref{Corollary:regular function,space with functions}, the induced map 
    \begin{equation*}
        \text{Frac}(\mathscr{O}_X(U))\rightarrow K(X) 
    \end{equation*}
    is an isomorphism.
\end{prop}

\end{comment}
\chapter{Theory of Scheme}
\section{Sheaf Theory}
\begin{defn}[presheaf]
    Let $\left(\operatorname{Ouv}_X\right)$ be the category whose objects are the open sets of $X$ and, for two open sets $U, V \subseteq X, \operatorname{Hom}(U, V)$ is empty if $U \nsubseteq V$, and consists of the inclusion map $U \rightarrow V$ if $U \subseteq V$ (composition of morphisms being the composition of the inclusion maps).
    A presheaf is a contravariant functor $\mathscr{F}$ from the category $\left(\mathrm{Ouv}_X\right)$ to the category of category $\mathcal{C}$(such as the category of abelian groups, the category of rings, the category of $R$-modules, or the category of $R$-algebras)
\end{defn}
\begin{defn}
    Let $\mathscr{F}$ be a presheaf on a topological space $X$, let $U$ be an open set in $X$ and let $\mathscr{U}=\left(U_i\right)_{i \in I}$ be an open covering of $U$. We define maps (depending on $\mathscr{U}$ )
    $$
        \begin{gathered}
            \rho: \mathscr{F}(U) \rightarrow \prod_{i \in I} \mathscr{F}\left(U_i\right), \quad s \mapsto\left(s_{\mid U_i}\right)_i \\
            \sigma: \prod_{i \in I} \mathscr{F}\left(U_i\right) \rightarrow \prod_{(i, j) \in I \times I} \mathscr{F}\left(U_i \cap U_j\right), \quad\left(s_i\right)_i \mapsto\left(s_{i \mid U_i \cap U_j}\right)_{(i, j)}, \\
            \sigma^{\prime}: \prod_{i \in I} \mathscr{F}\left(U_i\right) \rightarrow \prod_{(i, j) \in I \times I} \mathscr{F}\left(U_i \cap U_j\right), \quad\left(s_i\right)_i \mapsto\left(s_{j \mid U_i \cap U_j}\right)_{(i, j)} .
        \end{gathered}
    $$

    The presheaf $\mathscr{F}$ is called a sheaf, if it satisfies for all $U$ and all coverings $\left(U_i\right)$ as above the following condition:

    % https://q.uiver.app/#q=WzAsMyxbMCwwLCIgXFxtYXRoc2Nye0Z9KFUpICJdLFsxLDAsIlxccHJvZFxcbGltaXRzX3tpIFxcaW4gSX0gXFxtYXRoc2Nye0Z9XFxsZWZ0KFVfaVxccmlnaHQpIl0sWzIsMCwiXFxwcm9kXFxsaW1pdHNfeyhpLCBqKSBcXGluIEkgXFx0aW1lcyBJfSBcXG1hdGhzY3J7Rn1cXGxlZnQoVV9pIFxcY2FwIFVfalxccmlnaHQpIl0sWzAsMSwiXFxyaG8iXSxbMSwyLCJcXHNpZ21hIiwwLHsib2Zmc2V0IjotMn1dLFsxLDIsIlxcc2lnbWFee1xccHJpbWV9IiwyLHsib2Zmc2V0IjoxfV1d
    \[\begin{tikzcd}
            { \mathscr{F}(U) } & {\prod\limits_{i \in I} \mathscr{F}\left(U_i\right)} & {\prod\limits_{(i, j) \in I \times I} \mathscr{F}\left(U_i \cap U_j\right)}
            \arrow["\rho", from=1-1, to=1-2]
            \arrow["\sigma", shift left=2, from=1-2, to=1-3]
            \arrow["{\sigma^{\prime}}"', shift right, from=1-2, to=1-3]
        \end{tikzcd}\]
    is exact. This means that the map $\rho$ is injective and that its image is the set of elements $\left(s_i\right)_{i \in I} \in \prod_{i \in I} \mathscr{F}\left(U_i\right)$ such that $\sigma\left(\left(s_i\right)_i\right)=\sigma^{\prime}\left(\left(s_i\right)_i\right)$.

    In other words, a presheaf $\mathscr{F}$ is a sheaf if and only if for all open sets $U$ in $X$ and every open covering $U=\bigcup_i U_i$ the following two conditions hold:
    \begin{enu}
        \item (Sh1) Let $s, s^{\prime} \in \mathscr{F}(U)$ with $s_{\mid U_i}=s^{\prime}{ }_{\mid U_i}$ for all $i$. Then $s=s^{\prime}$.
        \item (Sh2) Given $s_i \in \mathscr{F}\left(U_i\right)$ for all $i$ such that $s_{i \mid U_i \cap U_j}=s_{j \mid U_i \cap U_j}$ for all $i, j$. Then there exists an $s \in \mathscr{F}(U)$ such that $s_{\mid U_i}=s_i$ (note that $s$ is unique by (Sh1)).
    \end{enu}
    \begin{defn}[restriction of sheaf]
        If $\mathscr{F}$ is a presheaf on a topological space $X$ and $U$ is an open subspace of $X$,
        we obtain a presheaf $\mathscr{F}\mid_U$ on $U$ by setting $\mathscr{F}\mid_U(V)=\mathscr{F}(V)$ for every open subset $V$ in $U$.
        If $\mathscr{F}$ is a sheaf, $\mathscr{F}\mid_U$ is a sheaf on $U$. We call $\mathscr{F} \mid_U$ the restriction of $\mathscr{F}$ to $U$.
    \end{defn}
\end{defn}
\begin{defn}
    The inductive limit
    $$
        \mathscr{F}_x:=\underset{\overrightarrow{U \ni x}}{\lim } \mathscr{F}(U)
    $$
    is called the stalk of $\mathscr{F}$ in $x$.
    In other words, $\mathscr{F}_x$ is the set of equivalence classes of pairs $(U, s)$, where $U$ is an open neighborhood of $x$ and $s \in \mathscr{F}(U)$. Here two such pairs $\left(U_1, s_1\right)$ and $\left(U_2, s_2\right)$ are equivalent, if there exists an open neighborhood $V$ of $x$ with $V \subseteq U_1 \cap U_2$ such that $s_{1 \mid V}=s_{2 \mid V}$.
    For each open neighborhood $U$ of $x$ we have a canonical map
    $$
        \mathscr{F}(U) \rightarrow \mathscr{F}_x, \quad s \mapsto s_x
    $$
    which sends $s \in \mathscr{F}(U)$ to the class of $(U, s)$ in $\mathscr{F}_x$. We call $s_x$ the germ of $s$ in $x$.
    If $\varphi: \mathscr{F} \rightarrow \mathscr{G}$ is a morphism of presheaves on $X$, we have an induced map
    $$
        \mathscr{F}_x\rightarrow \mathscr{G}_x
    $$
    of the stalks in $x$ by Proposition~\ref{proposition:morphism induced by direct system}. We obtain a functor $\mathscr{F} \mapsto \mathscr{F}_x$ from the category of presheaves on $X$ to the category of sets.

    If $\mathscr{F}$ is a presheaf with values in $\mathcal{C}$, where $\mathcal{C}$ is the category of abelian groups, of rings, or any category in which filtered inductive limits exist, then the stalk $\mathscr{F}_x$ is an object in $\mathcal{C}$ and we obtain a functor $\mathscr{F} \mapsto \mathscr{F}_x$ from the category of presheaves on $X$ with values in $\mathcal{C}$ to the category $\mathcal{C}$.
\end{defn}
\begin{prop}
    Let $X$ be a topological space, $\mathscr{F}$ and $\mathscr{G}$ presheaves on $X$, and let $\varphi, \psi: \mathscr{F} \rightarrow \mathscr{G}$ be two morphisms of presheaves.
    \begin{enu}
        \item The induced maps on stalks $\varphi_x: \mathscr{F}_x \rightarrow \mathscr{G}_x$ are injective for all $x \in X$ if $\varphi_U: \mathscr{F}(U) \rightarrow \mathscr{G}(U)$ is injective for all open subsets $U \subseteq X$.
        \item Assume that $\mathscr{F}$ is a sheaf. Then the induced maps on stalks $\varphi_x: \mathscr{F}_x \rightarrow \mathscr{G}_x$ are injective for all $x \in X$ \blue{if and only if} $\varphi_U: \mathscr{F}(U) \rightarrow \mathscr{G}(U)$ is injective for all open subsets $U \subseteq X$.
        \item If $\mathscr{F}$ and $\mathscr{G}$ are both sheaves, the maps $\varphi_x$ are bijective for all $x \in X$ if and only if $\varphi_U$ is bijective for all open subsets $U \subseteq X$.
        \item If $\mathscr{F}$ and $\mathscr{G}$ are both sheaves, the morphisms $\varphi$ and $\psi$ are equal if and only if $\varphi_x=\psi_x$ for all $x \in X$.
    \end{enu}
    \label{proposition:characterizations of morphism between sheaves}
\end{prop}
\begin{prooff}
    For $U \subseteq X$ open consider the map
    $$
        \mathscr{F}(U) \rightarrow \prod_{x \in U} \mathscr{F}_x, \quad s \mapsto\left(s_x\right)_{x \in U}
    $$

    We claim that this map is injective if $\mathscr{F}$ is a sheaf. Indeed let $s, t \in \mathscr{F}(U)$ such that $s_x=t_x$ for all $x \in U$. Then for all $x \in U$ there exists an open neighborhood $V_x \subseteq U$ of $x$ such that $s_{\mid V_x}=t_{\mid V_x}$. Clearly, $U=\bigcup_{x \in U} V_x$ and therefore $s=t$ by sheaf condition (Sh1).
    Using the commutative diagram
    % https://q.uiver.app/#q=WzAsNCxbMCwwLCJcXG1hdGhzY3J7Rn0oVSkiXSxbMSwwLCJcXHByb2RcXG1hdGhzY3J7Rn1feCJdLFsxLDEsIlxccHJvZFxcbWF0aHNjcntHfV94Il0sWzAsMSwiXFxtYXRoc2Nye0d9KFUpIl0sWzAsMV0sWzEsMiwiXFxwcm9kXFx2YXJwaGlfeCJdLFswLDMsIlxcdmFycGhpX1UiXSxbMywyXV0=
    \[\begin{tikzcd}
            {\mathscr{F}(U)} & {\prod\mathscr{F}_x} \\
            {\mathscr{G}(U)} & {\prod\mathscr{G}_x}
            \arrow[from=1-1, to=1-2]
            \arrow["{\prod\varphi_x}", from=1-2, to=2-2]
            \arrow["{\varphi_U}", from=1-1, to=2-1]
            \arrow[from=2-1, to=2-2]
        \end{tikzcd}\]
    and Proposition~\ref{proposition:exact sequnce induced by direct system}, (1) and (3) hold.

    (2): By proposition~\ref{proposition:exact sequnce induced by direct system},  it suffice to show the bijectivity of $\varphi_x$ for all $x \in U$ implies the surjectivity of $\varphi_U$. Let $t \in \mathscr{G}(U)$. For all $x \in U$ we choose an open neighborhood $U^x$ of $x$ in $U$ and $s^x \in \mathscr{F}\left(U^x\right)$ such that $\left(\varphi_{U^x}\left(s^x\right)\right)_x=t_x$. Then there exists an open neighborhood $V^x \subseteq U^x$ of $x$ with $\varphi_{V^x}\left(s^x{ }_{\mid V^x}\right)=t_{\mid V^x}$. Then $\left(V^x\right)_{x \in U}$ is an open covering of $U$ and for $x, y \in U$
    $$
        \varphi_{V^x \cap V^y}\left(s^x{ }_{\mid V^x \cap V^y}\right)=t_{\mid V^x \cap V^y}=\varphi_{V^x \cap V^y}\left(s^y \mid V^x \cap V^y\right) .
    $$

    As we already know that $\varphi_{V^x \cap V^y}$ is injective, this shows $s^x\left|V^x \cap V^y=s^y\right| V^x \cap V^y$ and the sheaf condition (Sh2) ensures that we find $s \in \mathscr{F}(U)$ such that $s_{\mid V^x}=s^x{ }_{\mid V^x}$ for all $x \in U$. Clearly, we have $\varphi_U(s)_x=t_x$ for all $x \in U$ and hence $\varphi_U(s)=t$.
\end{prooff}
\begin{defn}
    A morphism $\varphi: \mathscr{F} \rightarrow \mathscr{G}$ of sheaves injective (resp. surjective, resp. bijective) if $\varphi_x: \mathscr{F}_x \rightarrow \mathscr{G}_x$ is injective (resp. surjective, resp. bijective) for all $x \in X$.
\end{defn}
\begin{rema}
    If $\varphi: \mathscr{F} \rightarrow \mathscr{G}$ is a morphism of sheaves, $\varphi$ is surjective if and only if for all open subsets $U \subseteq X$ and every $t \in \mathscr{G}(U)$ there exist an open covering $U=\bigcup_i U_i$ (depending on $t$ ) and sections $s_i \in \mathscr{F}\left(U_i\right)$ such that $\varphi_{U_i}\left(s_i\right)=t_{\mid U_i}$, i.e., locally we can find a preimage of $t$. But the surjectivity of $\varphi$ does not imply that $\varphi_U: \mathscr{F}(U) \rightarrow \mathscr{G}(U)$ is surjective for all open sets $U$ of $X$
\end{rema}
\begin{defn}
    If $\mathscr{F}, \mathscr{G}$ are (pre-)sheaves on $X$ such that $\mathscr{F}(U) \subseteq \mathscr{G}(U)$ for all $U \subseteq X$ open, and such that the following diagram commute
    % https://q.uiver.app/#q=WzAsNCxbMCwwLCJcXG1hdGhzY3J7Rn0oVSkiXSxbMSwwLCJcXG1hdGhzY3J7R30oVSkiXSxbMCwxLCJcXG1hdGhzY3J7Rn0oVikiXSxbMSwxLCJcXG1hdGhzY3J7R30oVikiXSxbMCwxLCJcXHN1YnNldCJdLFsyLDAsIlxcdGV4dHtyZXN9X1VeViJdLFsyLDMsIlxcc3Vic2V0IiwyXSxbMywxLCJcXHRleHR7cmVzfV9VXlYiLDJdXQ==
    \[\begin{tikzcd}
            {\mathscr{F}(U)} & {\mathscr{G}(U)} \\
            {\mathscr{F}(V)} & {\mathscr{G}(V)}
            \arrow["\subset", from=1-1, to=1-2]
            \arrow["{\text{res}_U^V}", from=2-1, to=1-1]
            \arrow["\subset"', from=2-1, to=2-2]
            \arrow["{\text{res}_U^V}"', from=2-2, to=1-2]
        \end{tikzcd}\]
    we call $\mathscr{F}$ sub(pre-)sheaf of $\mathscr{G}$.
\end{defn}
\begin{defn}[sheafification]
    Let $\mathscr{F}$ be a presheaf on a topological space $X$. 
    Then there exists a $\operatorname{pair}\left(\tilde{\mathscr{F}}, \iota_{\mathscr{F}}\right)$, where $\tilde{\mathscr{F}}$ is a sheaf on $X$ and $\iota_{\mathscr{F}}: \mathscr{F} \rightarrow \tilde{\mathscr{F}}$ is a morphism of presheaves, such that the following holds: If $\mathscr{G}$ is a sheaf on $X$
    and $\varphi: \mathscr{F} \rightarrow \mathscr{G}$ is a morphism of presheaves, then there exists a unique morphism of sheaves $\tilde{\varphi}: \tilde{\mathscr{F}} \rightarrow \mathscr{G}$ with $\tilde{\varphi} \circ \iota_\mathscr{F}=\varphi$.
    And the following properties hold:
    \begin{enu}
        \item For all $x \in X$ the map on stalks $\iota_{\mathscr{F}, x}: \mathscr{F}_x \rightarrow \tilde{\mathscr{F}}_x$ is bijective.
        \item For every presheaf $\mathscr{F},\mathscr{G}$ on $X$ and every morphism of presheaves $\varphi: \mathscr{F} \rightarrow \mathscr{G}$ there exists a unique morphism $\tilde{\varphi}: \tilde{\mathscr{F}}\rightarrow \tilde{\mathscr{G}}$ making the diagram
        % https://q.uiver.app/#q=WzAsNCxbMCwwLCIgXFxtYXRoc2Nye0Z9Il0sWzEsMCwiXFx0aWxkZXtcXG1hdGhzY3J7Rn19Il0sWzAsMSwiIFxcbWF0aHNjcntHfSJdLFsxLDEsIlxcdGlsZGV7XFxtYXRoc2Nye0d9fSJdLFswLDEsIlxcaW90YV97XFxtYXRoc2Nye0Z9fSJdLFswLDIsIlxcdmFycGhpIiwyXSxbMiwzLCJcXGlvdGFfe1xcbWF0aHNjcntHfX0iLDJdLFsxLDMsIlxcdGlsZGV7XFx2YXJwaGl9Il1d
        \[\begin{tikzcd}
                { \mathscr{F}} & {\tilde{\mathscr{F}}} \\
                { \mathscr{G}} & {\tilde{\mathscr{G}}}
                \arrow["{\iota_{\mathscr{F}}}", from=1-1, to=1-2]
                \arrow["\varphi"', from=1-1, to=2-1]
                \arrow["{\iota_{\mathscr{G}}}"', from=2-1, to=2-2]
                \arrow["{\tilde{\varphi}}", from=1-2, to=2-2]
            \end{tikzcd}\]
        commutative.
        %\item $\varphi$ is injective(surjective, bijective) if and only if $\tilde{\varphi}$ is injective(surjective, bijective).
    \end{enu}
    In particular, $\mathscr{F} \mapsto \tilde{\mathscr{F}}$ is a functor from the category of presheaves on $X$ to the category of sheaves on $X$.
    \label{proposition:sheafification}
\end{defn}
\begin{prooff}
    For $U \subseteq X$ open, elements of $\tilde{\mathscr{F}}(U)$ are by definition families of elements in the stalks of $\mathscr{F}$ which locally give rise to sections of $\mathscr{F}$. More precisely, we define
    \begin{align*}
        \tilde{\mathscr{F}}(U):=\left\{\left(s_x\right) \in \prod_{x \in U} \mathscr{F}_x : \forall x \in U, \exists  \text { an open neighborhood } W \subseteq U \text { of } x,\right. \\
        \text { and } \left.t \in \mathscr{F}(W) \text{ s.t. } \forall w \in W: s_w=t_w\right\} .
    \end{align*}
    For $U \subseteq V$ the restriction map $\tilde{\mathscr{F}}(V) \rightarrow \tilde{\mathscr{F}}(U)$ is induced by the natural projection $\prod_{x \in V} \mathscr{F}_x \rightarrow \prod_{x \in U} \mathscr{F}_x$. Then it is easy to check that $\tilde{\mathscr{F}}$ is a sheaf.

    For $U \subseteq X$ open, we define $\iota_{\mathscr{F}, U}: \mathscr{F}(U) \rightarrow \tilde{\mathscr{F}}(U)$ by $s \mapsto\left(s_x\right)_{x \in U}$. The definition of $\tilde{\mathscr{F}}$ shows that $\iota_{\mathscr{F}, x}: \mathscr{F}_x \rightarrow \tilde{\mathscr{F}}_x$ is bijective.

    Now let $\mathscr{G}$ be a presheaf on $X$ and let $\varphi: \mathscr{F} \rightarrow \mathscr{G}$ be a morphism. Sending $\left(s_x\right)_x \in \tilde{\mathscr{F}}(U)$ to $\left(\varphi_x\left(s_x\right)\right)_x \in \tilde{\mathscr{G}}(U)$ defines a morphism $\tilde{\mathscr{F}} \rightarrow \tilde{\mathscr{G}}$. By Proposition~\ref{proposition:characterizations of morphism between sheaves}, this is the unique morphism making the diagram commutative.

    If we assume in addition that $\mathscr{G}$ is a sheaf, then the morphism of sheaves $\iota_{\mathscr{G}}: \mathscr{G} \rightarrow \tilde{\mathscr{G}}$, which is bijective on stalks, is an isomorphism by Proposition~\ref{proposition:characterizations of morphism between sheaves}(3). Composing the morphism $\tilde{\mathscr{F}} \rightarrow \tilde{\mathscr{G}}$ with $\iota_{\mathscr{G}}^{-1}$, we obtain the morphism $\tilde{\varphi}: \tilde{\mathscr{F}} \rightarrow \mathscr{G}$. Finally, the uniqueness of $\left(\tilde{\mathscr{F}}, \iota_{\mathscr{F}}\right)$ is a formal consequence.
\end{prooff}
\begin{rema}
    For every presheaf $\mathscr{F},\mathscr{G}$ on $X$ and every morphism of presheaves $\varphi: \mathscr{F} \rightarrow \mathscr{G}$ 
    there exists a unique morphism $\tilde{\varphi}: \tilde{\mathscr{F}}\rightarrow \tilde{\mathscr{G}}$ making the diagram
        % https://q.uiver.app/#q=WzAsNCxbMCwwLCIgXFxtYXRoc2Nye0Z9Il0sWzEsMCwiXFx0aWxkZXtcXG1hdGhzY3J7Rn19Il0sWzAsMSwiIFxcbWF0aHNjcntHfSJdLFsxLDEsIlxcdGlsZGV7XFxtYXRoc2Nye0d9fSJdLFswLDEsIlxcaW90YV97XFxtYXRoc2Nye0Z9fSJdLFswLDIsIlxcdmFycGhpIiwyXSxbMiwzLCJcXGlvdGFfe1xcbWF0aHNjcntHfX0iLDJdLFsxLDMsIlxcdGlsZGV7XFx2YXJwaGl9Il1d
        \[\begin{tikzcd}
                { \mathscr{F}} & {\tilde{\mathscr{F}}} \\
                { \mathscr{G}} & {\tilde{\mathscr{G}}}
                \arrow["{\iota_{\mathscr{F}}}", from=1-1, to=1-2]
                \arrow["\varphi"', from=1-1, to=2-1]
                \arrow["{\iota_{\mathscr{G}}}"', from=2-1, to=2-2]
                \arrow["{\tilde{\varphi}}", from=1-2, to=2-2]
            \end{tikzcd}\]
        commutative.
    If in addition $\mathscr{G}$ is a sheaf and $\varphi_{U}$ 
    is injective for all $U$ open in $X$, we have $\iota_{\mathscr{F},U}$ is injective for all 
    $U$ open in $X$.
    \label{proposition: subsheaf, sheafification}
\end{rema}



\begin{defn}[direct image]
    Let $f: X \rightarrow Y$ be a continuous map of topological spaces. Let $\mathscr{F}$ be a presheaf on $X$. We define a presheaf $f_* \mathscr{F}$ on $Y$ by
    $$
        \left(f_* \mathscr{F}\right)(V)=\mathscr{F}\left(f^{-1}(V)\right)
    $$
    the restriction maps given by the restriction maps for $\mathscr{F}$. We call $f_* \mathscr{F}$ the direct image of $\mathscr{F}$ under $f$. 
    
    Whenever $\varphi: \mathscr{F}_1 \rightarrow \mathscr{F}_2$ is a morphism of presheaves, the family of maps $f_*(\varphi)_V:=\varphi_{f^{-1}(V)}$ for $V \subseteq Y$ open is a morphism $f_*(\varphi): f_* \mathscr{F}_1 \rightarrow f_* \mathscr{F}_2$. Therefore $f_*$ is a functor from the category of presheaves on $X$ to the category of presheaves on $Y$.
\end{defn}
\begin{prop}
    \begin{enu}
        \item If $\mathscr{F}$ is a sheaf on $X, f_* \mathscr{F}$ is a sheaf on $Y$. Therefore $f_*$ also defines a functor $f_*:(\operatorname{Sh}(X)) \rightarrow(\operatorname{Sh}(Y))$.
        \item If $g: Y \rightarrow Z$ is a second continuous map, there exists an identity $g_*\left(f_* \mathscr{F}\right)=(g \circ f)_* \mathscr{F}$ which is functorial in $\mathscr{F}$.
    \end{enu}
\end{prop}
\begin{defn}[inverse image]
    Let $f: X \rightarrow Y$ be a continuous map and let $\mathscr{G}$ be a presheaf on $Y$. Define a presheaf on $X$ by
    $$
        U \mapsto \varinjlim_{V \supseteq f(U)} \mathscr{G}(V),
    $$
    the restriction maps being induced by the restriction maps of $\mathscr{G}$ and the universal property of direct limit:
    % https://q.uiver.app/#q=WzAsNCxbMCwwLCJcXHZhcmluamxpbV97ViBcXHN1cHNldGVxIGYoVSl9IFxcbWF0aHNjcntHfShWKSJdLFsxLDEsIlxcbWF0aHNjcntHfShWXzEpIl0sWzEsMiwiXFxtYXRoc2Nye0d9KFZfMikiXSxbMiwwLCJcXHZhcmluamxpbV97ViBcXHN1cHNldGVxIGYoVyl9IFxcbWF0aHNjcntHfShWKSJdLFsxLDBdLFsxLDJdLFsyLDBdLFsxLDNdLFsyLDNdLFswLDMsIlxcdGV4dHtyZXN9XntVfV97V30iLDIseyJzdHlsZSI6eyJib2R5Ijp7Im5hbWUiOiJkYXNoZWQifX19XV0=
    \[\begin{tikzcd}
            {\varinjlim_{V \supseteq f(U)} \mathscr{G}(V)} && {\varinjlim_{V \supseteq f(W)} \mathscr{G}(V)} \\
            & {\mathscr{G}(V_1)} \\
            & {\mathscr{G}(V_2)}
            \arrow[from=2-2, to=1-1]
            \arrow[from=2-2, to=3-2]
            \arrow[from=3-2, to=1-1]
            \arrow[from=2-2, to=1-3]
            \arrow[from=3-2, to=1-3]
            \arrow["{\text{res}^{U}_{W}}"', dashed, from=1-1, to=1-3]
        \end{tikzcd}\]
    We denote this presheaf by $f^{+} \mathscr{G}$. Let $f^{-1} \mathscr{G}$ be the sheafification of $f^{+} \mathscr{G}$. We call $f^{-1} \mathscr{G}$ the inverse image of $\mathscr{G}$ under $f$.
\end{defn}
% \begin{prop}
%     If $X$ is an open subspace of $Y$, and $f:X\rightarrow Y$ is the inclusion map, $\mathscr{G}$ is a sheaf on $Y$, then $f^{-1}\mathscr{G}\simeq \mathscr{G}|_X$.
% \end{prop}
\begin{prop}
    $f^{-1}$ is a functor from categroy of presheaf on $Y$ to categroy of sheaf on $X$.
\end{prop}
\begin{prooff}
    If $\varphi:\mathscr{G}_1\rightarrow \mathscr{G}_2$ is a morphism of presheaf on $Y$, then $f^{-1}\varphi: f^{-1}\mathscr{G}_1\rightarrow  f^{-1}\mathscr{G}_1$ is induced by universal property of direct limit and Proposition~\ref{proposition:sheafification}.
\end{prooff}
\begin{prop}[stalks of inverse image]
    Notice that
    $$
        \left(f^{-1} \mathscr{G}\right)_x \cong\left(f^{+} \mathscr{G}\right)_x=\varinjlim_{x \in U}\left(f^{+}\mathscr{G}\right)(U)
    $$
    Since $f$ is continous, by uniqueness of direct limit, 
    $$
        \varinjlim_{x\in U}\varinjlim_{f(U)\subset V}\mathscr{G}(V)  \cong \varinjlim_{f(x)\in V}\mathscr{G}(V)
    $$

    \label{stalks of direct image}
\end{prop}
\begin{prooff}
    $$
    \varinjlim_{x\in U}\varinjlim_{f(U)\subset V}\mathscr{G}(V)  \cong \varinjlim_{f(x)\in V}\mathscr{G}(V)
    $$
    is given by 
    $[[s]],s\in\mathscr{G}(V)\rightarrow [s],s\in\mathscr{G}(V)$ since $f$ is continous.
\end{prooff}
\begin{prop}
    Now let $g: Y \rightarrow Z$ be a second continuous map and let $\mathscr{H}$ be a presheaf on $Z$. 
    By the definition of $f^{+}$ and $g^{+}$, 
    $f^{+}\left(g^{+} \mathscr{H}\right)\cong (g \circ f)^{+} \mathscr{H}$. By taking sheafification, 
    \begin{equation*}
        f^{-1}\left(g^{+} \mathscr{H}\right)\cong (g \circ f)^{-1} \mathscr{H}
    \end{equation*}
    Since there's natural morphism of sheaves $f^{-1}g^{+}\mathscr{H}\rightarrow f^{-1}(g^{-1})\mathscr{H}$ and the morphism at stalks are isomorphism, 
    we have 
    $$
        f^{-1}\left(g^{-1} \mathscr{H}\right) \cong f^{-1}\left(g^{+} \mathscr{H}\right)\cong (g \circ f)^{-1} \mathscr{H},
    $$
\end{prop}
\begin{theo}[adjoint pair $(f^{-1},f_*)$]
    Let $f: X \rightarrow Y$ be a continuous map, let $\mathscr{F}$ be a sheaf on $X$ and let $\mathscr{G}$ be a presheaf on $Y$. Then there is a bijection
    $$
        \begin{aligned}
            \operatorname{Hom}_{(\operatorname{Sh}(X))}\left(f^{-1} \mathscr{G}, \mathscr{F}\right) & \leftrightarrow \operatorname{Hom}_{(\operatorname{PreSh}(Y))}\left(\mathscr{G}, f_* \mathscr{F}\right), \\
            \varphi                                                                                 & \rightarrow \varphi^b,                                                                                       \\
            \psi^{\sharp}                                                                           & \leftarrow \psi
        \end{aligned}
    $$
    and $(f^{-1},f_*)$ is an adjoint pair between $\operatorname{PreSh}(Y)$ and $\operatorname{Sh}(X)$.

    \label{theorem:f-1,f_* is an adjoint pair}
\end{theo}
\begin{prooff}
    Let $\varphi: f^{-1} \mathscr{G} \rightarrow \mathscr{F}$ be a morphism of sheaves on $X$, and let $V \subseteq Y$ be open. Since $f\left(f^{-1}(V)\right) \subseteq V$, we have a map $\mathscr{G}(V) \rightarrow f^{+} \mathscr{G}\left(f^{-1}(V)\right)$, and we define $\varphi_V^b$ as the composition
    $$
        \mathscr{G}(V) \rightarrow f^{+} \mathscr{G}\left(f^{-1}(V)\right) \longrightarrow f^{-1} \mathscr{G}\left(f^{-1}(V)\right) \xrightarrow{\varphi_{f^{-1}(V)}} \mathscr{F}\left(f^{-1}(V)\right)=f_* \mathscr{F}(V) .
    $$

    Conversely, let $\psi: \mathscr{G} \rightarrow f_* \mathscr{F}$ be a morphism of presheaves on $Y$. To define the morphism $\psi^{\sharp}$ it suffices to define a morphism of presheaves $f^{+} \mathscr{G} \rightarrow \mathscr{F}$, which we call again $\psi^{\sharp}$. Let $U$ be open in $X$, and $s \in f^{+} \mathscr{G}(U)$. If $V$ is some open neighborhood of $f(U), U$ is contained in $f^{-1}(V)$. Let $V$ be such a neighborhood such that there exists $s_V \in \mathscr{G}(V)$ representing $s$. Then $\psi_V\left(s_V\right) \in f_* \mathscr{F}(V)=\mathscr{F}\left(f^{-1}(V)\right)$. Let $\psi_U^{\sharp}(s) \in \mathscr{F}(U)$ be the restriction of the section $\psi_V\left(s_V\right)$ to $U$.
\end{prooff}
\begin{prop}
    Let $f: X \rightarrow Y$ be a continuous map, let $\mathscr{F}$ be a sheaf on $X$ and let $\mathscr{G}$ be a presheaf on $Y$, and a morphism of presheaves $\psi: \mathscr{G} \rightarrow f_* \mathscr{F}$. Then for each $x \in X$, the map
    $$
        \psi_x^{\sharp}: \mathscr{G}_{f(x)} \cong \left(f^{-1} \mathscr{G}\right)_x \longrightarrow \mathscr{F}_x
    $$
    induced by $\psi^{\sharp}: f^{-1} \mathscr{G} \rightarrow \mathscr{F}$ on stalks can be described in terms of $\psi$ as follows:
    For every open neighborhood $V \subseteq Y$ of $f(x)$, we have maps
    $$
        \mathscr{G}(V) \xrightarrow{\psi_V} \mathscr{F}\left(f^{-1}(V)\right) \longrightarrow \mathscr{F}_x,
    $$
    and taking the inductive limit over all $V$ we obtain the map $\psi_x^{\sharp}: \mathscr{G}_{f(x)} \rightarrow \mathscr{F}_x$.
\end{prop}
\begin{prop}
    $X$ be a topological space, then category of sheaves on $X$ is an abelian category.
\end{prop}
\begin{prooff}
    Cokernel exists: If $\mathcal{F}, \mathcal{G}$ are sheaves and $\varphi: \mathcal{F} \rightarrow \mathcal{G}$ is a sheaf map, then coker $\varphi$ exists. 
% https://q.uiver.app/#q=WzAsNSxbMCwwLCJcXG1hdGhjYWx7Rn0iXSxbMSwwLCJcXG1hdGhjYWx7R30iXSxbMiwxLCJcXG1hdGhjYWx7R30vXFx0ZXh0e2ltfVxcbWF0aGNhbHtGfSJdLFszLDAsIihcXG1hdGhjYWx7R30vXFx0ZXh0e2ltfVxcbWF0aGNhbHtGfSleKyJdLFsyLDIsIlgiXSxbMCwxLCJcXHZhcnBoaSJdLFsxLDIsIlxccGkiXSxbMiwzLCJcXGlvdGEiXSxbMSw0LCJ1IiwyXSxbMSwzLCJcXGlvdGFcXGNpcmNcXHBpIl0sWzAsNCwiMCIsMl0sWzIsNCwiIiwyLHsic3R5bGUiOnsiYm9keSI6eyJuYW1lIjoiZG90dGVkIn19fV0sWzMsNCwiXFx0aGV0YSIsMCx7InN0eWxlIjp7ImJvZHkiOnsibmFtZSI6ImRvdHRlZCJ9fX1dXQ==
\[\begin{tikzcd}
	{\mathcal{F}} & {\mathcal{G}} && {(\mathcal{G}/\text{im}\mathcal{F})^+} \\
	&& {\mathcal{G}/\text{im}\mathcal{F}} \\
	&& X
	\arrow["\varphi", from=1-1, to=1-2]
	\arrow["0"', from=1-1, to=3-3]
	\arrow["{\iota\circ\pi}", from=1-2, to=1-4]
	\arrow["\pi", from=1-2, to=2-3]
	\arrow["u"', from=1-2, to=3-3]
	\arrow["\theta", dotted, from=1-4, to=3-3]
	\arrow["\iota", from=2-3, to=1-4]
	\arrow[dotted, from=2-3, to=3-3]
\end{tikzcd}\]
Here $+$ denotes the sheafification and $\theta$ is induced by universal property of sheafification.

Ab2: 
% https://q.uiver.app/#q=WzAsOSxbMSwxLCJcXG1hdGhjYWx7Rn0iXSxbMiwxLCJcXG1hdGhjYWx7R30iXSxbMCwxLCJcXHRleHR7S2VyfVxcdmFycGhpIl0sWzEsMiwiXFxtYXRoY2Fse0Z9L1xcdGV4dHtLZXJ9XFx2YXJwaGkiXSxbMSwzLCIoXFxtYXRoY2Fse0Z9L1xcdGV4dHtLZXJ9XFx2YXJwaGkpXisiXSxbMiwyLCJcXHRleHR7aW19XFxtYXRoY2Fse0Z9Il0sWzIsMywiKFxcdGV4dHtpbX1cXG1hdGhjYWx7Rn0pXisiXSxbMiwwLCJcXG1hdGhjYWx7R30vXFx0ZXh0e2ltfVxcbWF0aGNhbHtGfSJdLFszLDEsIihcXG1hdGhjYWx7R30vXFx0ZXh0e2ltfVxcbWF0aGNhbHtGfSleKz1cXHRleHR7Y29rZXJ9XFx2YXJwaGkiXSxbMCwxLCJcXHZhcnBoaSJdLFsyLDAsInUiXSxbMCwzXSxbMyw0XSxbNSwxXSxbMyw1LCJcXHNpbWVxIl0sWzUsNl0sWzQsNiwiXFxzaW1lcSJdLFsxLDcsIlxccGkiXSxbNyw4XSxbNiwxLCJsIiwyLHsiY3VydmUiOjUsInN0eWxlIjp7ImJvZHkiOnsibmFtZSI6ImRhc2hlZCJ9fX1dLFsxLDgsInYiXV0=
\[\begin{tikzcd}
	&& {\mathcal{G}/\text{im}\mathcal{F}} \\
	{\text{Ker}\varphi} & {\mathcal{F}} & {\mathcal{G}} & {(\mathcal{G}/\text{im}\mathcal{F})^+=\text{coker}\varphi} \\
	& {\mathcal{F}/\text{Ker}\varphi} & {\text{im}\mathcal{F}} \\
	& {(\mathcal{F}/\text{Ker}\varphi)^+} & {(\text{im}\mathcal{F})^+}
	\arrow[from=1-3, to=2-4]
	\arrow["u", from=2-1, to=2-2]
	\arrow["\varphi", from=2-2, to=2-3]
	\arrow[from=2-2, to=3-2]
	\arrow["\pi", from=2-3, to=1-3]
	\arrow["v", from=2-3, to=2-4]
	\arrow["\simeq", from=3-2, to=3-3]
	\arrow[from=3-2, to=4-2]
	\arrow[from=3-3, to=2-3]
	\arrow[from=3-3, to=4-3]
	\arrow["\simeq", from=4-2, to=4-3]
	\arrow["l"', curve={height=30pt}, dashed, from=4-3, to=2-3]
\end{tikzcd}\]
By the construction of cokernel, $(\mathcal{F}/\text{Ker}\varphi)^+$ is the cokernel of $u$.  
Since kernel of $v$ contains $\text{im}\mathcal{F}$, by universal property of sheafification, $l_U$ is injective for all $U$ open in $X$ and 
the image of $l$ lie in the kernel of $v$. Now it suffice to show $l:(\text{im}\mathcal{F})^+\rightarrow \text{ker}(v)$ is isomorphism on stalk. 
Notice that morphisms on stalk is clearly injective and for some $[g]\in \text{ker}(v)_x$, where $g\in \mathscr{G}(U)$, since 
$\pi_x([g])=0$(By Proposition\,\ref{proposition:sheafification}), there's $V\subset U$ such that $\pi_V(\left.g\right|_V)=0$. Hence, $\left.g\right|_V\in \text{im}(\mathscr{F})(V)$ which 
implies $l_x$ is surjective. 
Hence, $(\text{im}\mathcal{F})^+$ is the kernel of $v$.
\end{prooff}
\begin{prop}
$\varphi:\mathcal{F}\rightarrow \mathcal{G}$ be a morphism of sheaves, 
then coker $\varphi=0$ if and only if $\varphi_x$ be surjective for all $x\in X$.  
\end{prop}
\begin{prooff}
    coker$=0$ implies $\varphi_x$ is surjective for all $x$: 
    By above diagram, if coker $\varphi=0$, we have $l: (\text{im}\mathcal{F})^+ \rightarrow \mathcal{G}$ is an isomorphism of sheaves. Hence, 
    the map $\text{im}\mathcal{F}\rightarrow \mathcal{G}$ is surjective on stalks. Hence, it suffice to check 
    $\mathcal{F}\rightarrow \mathcal{F}/\text{Ker}\varphi$ is surjective on stalk, which is obvious. 

    $\varphi_x$ is surjective for all $x$implies coker$=0$: it suffice to show $(\text{coker}\varphi)_x$ for all $x\in X$. Since 
    $\varphi_x$ is surjective, $l:(\text{im}\mathcal{F})^+\rightarrow \mathcal{G}$ is an isomorphism of sheaves. Hence, the kernel of $v$ is 
    $\mathcal{G}$. Then $v_x$ is surjective and $=0$ for all $x\in X$. 
\end{prooff}
\begin{prop}
    Let $X$ be a topological space and $i: Z \rightarrow X$ the inclusion of a subspace $Z$. Let $\mathscr{F}$ be a sheaf on $Z$. Show the following properties for the stalks $i_*(\mathscr{F})_x$.
\begin{enu} 
\item For all $x \notin \bar{Z}, i_*(\mathscr{F})_x$ is a singleton (i.e., a set consisting of one element).
\item For all $x \in Z, i_*(\mathscr{F})_x=\mathscr{F}_x$.
\item If $Z=\bbrace{x}$ and $\mathscr{F}$ is a constant sheaf on $Z$ with value $E$, then $i_*(\mathscr{F})$ 
      is called skyscraper sheaf in $x$ with value $E$.
\end{enu}
\label{proposition: stalks of direct image}
\end{prop}
\begin{theo}
    $X$ be a topological space and $Z$ be a closed subset of $X$ with $i:Z\rightarrow X$ be the embedding, 
    $\mathscr{G}$ is a sheaf on $X$ supported on $Z$( That is, $\text{Supp}\mathscr{G}\subset Z$ ), then $i^{-1}\mathscr{G}$
    is a sheaf on $Z$. On the other hand, if $\mathscr{F}$ is a sheaf on $Z$, by Proposition\,\ref{stalks of direct image}, 
    $i_*\mathscr{F}$ is a sheaf supported on $Z$. 
   % https://q.uiver.app/#q=WzAsMixbMCwwLCJcXGxlZnRcXHtcXHRleHR7c2hlYWYgb24gfVggXFx0ZXh0eyBzdXBwb3RlZCBvbiB9WlxccmlnaHRcXH0iXSxbMSwwLCJcXGxlZnRcXHtcXHRleHR7c2hlYWYgb24gfVogXFxyaWdodFxcfSJdLFswLDEsImleey0xfSIsMCx7ImN1cnZlIjotMn1dLFsxLDAsImlfKiIsMCx7ImN1cnZlIjotMn1dXQ==
% https://q.uiver.app/#q=WzAsNixbMCwwLCJcXGxlZnRcXHtcXHRleHR7c2hlYWYgb24gfVggXFx0ZXh0eyBzdXBwb3RlZCBvbiB9WlxccmlnaHRcXH0iXSxbMSwwLCJcXGxlZnRcXHtcXHRleHR7c2hlYWYgb24gfVogXFxyaWdodFxcfSJdLFswLDEsIlxcbWF0aHNjcntGfSJdLFsxLDEsImleey0xfVxcbWF0aHNjcntGfSJdLFsxLDIsIlxcbWF0aHNjcntHfSJdLFswLDIsImlfKlxcbWF0aHNjcntHfSJdLFswLDEsImleey0xfSIsMCx7ImN1cnZlIjotMn1dLFsxLDAsImlfKiIsMCx7ImN1cnZlIjotMn1dLFsyLDNdLFs0LDVdXQ==
\[\begin{tikzcd}
	{\left\{\text{sheaf on }X \text{ suppoted on }Z\right\}} & {\left\{\text{sheaf on }Z \right\}} \\
	{\mathscr{F}} & {i^{-1}\mathscr{F}} \\
	{i_*\mathscr{G}} & {\mathscr{G}}
	\arrow["{i^{-1}}", curve={height=-12pt}, from=1-1, to=1-2]
	\arrow["{i_*}", curve={height=-12pt}, from=1-2, to=1-1]
	\arrow[from=2-1, to=2-2]
	\arrow[from=3-2, to=3-1]
\end{tikzcd}\]
    Moreover, for a sheaf $\mathscr{F}$ supported on $Z$, 
    the identify map $i^{-1}\mathscr{F}\rightarrow i^{-1}\mathscr{F}$ induces $\varphi$ a natural isomorphism of sheaves 
    \begin{equation*}
        \mathscr{F}\rightarrow  i_*i^{-1}\mathscr{F}
    \end{equation*}

    And, for a sheaf $\mathscr{G}$ on $Z$, 
    the identify map $i_*\mathscr{G}\rightarrow i_*\mathscr{G}$ induces $\varphi$ a natural isomorphism of sheaves 
    \begin{equation*}
        i^{-1}i_*\mathscr{G}\rightarrow \mathscr{G}
    \end{equation*}
    \label{proposition: closed subset, sheaf correspondence}
\end{theo}
\begin{prooff}
    Since for all $x\in Z$
    \begin{equation*}
        \varphi_x: \mathscr{F}_x\rightarrow (i_*i^{-1}\mathscr{F})_x\simeq (i^{-1}\mathscr{F})_x \simeq \mathscr{F}_x 
    \end{equation*}
    is an identity map and for all $x\notin Z$, $\mathscr{F}_x=0=(i_*i^{-1}\mathscr{F})_x=0$ by Proposition\,\ref{stalks of direct image}, 
    we have $\mathscr{F}\simeq i_*i^{-1}\mathscr{F}$
\end{prooff}











\section{Ringed Space} 
\begin{defn}
    A ringed space is a pair $\left(X, \mathscr{O}_X\right)$, where $X$ is a topological space and where $\mathscr{O}_X$ is a sheaf of (commutative) rings on $X$.

    If $\left(X, \mathscr{O}_X\right)$ and $\left(Y, \mathscr{O}_Y\right)$ are ringed spaces, we define a morphism of ringed spaces $\left(X, \mathscr{O}_X\right) \rightarrow\left(Y, \mathscr{O}_Y\right)$ as a pair $\left(f, f^b\right)$, where $f: X \rightarrow Y$ is a continuous map and where $f^b: \mathscr{O}_Y \rightarrow f_* \mathscr{O}_X$ is a homomorphism of sheaves of rings on $Y$.

\end{defn}
\begin{defn}
    If $A$ is a local ring, we denote by $\mathfrak{m}_A$ its maximal ideal and by $\kappa(A)=A / \mathfrak{m}_A$ its residue field. A homomorphism of local rings $\varphi: A \rightarrow B$ is called local, if $\varphi\left(\mathfrak{m}_A\right) \subseteq \mathfrak{m}_B$.
    
    A morphism $\left(f, f^b\right): X \rightarrow Y$ of ringed spaces induces morphisms on the stalks as follows. Let $x \in X$. 
    Let $f^{\sharp}: f^{-1} \mathscr{O}_Y \rightarrow \mathscr{O}_X$ be the morphism corresponding to $f^b$ by adjointness. Using the identification $\left(f^{-1} \mathscr{O}_Y\right)_x=\mathscr{O}_{Y, f(x)}$, we get
    $$
    f_x^{\sharp}: \mathscr{O}_{Y, f(x)} \rightarrow \mathscr{O}_{X, x}
    $$
\end{defn}
\begin{defn}
    A locally ringed space is a ringed space $\left(X, \mathscr{O}_X\right)$ such that for all $x \in X$ the stalk $\mathscr{O}_{X, x}$ is a local ring.
    
    A morphism of locally ringed spaces $\left(X, \mathscr{O}_X\right) \rightarrow\left(Y, \mathscr{O}_Y\right)$ is a morphism of ringed spaces $\left(f, f^b\right)$ such that for all $x \in X$ the induced homomorphism on stalks
    $$
    f_x^{\sharp}:\left(f^{-1} \mathscr{O}_Y\right)_x=\mathscr{O}_{Y, f(x)} \rightarrow \mathscr{O}_{X, x}
    $$
    is a local ring homomorphism.
\end{defn}
\begin{defn}
    Let $\left(X, \mathscr{O}_X\right)$ be a locally ringed space and $x \in X$. We call the stalk $\mathscr{O}_{X, x}$ the local ring of $X$ in $x$, denote by $\mathfrak{m}_x$ the maximal ideal of $\mathscr{O}_{X, x}$, and by $\kappa(x)=\mathscr{O}_{X, x} / \mathfrak{m}_x$ the residue field. If $U$ is an open neighborhood of $x$ and $f \in \mathscr{O}_X(U)$, we denote by $f(x) \in \kappa(x)$ the image of $f$ under the canonical homomorphisms $\mathscr{O}_X(U) \rightarrow \mathscr{O}_{X, x} \rightarrow \kappa(x)$.
\end{defn}
% \begin{exam}[prevariety is locally ringed space]
% Let $\left(X, \mathscr{O}_X\right)$ be a prevariety over an algebraically closed field $k$. 
% For $x\in X$, $U$ be an neighborhood of $x$, take $f:U\rightarrow k\in\mathscr{O}_X(U)$.
% By Corollary\,\ref{Corollary:regular function,space with functions}, $f$ is locally invertible at $x$.
% Hence, $\mathscr{O}_{X,x}$ is a local ring with a unique maximal ideal 
% \begin{equation*}
%     m_x=\bbrace{[f]: f(x)=0}
% \end{equation*}
% Then 
% For $U \subseteq X$ now $\mathscr{O}_X(U)$ consists of certain functions $f: U \rightarrow k$. 
% Therefore $\left(X, \mathscr{O}_X\right)$ is a locally ringed space and the surjective homomorphism $\mathscr{O}_{X, x} \rightarrow k,[f] \mapsto f(x)$, induces an isomorphism $\kappa(x) \xrightarrow{\sim} k$ for all $x \in X$
% \end{exam}
\begin{defn}[sheaf of ring on $\spec{A}$]
    For each prime ideal $\mathfrak{p} \subseteq A$, let $A_{\mathfrak{p}}$ 
    be the localization of $A$ at $\mathfrak{p}$. For an open set $U \subseteq \operatorname{Spec} A$, we define $\mathcal{O}(U)$ to be the set of functions 
    $$
    s: U \rightarrow \coprod_{\mathfrak{p} \in U} A_p
    $$
    , such that $s(\mathfrak{p}) \in A_p$ for each $\mathfrak{p}$, and such that $s$ is locally a quotient of elements of $A$ : to be precise, we require that for each $\mathfrak{p} \in U$, there is a neighborhood $V$ of $\mathfrak{p}$, contained in $U$, and elements $a, f \in A$, such that for each 
    $\mathfrak{q} \in V, f \notin \mathfrak{q}$, and $s(\mathfrak{q})=a / f$ in $A_\mathfrak{q}$. 
\end{defn}
\begin{prop}
    Let $A$ be a ring, and $(\operatorname{Spec} A, \mathcal{O})$ its spectrum.
\begin{enu} 
    \item For any $\mathfrak{p} \in \operatorname{Spec} A$, the stalk $\mathcal{O}_{\mathfrak{p}}$ of the sheaf $\mathcal{O}$ is isomorphic to the local ring $A_\mathfrak{p}$.
    \item For any element $f \in A$, the ring $\mathcal{O}(D(f))$ is isomorphic to the localized $\operatorname{ring} A_f$.
    \item In particular, $\Gamma(\operatorname{Spec} A, \mathcal{O}) \cong A$.
\end{enu}
\end{prop}
\begin{prooff}
    (1):First we define a homomorphism from $\mathcal{O}_{\mathfrak{p}}$ to $A_{\mathfrak{p}}$ by sending any local section $s$ in a neighborhood of $\mathfrak{p}$ to its value $s(\mathfrak{p}) \in A_{\mathfrak{p}}$. This gives a well-defined \
    homomorphism $\varphi$ from $\mathcal{O}_\mathfrak{p}$ to $A_\mathfrak{p}$. 
    The map $\varphi$ is surjective, because any element of $A_{\mathfrak{p}}$ can be represented as a quotient $a / f$, with $a, f \in A, f \notin \mathfrak{p}$. Then $D(f)$ will be an open neighborhood of $\mathfrak{p}$, and $a / f$ defines a section of $\mathcal{O}$ over $D(f)$ whose value at $\mathfrak{p}$ is the given element. To show that $\varphi$ is injective, let $U$ be a neighborhood of $\mathfrak{p}$, and let $s, t \in \mathcal{O}(U)$ be elements having the same value $s(\mathfrak{p})=t(\mathfrak{p})$ at $\mathfrak{p}$. By shrinking $U$ if necessary, we may assume that $s=a / f$, and $t=b / g$ on $U$, where $a, b, f, g \in A$, and 
    $f, g \notin \mathfrak{p}$.
    Since $a / f$ and $b / g$ have the same image in $A_\mathfrak{p}$, it follows from the definition of localization that there is an $h \notin \mathfrak{p}$ such that $h(g a-f b)=0$ in $A$. Therefore $a / f=b / g$ in every local ring $A_{\mathfrak{q}}$ such that $f, g, h \notin \mathfrak{q}$. But the set of such $\mathfrak{q}$ is the open set $D(f) \cap$ $D(g) \cap D(h)$, which contains $\mathfrak{p}$. 

    (2): We define a homomorphism $\psi: A_f \rightarrow \mathcal{O}(D(f))$ by sending $a / f^n$ to the section $s \in \mathcal{O}(D(f))$ which assigns to each $\mathfrak{p}$ the image of $a / f^n$ in $A_\mathfrak{p}$.
\end{prooff}
\begin{coro}
    $(\operatorname{Spec} A, \mathcal{O}_{\spec{A}})$ is a locally ringed space.
\end{coro}
\begin{prop}
    $A,B$ are commutative rings, 
    \begin{enu} 
        \item If $\varphi: A \rightarrow B$ is a homomorphism of rings, then $\varphi$ induces a natural morphism of locally ringed spaces
        $$
        \left(f, f^{b}\right):\left(\operatorname{Spec} B, \mathcal{O}_{\text {Spec } B}\right) \rightarrow\left(\operatorname{Spec} A, \mathcal{O}_{\operatorname{Spec} A}\right)
        $$
        where 
        \begin{align*}
            f^b_U:& \mathcal{O}_{\spec{A}}(U)\rightarrow f_*\mathcal{O}_{\spec{A}}(U) \\ 
            &(s:U\rightarrow \coprod_{\mathfrak{p} \in U} A_\mathfrak{p})\mapsto (s\p: f^{-1}(U)\rightarrow U\rightarrow \coprod_{\mathfrak{p} \in U}A_\mathfrak{p}\rightarrow \coprod_{\mathfrak{q} \in f^{-1}(U)} B_\mathfrak{q})
        \end{align*}
        \item If $A$ and $B$ are rings, then any morphism of locally ringed spaces from $\operatorname{Spec} B$ to $\operatorname{Spec} A$ is induced by a homomorphism of rings $\varphi: A \rightarrow B$ as in $(1)$.
    \end{enu}
\end{prop}
\begin{prooff}
    (1): Assume $\mathfrak{p}\in \spec{B}$ and $\varphi^{-1}(\mathfrak{p})=\mathfrak{q}$. Then the ring homomorphism 
    \begin{equation*}
        \varphi_{\mathfrak{p}}: A_\mathfrak{q}\rightarrow B_\mathfrak{p}
    \end{equation*}
    induced by universal property of localization is a local ring homomorphism. 

    (2): Conversely, suppose given a morphism of locally ringed spaces $\left(f, f^{\#}\right)$ from $\operatorname{Spec} B$ to $\operatorname{Spec} A$. 
    Taking global sections, $f^{\#}$ induces a homomorphism of rings $\varphi: \Gamma\left(\operatorname{Spec} A, \mathcal{O}_{\text {Spec } A}\right) \rightarrow \Gamma\left(\operatorname{Spec} B, \mathcal{O}_{\text {Spec } B}\right)$. 
    These rings are $A$ and $B$, respectively, so we have a homomorphism $\varphi: A \rightarrow B$. 
    For any $\mathfrak{p} \in \operatorname{Spec} B$, we have an induced local homomorphism on the stalks(universal property of direct limit), 
    $\mathcal{O}_{\text {Spec } A, f(p)} \rightarrow \mathcal{O}_{\text {Spec } B, p}$ or $A_{f(p)} \rightarrow B_p$, which must be compatible with the map $\varphi$ on global sections. 
    In other words, we have a commutative diagram
    % https://q.uiver.app/#q=WzAsNCxbMCwwLCJBIl0sWzEsMCwiQiJdLFswLDEsIkFfe2YoXFxtYXRoZnJha3twfSl9Il0sWzEsMSwiQl9cXG1hdGhmcmFre3B9Il0sWzAsMSwiXFx2YXJwaGkiXSxbMCwyXSxbMSwzXSxbMiwzLCJmX3tcXG1hdGhmcmFre3B9fV5cXCMiLDJdXQ==
\[\begin{tikzcd}
	A & B \\
	{A_{f(\mathfrak{p})}} & {B_\mathfrak{p}}
	\arrow["\varphi", from=1-1, to=1-2]
	\arrow[from=1-1, to=2-1]
	\arrow[from=1-2, to=2-2]
	\arrow["{f_{\mathfrak{p}}^\#}"', from=2-1, to=2-2]
\end{tikzcd}\]
    Since $f^{\#}$ is a local homomorphism, it follows that $\varphi^{-1}(\mathfrak{p})=f(\mathfrak{p})$, which shows that $f$ coincides with the map $\operatorname{Spec} B \rightarrow \operatorname{Spec} A$ induced by $\varphi$. 

    By universal property of localization, $\varphi_\mathfrak{p}={f_{\mathfrak{p}}^\#}$. Then by Theorem\,\ref{proposition:characterizations of morphism between sheaves}(3), 
    $(f,f^\#)$ is induced by $\varphi$.
\end{prooff}
\begin{coro}
    
\end{coro}
\begin{defn}
    A locally ringed space $\left(X, \mathscr{O}_X\right)$ is called affine scheme, if there exists a ring $A$ such that $\left(X, \mathscr{O}_X\right)$ is isomorphic to $\left(\operatorname{Spec} A, \mathscr{O}_{\mathrm{Spec} A}\right)$.
\end{defn}
\begin{defn}
$A$ scheme is a locally ringed space $\left(X, \mathscr{O}_X\right)$ which admits an open covering $X=\bigcup_{i \in I} U_i$ such that all locally ringed spaces $\left(U_i,\left.\mathscr{O}_X\right|_{U_i}\right)$ are affine schemes. A morphism of schemes is a morphism of locally ringed spaces.
\end{defn}
\begin{defn}[principal oepn subschmems of an affine scheme]
    Let $X=\operatorname{Spec} A$ be an affine scheme. For $f \in A$ let $j: \operatorname{Spec} A_f \rightarrow \operatorname{Spec} A$ be the morphism of affine schemes that corresponds to the canonical homomorphism $A \rightarrow A_f$. Then $j$ induces a homeomorphism of Spec $A_f$ onto $D(f)$. 
    Moreover, for all $x \in D(f), j_x^{\sharp}$ is the canonical isomorphism 
    $A_{\mathfrak{p}_x} \xrightarrow{\sim}\left(A_f\right)_{\mathfrak{p}_x}$ by Algebra Theorem\,\ref{theorem: localization twice}. 
    Hence we see that $\left(j, j^{\sharp}\right)$ induces an isomorphism 
    of the affine scheme $\operatorname{Spec} A_f$ with the locally ringed space $\left(D(f), \mathscr{O}_{X \mid D(f)}\right)$.
    
\end{defn}
\begin{defn}[closed subschmems of affine schemes]
    Let $X=\operatorname{Spec} A$ be an affine scheme. For an ideal $\mathfrak{a}$ of $A$ let $i: \operatorname{Spec} A / \mathfrak{a} \rightarrow \operatorname{Spec} A$ be the morphism of affine schemes that corresponds to the canonical homomorphism $A \rightarrow A / \mathfrak{a}$. 
    Then $i$ induces a homeomorphism of $\operatorname{Spec} A / \mathfrak{a}$ onto the closed subset $V(\mathfrak{a})$ of $\operatorname{Spec} A$. Moreover, for all $x \in V(\mathfrak{a})$ the morphism $i_x^b$ is the canonical surjective homomorphism 
    $A_{\mathfrak{p}_x} \rightarrow(A / \mathfrak{a})_{\overline{\mathfrak{p}_x}}$ where $\overline{\mathfrak{p}}_x$ is the image of $\mathfrak{p}_x$ in $A / \mathfrak{a}$.
\end{defn}
\section{Basic Propositions}
\begin{defn}
    Let $S$ be a fixed scheme. The category (Sch/S) of schemes over $S$ (or of $S$-schemes) is the category whose objects are the morphisms $X \rightarrow S$ of schemes, and whose morphisms $\operatorname{Hom}(X \rightarrow S, Y \rightarrow S)$ are the morphisms $X \rightarrow Y$ of schemes with the property that
% https://q.uiver.app/#q=WzAsMyxbMCwwLCJYIl0sWzIsMCwiWSJdLFsxLDEsIlMiXSxbMCwxXSxbMCwyXSxbMSwyXV0=
\[\begin{tikzcd}
	X && Y \\
	& S
	\arrow[from=1-1, to=1-3]
	\arrow[from=1-1, to=2-2]
	\arrow[from=1-3, to=2-2]
\end{tikzcd}\]
commutes.
\end{defn}
\begin{prop}[open subscheme]
    (1) Let $X$ be a scheme, and $U \subseteq X$ an open subset. Then the locally ringed space $\left(U, \mathscr{O}_{X \mid U}\right)$ is a scheme. We call $U$ an open subscheme of $X$. If $U$ is an affine scheme, then $U$ is called an affine open subscheme.
    
    (2) Let $X$ be a scheme. The affine open subschemes are a basis of the topology.

    (3) There's a canonical morphism between scheme $(\left.U,\mathscr{O}_X\right|_U)$ and 
    $(X,\mathscr{O}_X)$.

    (4): $(f,f^b): (X,\mathscr{O}_X)\rightarrow (Y,\mathscr{O}_Y)$ is a morphism of scheme and $f(X)\subset U$ for some open subset of $Y$, then there's a natrual 
    morphism $(X,\mathscr{O}_X)\rightarrow (U,\left. \mathscr{O}_Y\right|_U)$ making the following diagram ccommute 
    % https://q.uiver.app/#q=WzAsMyxbMCwwLCIoWCxcXG1hdGhzY3J7T31fWCkiXSxbMSwwLCIoWSxcXG1hdGhzY3J7T31fWSkiXSxbMSwxLCIoVSxcXGxlZnQuIFxcbWF0aHNjcntPfV9ZXFxyaWdodHxfVSkiXSxbMCwxXSxbMiwxXSxbMCwyXV0=
\[\begin{tikzcd}
	{(X,\mathscr{O}_X)} & {(Y,\mathscr{O}_Y)} \\
	& {(U,\left. \mathscr{O}_Y\right|_U)}
	\arrow[from=1-1, to=1-2]
	\arrow[from=1-1, to=2-2]
	\arrow[from=2-2, to=1-2]
\end{tikzcd}\]
\end{prop}
\begin{prooff}
    (3):  For all the $V$ open in $X$, the restriction maps 
    \begin{equation*}
        \Gamma(V,\mathscr{O}_X)\rightarrow  \Gamma(V\cap U,\left.\mathscr{O}_X\right|_U)
    \end{equation*}
    induce
    a morphism $j^{b}: \mathscr{O}_X \rightarrow j_*(\left.\mathscr{O}_X\right|_U)$ of sheaves.
    
    Hence, there's a canonical morphism $(U,\left.\mathscr{O}_X\right|_U)\rightarrow (X,\mathscr{O}_X)$ of scheme.

    

\end{prooff}
\begin{lem}[Nike's Trick]
    Let $X$ be a scheme, and let $U, V$ be affine open subschemes of $X$. Then there exists for all $x \in U \cap V$ an open subscheme $W \subseteq U \cap V$ with $W \ni x$ such that $W$ is principal open in $U$ as well as in $V$.
\end{lem}
\begin{prooff}
    We may assume $x\in V\subset U $ and $U,V$ are all open affine, hence 
    \begin{equation*}
        (j,j^b): (V,\left.\mathscr{O}_X\right|_{V})\rightarrow (U,\left.\mathscr{O}_X\right|_{U})
    \end{equation*}
    is a morphism of scheme.
% https://q.uiver.app/#q=WzAsNixbMCwwLCIoVixcXGxlZnQuXFxtYXRoc2Nye099X1hcXHJpZ2h0fF97Vn0pIl0sWzAsMSwiKFUsXFxsZWZ0LlxcbWF0aHNjcntPfV9YXFxyaWdodHxfe1V9KSJdLFsxLDAsIlxcbGVmdChcXG9wZXJhdG9ybmFtZXtTcGVjfSBBLCBcXG1hdGhzY3J7T31fe1xcbWF0aHJte1NwZWN9IEF9XFxyaWdodCkiXSxbMSwxLCJcXGxlZnQoXFxvcGVyYXRvcm5hbWV7U3BlY30gQiwgXFxtYXRoc2Nye099X3tcXG1hdGhybXtTcGVjfSBCfVxccmlnaHQpIl0sWzIsMSwiQSJdLFsyLDAsIkIiXSxbMCwxLCJqIl0sWzAsMiwiXFxzaW1lcSJdLFsxLDMsIlxcc2ltZXEiXSxbMiwzLCJcXHZhcnBoaSJdLFs0LDUsIlxccGhpIl1d
\[\begin{tikzcd}
	{(V,\left.\mathscr{O}_X\right|_{V})} & {\left(\operatorname{Spec} A, \mathscr{O}_{\mathrm{Spec} A}\right)} & B \\
	{(U,\left.\mathscr{O}_X\right|_{U})} & {\left(\operatorname{Spec} B, \mathscr{O}_{\mathrm{Spec} B}\right)} & A
	\arrow["\simeq", from=1-1, to=1-2]
	\arrow["j", from=1-1, to=2-1]
	\arrow["\varphi", from=1-2, to=2-2]
	\arrow["\simeq", from=2-1, to=2-2]
	\arrow["\phi", from=2-3, to=1-3]
\end{tikzcd}\]
Take $f\in B$ such that the principal open subset $D(f)$ satisfies 
$x\in D(f)\subset V\subset U$, then 
\begin{equation*}
    D(f)=j^{-1}(D(f))=\varphi^{-1}(D(f))=D(\phi(f))
\end{equation*}
\end{prooff}
\begin{lem}[Gluing of morphisms]
   Let $X, Y$ be schemes.
   If $X=\bigcup_i U_i$ is an open covering, 
   then a family of morphisms $\varphi_i : (U_i,\left.\mathscr{O}_X\right|_{U_i}) \rightarrow (Y,\mathscr{O}_Y)$ 
   glues to a morphism $(f,f^b): (X,\mathscr{O}_X) \rightarrow (Y,\mathscr{O}_Y) $ if and only if 
   the morphisms coincide on intersections $U_i \cap U_j$, and the resulting morphism $X \rightarrow Y$ is uniquely determined.
   \label{lemma: gluing of morphisms}
\end{lem}
\begin{prooff}
    Firstly, define 
    \begin{equation*}
        f:X\rightarrow Y, x\mapsto \varphi_i(x) \text{ if } x\in U_i
    \end{equation*}
    For some $V$ open in $Y$, we can obtain $\varphi_V$ by the following diagram: 
% https://q.uiver.app/#q=WzAsNCxbMCwwLCJcXG1hdGhzY3J7T31fWShWKSJdLFswLDEsIlxcbWF0aHNjcntPfV9ZKFYpIl0sWzEsMSwiKFxcdmFycGhpX2kpXypcXGxlZnQuXFxtYXRoc2Nye099X3tYfVxccmlnaHR8X3tVX2l9KFYpPVxcbWF0aHNjcntPfV9YKFVfaVxcY2FwIGZeey0xfShWKSkiXSxbMSwwLCJmXypcXG1hdGhzY3J7T31fWChWKT1cXG1hdGhzY3J7T31fWChmXnstMX0oVikpIl0sWzEsMiwiKFxcdmFycGhpX2kpX1YiXSxbMiwzLCJcXHRleHR7Z2x1ZX0iXSxbMCwxLCJcXHRleHR7aWR9Il0sWzAsMywiXFx2YXJwaGlfViIsMCx7InN0eWxlIjp7ImJvZHkiOnsibmFtZSI6ImRhc2hlZCJ9fX1dXQ==
\[\begin{tikzcd}
	{\mathscr{O}_Y(V)} & {f_*\mathscr{O}_X(V)=\mathscr{O}_X(f^{-1}(V))} \\
	{\mathscr{O}_Y(V)} & {(\varphi_i)_*\left.\mathscr{O}_{X}\right|_{U_i}(V)=\mathscr{O}_X(U_i\cap f^{-1}(V))}
	\arrow["{\varphi_V}", dashed, from=1-1, to=1-2]
	\arrow["{\text{id}}", from=1-1, to=2-1]
	\arrow["{(\varphi_i)_V}", from=2-1, to=2-2]
	\arrow["{\text{glue}}", from=2-2, to=1-2]
\end{tikzcd}\]



\end{prooff}
\begin{exam}[zero section]
    Consider $\bb{A}_R^{n+1}=\spec{R[T_0,\dots,T_n]}$, define 
    $$
    \bb{A}_R^{n+1}-\bbrace{0}=\bigcup_{i=0}^n D(T_i)
    $$
    be an open subscheme of $\bb{A}_R^{n+1}$. Since there's natural morphism $p_i$ given by     
    $$
    p_i: D\left(T_i\right)=\operatorname{Spec} R\left[T_0, \ldots, T_n, T_i^{-1}\right] \rightarrow D_{+}\left(X_i\right)=\operatorname{Spec} R\left[\frac{X_0}{X_i}, \ldots, \frac{\widehat{X_i}}{X_i}, \ldots, \frac{X_n}{X_i}\right]
    $$, 
    by gluing of morphisms of scheme, there's a natural morphism 
    \begin{equation*}
       p: \bb{A}_R^{n+1}-\bbrace{0}\rightarrow \bb{P}_R^n 
    \end{equation*}
\end{exam}
\begin{exam}
    Consider $X=\spec{\bb{R}[x,y]}-\bbrace{0}$ and $p:X\rightarrow \bb{P}_\bb{R}^1$. For $(\alpha,\beta)\neq (0,0)$, 
    \begin{equation*}
        p((x-\alpha,y-\beta))=(\alpha y-\beta x)
    \end{equation*}
\end{exam}
\begin{exam}
    Let $A$ be an $R$-algebra, 
    let $f: \operatorname{Spec} A \rightarrow \mathbb{A}_R^n$ be an $R$-morphism, and denote the corresponding $R$-algebra homomorphism by $\varphi: R\left[T_1, \ldots, T_n\right] \rightarrow A$. Set $a_i=\varphi\left(T_i\right) \in A$. 
    Then $f$ factors through $ \bb{A}_R^{n}-\bbrace{0}$
    if and only if for all $\mathfrak{p}\in \spec{A}$, 
    $$
        f(\mathfrak{p})\in \bigcup_{i=0}^n D(T_i)
    $$
    Equivalently, there's no such prime ideal $\mathfrak{p} \subset A$ such that 
    $\varphi^{-1}(\mathfrak{p})\supset (T_1,\dots,T_n)$. Since 
    $\varphi^{-1}(\mathfrak{p})\supset (T_1,\dots,T_n)$ if and only 
    if  $\mathfrak{p}\supset (\varphi(T_1),\dots,\varphi(T_n))$, we have 
    \begin{equation*}
        \text{Hom}_{R}(\spec{A},\bb{A}_R^{n}-\bbrace{0})=\bbrace{\varphi\in \text{Hom}_{(R-\mathrm{Alg})}(R[x_1,\dots,x_n], A): (\varphi(x_1),\dots,\varphi(x_n))=(1)}
    \end{equation*}
    
\end{exam}
\begin{exam}[$\bb{G}_m$]
    Set $X=\operatorname{Spec} R\left[U, U^{-1}\right]=R[U,T]/(UT-1)$. 
    Then we obtain for every $R$-scheme $T$
    $$
    \text{Hom}_R(T,X)=\operatorname{Hom}_{(R-\mathrm{Alg})}\left(R\left[U, U^{-1}\right], \Gamma\left(T, \mathscr{O}_T\right)\right)=\Gamma\left(T, \mathscr{O}_T\right)^{\times} .
    $$    
\end{exam}





\begin{prop}
    Let $\left(X, \mathscr{O}_X\right)$ be a scheme, $Y=\operatorname{Spec} A$ an affine scheme.
    Then the natural map
    $$
    \operatorname{Hom}(X, Y) \longrightarrow \operatorname{Hom}\left(A, \Gamma\left(X, \mathscr{O}_X\right)\right), \quad\left(f, f^b\right) \mapsto f_Y^b,
    $$
    is a functorial bijection. Here the set on the left side denotes the set of morphisms $X \rightarrow Y$ of scheme, and the set on the right side denotes the set of ring homomorphisms $A \rightarrow \Gamma\left(X, \mathscr{O}_X\right)$.
\end{prop}
\begin{prooff}
% https://q.uiver.app/#q=WzAsNCxbMCwwLCJcXHRleHR7SG9tfShYLFkpIl0sWzEsMCwiXFx0ZXh0e0hvbX0oQSxcXEdhbW1hKFgsXFxtYXRoc2Nye099X1gpKSJdLFsxLDEsIlxcdGV4dHtIb219KEEsXFxHYW1tYShVX2ksXFxtYXRoc2Nye099X1gpKSJdLFswLDEsIlxcdGV4dHtIb219KFVfaSxZKSJdLFswLDEsIiIsMCx7InN0eWxlIjp7ImJvZHkiOnsibmFtZSI6ImRhc2hlZCJ9fX1dLFsxLDIsIlxcdGV4dHtyZXN9IiwwLHsiY3VydmUiOi0xfV0sWzIsMSwiXFx0ZXh0e2dsdWV9IiwwLHsiY3VydmUiOi0xfV0sWzMsMCwiXFx0ZXh0e2dsdWV9IiwwLHsiY3VydmUiOi0xfV0sWzAsMywiXFx0ZXh0e3Jlc30iLDAseyJjdXJ2ZSI6LTF9XSxbMywyLCJcXHNpbWVxIl1d
\[\begin{tikzcd}
	{\text{Hom}(X,Y)} & {\text{Hom}(A,\Gamma(X,\mathscr{O}_X))} \\
	{\text{Hom}(U_i,Y)} & {\text{Hom}(A,\Gamma(U_i,\mathscr{O}_X))}
	\arrow[dashed, from=1-1, to=1-2]
	\arrow["{\text{res}}", curve={height=-6pt}, from=1-1, to=2-1]
	\arrow["{\text{res}}", curve={height=-6pt}, from=1-2, to=2-2]
	\arrow["{\text{glue}}", curve={height=-6pt}, from=2-1, to=1-1]
	\arrow["\simeq", from=2-1, to=2-2]
	\arrow["{\text{glue}}", curve={height=-6pt}, from=2-2, to=1-2]
\end{tikzcd}\]
Injective: For $f:X\rightarrow Y$, define $f_i:U_i\rightarrow X\rightarrow Y$ a morphism of scheme. It's easy to check the follow diagram commutes 
% https://q.uiver.app/#q=WzAsMyxbMCwwLCJBIl0sWzEsMCwiXFxHYW1tYShYLFxcbWF0aHNjcntPfV9YKSJdLFsxLDEsIlxcR2FtbWEoVV9pLFxcbWF0aHNjcntPfV9YKSJdLFswLDEsImZeYl9ZIl0sWzEsMiwial5iX1giXSxbMCwyLCIoZl9pKV5iX1kiLDJdXQ==
\[\begin{tikzcd}
	A & {\Gamma(X,\mathscr{O}_X)} \\
	& {\Gamma(U_i,\mathscr{O}_X)}
	\arrow["{f^b_Y}", from=1-1, to=1-2]
	\arrow["{(f_i)^b_Y}"', from=1-1, to=2-2]
	\arrow["{j^b_X}", from=1-2, to=2-2]
\end{tikzcd}\]
Hence, $(f,f^b)=(g,g^b)$ iff $(f_i,f_i^b)=(g_i,g_i^b)$ iff $(f_i)^b_Y=(g_i)^b_Y$ iff $f^b_Y=g^b_Y$

Surjective: 
% https://q.uiver.app/#q=WzAsOCxbMCwwLCJVX2kiXSxbMSwwLCJZIl0sWzEsMSwiVV9qIl0sWzAsMSwiViJdLFsyLDAsIlxcR2FtbWEoVV9pLFxcbWF0aHNjcntPfV9YKSJdLFsyLDEsIlxcR2FtbWEoVixcXG1hdGhzY3J7T31fWCkiXSxbMywwLCJBIl0sWzMsMSwiXFxHYW1tYShVX2osXFxtYXRoc2Nye099X1gpIl0sWzAsMSwiZl9pIl0sWzIsMSwiZl9qIiwyXSxbMywyLCJsX2oiLDJdLFszLDAsImxfaSJdLFs2LDQsIlxcdGlsZGV7Zl9pfSJdLFs2LDcsIlxcdGlsZGV7Zl9qfSIsMl0sWzcsNV0sWzQsNV1d
\[\begin{tikzcd}
	{U_i} & Y & {\Gamma(U_i,\mathscr{O}_X)} & A \\
	V & {U_j} & {\Gamma(V,\mathscr{O}_X)} & {\Gamma(U_j,\mathscr{O}_X)}
	\arrow["{f_i}", from=1-1, to=1-2]
	\arrow[from=1-3, to=2-3]
	\arrow["{\tilde{f_i}}", from=1-4, to=1-3]
	\arrow["{\tilde{f_j}}"', from=1-4, to=2-4]
	\arrow["{l_i}", from=2-1, to=1-1]
	\arrow["{l_j}"', from=2-1, to=2-2]
	\arrow["{f_j}"', from=2-2, to=1-2]
	\arrow[from=2-4, to=2-3]
\end{tikzcd}\]
Take $\tilde{f}\in \text{Hom}(A,\Gamma(X,\mathscr{O}_X))$, and define $\tilde{f_i}: \tilde{f}\circ \text{Res}^{X}_{U_i}$ and $f_i$ be the corresponding morphisms with respect to 
category equivalence(commutative rings and affine schemes).
Consider above diagram, $V$ is an affine open subset of $U_i\cap U_j$. Since opposite category of commutative rings is 
equivalent to category of affine scheme, the fact that the right diagram commutes implies the left diagram commute.  

\end{prooff}
\begin{prop}
    Let $\left(X, \mathscr{O}_X\right)$ be a $k$-scheme, $A$ be a $k$-algebra and $Y=\operatorname{Spec} A$ an affine scheme over $k$.
    Then the natural map
    $$
    \operatorname{Hom}_{\spec{k}}(X, Y) \longrightarrow \operatorname{Hom}_k \left(A, \Gamma\left(X, \mathscr{O}_X\right)\right), \quad\left(f, f^b\right) \mapsto f_Y^b,
    $$
    is a functorial bijection. Here the set on the left side denotes the set of morphisms $X \rightarrow Y$ of $k$-scheme, and the set on the right side denotes the set of $k$-algebra homomorphisms $A \rightarrow \Gamma\left(X, \mathscr{O}_X\right)$.
\end{prop}
\begin{prop}
    Let $X$ be a scheme. Let $x \in X$, and let $U \subseteq X$ be an affine open neighborhood of $x$, say $U=\operatorname{Spec} A$. Denote by $\mathfrak{p} \subset A$ the prime ideal of $A$ corresponding to $x$. Then $\mathscr{O}_{X, x}=\mathscr{O}_{U, x}=A_{\mathfrak{p}}$, and the natural homomorphism $A \rightarrow A_{\mathfrak{p}}$ gives us a morphism
    $$
    j_x: \operatorname{Spec} \mathscr{O}_{X, x}=\operatorname{Spec} A_{\mathfrak{p}} \rightarrow \operatorname{Spec} A=U \subseteq X
    $$
    of schemes. This morphism is independent of the choice of $U$.
\end{prop}
\begin{prooff}
    Assume $V$ is an open affine subset of $U$ with $x\in V$, $V=\spec{B}$ and $x=\mathfrak{q}$. 
    Then, it suffices to show $j_x$ induced by $V$ and 
    $j_x$ induced by $U$ identifies. 
    Consider the following commutative diagram 
    % https://q.uiver.app/#q=WzAsOSxbNCwwLCJYIl0sWzMsMSwiVSJdLFszLDIsIlYiXSxbMiwxLCJcXHRleHR7U3BlY31BIl0sWzIsMiwiXFx0ZXh0e1NwZWN9QiJdLFsxLDEsIlxcdGV4dHtTcGVjfUFfe1xcbWF0aGZyYWt7cH19Il0sWzEsMiwiXFx0ZXh0e1NwZWN9Ql97XFxtYXRoZnJha3txfX0iXSxbMCwxLCJcXHRleHR7U3BlY31cXG1hdGhjYWx7T31fe1gseH0iXSxbMCwyLCJcXHRleHR7U3BlY31cXG1hdGhjYWx7T31fe1gseH0iXSxbMSwwXSxbMiwwXSxbMiwxXSxbMywxLCJcXHNpbWVxIl0sWzQsMiwiXFxzaW1lcSJdLFs0LDMsIlxcdmFycGhpIiwyXSxbNSwzXSxbNiw0XSxbNiw1XSxbOCw3LCJcXHRleHR7aWR9IiwyXSxbNyw1LCJcXHNpbWVxIl0sWzgsNiwiXFxzaW1lcSJdXQ==
\[\begin{tikzcd}
	&&&& X \\
	{\text{Spec}\mathcal{O}_{X,x}} & {\text{Spec}A_{\mathfrak{p}}} & {\text{Spec}A} & U \\
	{\text{Spec}\mathcal{O}_{X,x}} & {\text{Spec}B_{\mathfrak{q}}} & {\text{Spec}B} & V
	\arrow["\simeq", from=2-1, to=2-2]
	\arrow[from=2-2, to=2-3]
	\arrow["\simeq", from=2-3, to=2-4]
	\arrow[from=2-4, to=1-5]
	\arrow["{\text{id}}"', from=3-1, to=2-1]
	\arrow["\simeq", from=3-1, to=3-2]
	\arrow[from=3-2, to=2-2]
	\arrow[from=3-2, to=3-3]
	\arrow["\varphi"', from=3-3, to=2-3]
	\arrow["\simeq", from=3-3, to=3-4]
	\arrow[from=3-4, to=1-5]
	\arrow[from=3-4, to=2-4]
\end{tikzcd}\]
where the morphism $\spec{B_\mathfrak{q}}\rightarrow \spec{A_\mathfrak{p}}$ is induced both by 
universal propoty of localization and the morphism of sheaves $\mathscr{O}_{\spec{A}}\rightarrow \varphi_*\mathscr{O}_{\spec{B}}$.
\end{prooff}
\begin{prop}
    The image of the canonical map $j_x:\text{Spec}\mathcal{O}_{X,x}\rightarrow X$ is 
    $$
    Z=\bbrace{y\in X: x\in \overline{\bbrace{y}}}=\bigcap_{x\in W, W \text{ open in }X} W
    $$
\end{prop}
\begin{prooff}
    Trivial.
\end{prooff}



\begin{prop}
    Let $\kappa(x)=\mathscr{O}_{X, x} / \mathfrak{m}_x$ be the residue class field of $x$ in $X$. We obtain a morphism of schemes
    $$
    i_x: \operatorname{Spec} \kappa(x) \longrightarrow \operatorname{Spec} \mathscr{O}_{X, x} \longrightarrow X
    $$
    called canonical. The image point of the unique point in $\operatorname{Spec} \kappa(x)$ is $x$.
    Notice that the map $\mathscr{O}_{X,x}\rightarrow \kappa(x)$  induced by considering the stalk of $i_x$ is exactly the projective map.



    Now let $K$ be any field, let $f: \operatorname{Spec} K \rightarrow X$ be a morphism, and let $x \in X$ be the image point of the unique point $p$ of Spec $K$. Since $f$ is a morphism of locally ringed spaces, $f$ induces a local homomorphism $\mathscr{O}_{X, x} \rightarrow K=\mathscr{O}_{\text {Spec } K, p}$, and hence a homomorphism $\iota: \kappa(x) \rightarrow K$ between the residue class fields. 
    
    Then, the morphism $f$ factors as $f=i_x \circ(\operatorname{Spec} \iota): \operatorname{Spec} K \rightarrow \operatorname{Spec} \kappa(x) \rightarrow X$ since 
    we have a commutative diagram in stalks of those sheaves: 
    % https://q.uiver.app/#q=WzAsMyxbMSwwLCJcXG1hdGhjYWx7T31fe1gseH0iXSxbMCwxLCIgXFxrYXBwYSh4KSJdLFswLDAsIksiXSxbMCwxXSxbMSwyXSxbMCwyXV0=
\[\begin{tikzcd}
	K & {\mathscr{O}_{X,x}} \\
	{ \kappa(x)}
	\arrow[from=1-2, to=1-1]
	\arrow[from=1-2, to=2-1]
	\arrow[from=2-1, to=1-1]
\end{tikzcd}\]
    
    The above construction gives rise to a bijection 
    $$
    \operatorname{Hom}(\operatorname{Spec} K, X) \longrightarrow\{(x, \iota) ; x \in X, \iota: \kappa(x) \rightarrow K\}
    $$
    This is because, 
    we can map an element $(x, \iota: \kappa(x) \rightarrow K)$ of the right hand side to the morphism
    $$
    \operatorname{Spec} K \xrightarrow{\operatorname{Spec} \iota} \operatorname{Spec} \kappa(x) \xrightarrow{i_x} X,
    $$
    and these two maps are inverse to each other.
\end{prop}
\begin{prop}
    Assume $(X,\mathscr{O}_X)\rightarrow \spec{k}$ be a $k$-scheme, then this map induces a local ring homomorphism
    \begin{equation*}
        k\rightarrow \mathscr{O}_{X,x}
    \end{equation*}
    which induecs a field extension 
    \begin{equation*}
        k\rightarrow \kappa(x) 
    \end{equation*} 
    Hence there's natural $k$-scheme structure on $\spec{\kappa(x)}$. Moreover, above natural morphism $i_x$ becomes a $k$-scheme morphism: 
% https://q.uiver.app/#q=WzAsNCxbMCwxLCJcXHRleHR7U3BlY31cXGthcHBhKHgpIl0sWzEsMSwiXFx0ZXh0e1NwZWN9ayJdLFswLDAsIihYLFxcbWF0aHNjcntPfV9YKSJdLFswLDIsIlxcdGV4dHtTcGVjfUsiXSxbMCwxXSxbMCwyLCJpX3giLDJdLFsyLDFdLFszLDFdLFszLDBdXQ==
\[\begin{tikzcd}
	{(X,\mathscr{O}_X)} \\
	{\text{Spec}\kappa(x)} & {\text{Spec}k} \\
	{\text{Spec}K}
	\arrow[from=1-1, to=2-2]
	\arrow["{i_x}"', from=2-1, to=1-1]
	\arrow[from=2-1, to=2-2]
	\arrow[from=3-1, to=2-1]
	\arrow[from=3-1, to=2-2]
\end{tikzcd}\]
Hence, if $k\rightarrow K$ be a field extension, there's a bijection
$$
\operatorname{Hom}_k(\operatorname{Spec} K, X) \longrightarrow\{(x, \iota) : x \in X, \iota: \kappa(x) \rightarrow K \quad k\text{-algebra homomorphism}\}
$$
And for an arbitrary $k$-scheme, define $X(K)=\operatorname{Hom}_k(\operatorname{Spec} K, X)$ to be its $K$-points.
\label{proposition: K points of k-scheme}
\end{prop}



\begin{defn}[Structure sheaf on $\text{Proj} S$]
    Let $S$ be a graded ring, 
    we will define a sheaf of rings $\mathscr{O}$ on $\operatorname{Proj} S$. 
    For each $\mathfrak{p} \in \operatorname{Proj} S$, we consider the ring $S_{(\mathfrak{p})}$ of elements of degree zero in the localized ring $T^{-1} S$, 
    where $T$ is the multiplicative system consisting of all homogeneous elements of $S$ which are not in $\mathfrak{p}$. For any open subset $U \subseteq \operatorname{Proj} S$, 
    we define $\mathscr{O}(U)$ to be the set of functions $s: U \rightarrow \coprod S_{(\mathfrak{p})}$ such that for each $\mathfrak{p} \in U, s(\mathfrak{p}) \in S_{(\mathfrak{p})}$, and such that $s$ is locally a quotient of elements of $S$ : for each $\mathfrak{p} \in U$, there exists a neighborhood $V$ of $\mathfrak{p}$ in $U$,
     and homogeneous elements $a, f$ in $S$, of the same degree, such that for all $\mathfrak{q} \in V, f \notin \mathfrak{q}$, and $s(\mathfrak{q})=a / f$ in $S_{(\mathfrak{q})}$. Now it is clear that $\mathscr{O}$ is a 
     presheaf of rings, with the natural restrictions, and it is also clear from the local nature of the definition that $\mathscr{O}$ is a sheaf.
\end{defn}
\begin{prop}
    Let $S$ be a graded ring.
\begin{enu} 
    \item For any $\mathfrak{p} \in \operatorname{Proj} S$, the stalk $\mathscr{O}_{\mathfrak{p}}$ is isomorphic to the local ring $S_{(\mathfrak{p})}$.
    \item For any homogeneous element $f \in S_{+}$, let $D_{+}(f)=\{\mathfrak{p} \in \operatorname{Proj} S \mid f \notin \mathfrak{p}\}$.
    Then $D_{+}(f)$ is open in Proj S. Furthermore, these open sets cover Proj $S$, and for each such open set, we have an isomorphism of locally ringed spaces
    $$
    \left(D_{+}(f),\left.\mathscr{O}\right|_{D_{+}(f)}\right) \cong \operatorname{Spec} S_{(f)}
    $$
    where 
    $$  
     S_{(f)}=\bbrace{a/f^n\in S_f: a \text{ homogeneous and }\deg(a)=n\deg(f),n\ge 0}
    $$

    In particular, the global section of $S$ is $S_0$.
\end{enu}
\end{prop}
\begin{prooff}
    (1): $S_{(\mathfrak{p})}$ is a local ring: %By Algebra Proposition\,\ref{proposition:correspondence spec of localization}, 
    % prime ideals of  $S_{(\mathfrak{p})}$ correspond to prime ideals of $S$ which do not meet with 
    % the homogenous elements 
    % in $S-\mathfrak{p}$. Since $\mathfrak{p}$ itself is a homogenous ideal, it correspondes to the unique maximal dieal in $S_{(\mathfrak{p})}$. 
    The unique maximal ideal of $S_{(\mathfrak{p})}$ is of the form 
    \begin{equation*}
        \bbrace{a/f: a\in\mathfrak{p}, f\notin \mathfrak{p}, \deg a=\deg f}
    \end{equation*}

    (2): Define 
    \begin{equation*}
        \varphi: D_+(f)\rightarrow \spec{S_{(f)}}, \mathfrak{a}\mapsto \mathfrak{a}S_f\cap S_{(f)}
    \end{equation*}
    
    $\varphi$ is injective: If $\mathfrak{a}S_f\cap S_{(f)}=\mathfrak{b}S_f\cap S_{(f)}$, 
    for some homogeneous element $s\in \mathfrak{a}$, there's $b\in \mathfrak{b}$ such that 
    \begin{equation*}
        \frac{s^n}{f^m}=\frac{b}{f^t}
    \end{equation*}
    for some integer $n,m,t$. Hence, $s^n\in \mathfrak{b}$ which implies $s\in \mathfrak{b}$. 

    $\varphi$ is surjective: $P$ be a prime ideal of $S_{(f)}$, define 
    \begin{equation*}
        \mathfrak{p}=\bbrace{s\in S: s/f^n\in P\text{ for some }n\ge 0} 
    \end{equation*}
    Then $\varphi(\mathfrak{p})=P$. 

    Isomorphism on stalk: For $\mathfrak{p}\in D_+(f)$, there's a natural ring homomorphism 
    \begin{equation*}
        S_{(f)}\rightarrow S_{(\mathfrak{p})}, a/f^n\mapsto a/f^n
    \end{equation*}
    and by universal property of localization, it induces a ring homomorphism
    \begin{equation*}
        \varphi_{\mathfrak{p}}: (S_{(f)})_{\varphi(\mathfrak{p})}\rightarrow S_{(\mathfrak{p})}
    \end{equation*}
    Acturally, $ \varphi_{\mathfrak{p}}$ is an isomorphism: injective is easy to check, and for some $a/g\in S_{\mathfrak{p}}$, notice that 
    \begin{equation*}
        \frac{a}{g}=\frac{ag^{\deg f -1 }}{f^{\deg g}}\frac{f^{\deg g}}{g^{\deg f}}
    \end{equation*}
    Hence, $ \varphi_{\mathfrak{p}}$ is surjective.

    Isomorphism $\varphi_{\mathfrak{p}}$ induces a isomorphism of sheaves 
    \begin{equation*}
        \varphi^b: \mathscr{O}_{\text{Spec}S_{(f)}} \simeq \varphi_*(\left.\mathscr{O}_{\text{Proj}S}\right|_{D^+(f)})
    \end{equation*}
\end{prooff}
\begin{prop}[morphisms between projective spectrum]
    Let $S$ be a graded ring. 
\begin{enu} 
    \item Let $\varphi: S \rightarrow T$ be a graded homomorphism of graded rings (preserving degrees). 
    Let $U=\left\{\mathfrak{p} \in \operatorname{Proj} T \mid \mathfrak{p}  \nsupseteq \varphi\left(S_{+}\right)\right\}$. Show that $U$ is an open subset of Proj $T$, and show that $\varphi$ determines a natural morphism 
    $f: U \rightarrow \operatorname{Proj} S$.
    \item The morphism $f$ can be an isomorphism even when $\varphi$ is not. For example, suppose that $\varphi_d: S_d \rightarrow T_d$ is an isomorphism for all $d \geqslant d_0$, where $d_0$ is an integer. Then show that $U=\operatorname{Proj} T$ and the morphism $f: \operatorname{Proj} T \rightarrow \operatorname{Proj} S$ is an isomorphism.
    \item Let $\varphi: S \rightarrow T$ be a surjective homomorphism of graded rings, preserving degrees. Then, the open set $U$ of is equal to Proj $T$, and the morphism $f$ :Proj $T \rightarrow$ Proj $S$ is a closed immersion.
    \item If $I \subseteq S$ is a homogeneous ideal, take $T=S / I$ and let $Y$ 
    be the closed subscheme of $X=\operatorname{Proj} S$ defined by 
    the closed immersion Proj $S / I \rightarrow X$. Show that different homogeneous ideals can give rise to the same closed subscheme. For example, let $d_0$ be an integer, and let $I^{\prime}=\bigoplus_{d \geqslant d_0} I_d$. Show that $I$ and $I^{\prime}$ determine the same closed subscheme.
\end{enu}
\end{prop}
\begin{prooff}
    (1): Since graded homomorphism preserves order, 
    $\mathfrak{p}\in U\mapsto \varphi^{-1}(\mathfrak{p})$ is a well-define map from 
    $U$ to Proj $S$. Notice that 
    \begin{equation*}
        U=\bigcup_{g\in \varphi(S_+)}D_+(g),
    \end{equation*}
    $U$ is a open subset of Proj $T$. And the morphism of presheaves $f^b$ is induced by the 
    natural local ring homomorphism 
    \begin{equation*}
        \varphi_\mathfrak{p}: S_{(f(\mathfrak{p}))}\rightarrow T_{(\mathfrak{p})}
    \end{equation*}
    And it's easy to check $f$ together with $f^b$ forms a morphism of scheme $(f,f^b):U\rightarrow $ Proj $S$. 

    (2): For $U=\text{Proj} T$, assume  $\mathfrak{p}\supset \varphi(S_+)$ and 
    $\mathfrak{p} \nsupseteq T_+$, 
    there's $a\in T_r$ with $r\ge 1$ such that $a\notin \mathfrak{p}$. 
    Consider the element $a^k$ for $k$ sufficiently large. Next step, we are going to show 
    $f: \text{Proj} T\rightarrow \text{Proj} S$ is an isomorphism. 

Since, $\varphi_d$ are isomorphic for all $d\ge d_0$,
$$\bbrace{\bbrace{D_+(t_i)}: t_i\in T_+,\deg t_i\ge d_0}$$ 
be a open covering of Proj $T$. Put $s_i=\varphi^{-1}\left(t_i\right)$, we also have 
$$
\bbrace{\bbrace{D_+(s_i)}: s_i\in S_+,\deg s_i\ge d_0}
$$
be a open covering of Proj $S$

$f_i=\left.f\right|_{D_{+}\left(t_i\right)} \rightarrow D_{+}\left(s_i\right)$ is a morphism of affine schemes (as $D_{+}\left(t_i\right) \simeq \operatorname{Spec} T_{\left(t_i\right)}$ and $\left.D_{+}\left(s_i\right) \simeq \operatorname{Spec} S_{\left(s_i\right)}\right)$ 
corresponding to the ring homomorphism $\varphi_i: S_{\left(s_i\right)} \rightarrow T_{\left(t_i\right)}$ induced by $\varphi$. 
But $\varphi_i$ is an isomorphism since $s_i$ has degree at least $d_0$, and $\varphi_d$ is an isomorphism for all $d \geq d_0$. Hence, $f$ is surjective. 

To show $f$ is injective, take $\mathfrak{p}_1,\mathfrak{p}_2\in $ Proj $T$ with $f(\mathfrak{p}_1)=f(\mathfrak{P}_2)$. We have 
$\mathfrak{p}_1\cap T_d=\mathfrak{p}_2\cap T_d$ for all $d\ge d_0$. If $t_r\in \mathfrak{p}_1\cap T_r$, 
take $s\notin \mathfrak{p}_2$, we have $s^k t_r\in \mathfrak{p}_2$. It implies $t_r\in \mathfrak{p}_2\cap T_r$. 


    (3):Since $\varphi: S\rightarrow T$ is surjective, $f$ is injective and 
    \begin{equation*}
        \varphi_\mathfrak{p}: S_{(f(\mathfrak{p}))}\rightarrow T_{(\mathfrak{p})}
    \end{equation*}
    is surjective. 
    
    Then, it suffice to check $f(\text{Proj} T)$ is a closed subset. 
    Notice that $\text{Ker}\varphi$ be a homogenous ideal 
    and for all $\mathfrak{p}\in \text{Proj}(T)$, 
    we have   
    \begin{equation*}
        f(\mathfrak{p})\subset \text{Ker}\varphi
    \end{equation*}
    Hence, $f(\text{Proj}(T))\subset V(\text{Ker}\varphi)$. On the other hand, 
    Since $\varphi$ is surjective, $T\simeq S/\text{Ker}\varphi$ as graded ring. Hence, 
    it's easy to show $ V(\text{Ker}\varphi)\subset f(\text{Proj}(T))$. 

    (4): By (2).
\end{prooff}
\begin{exam}
    Let $f_1, \ldots, f_r \in k\left[X_0, \ldots, X_n\right]$ be homogeneous polynomials and let $X=\text{Proj}\,k[X_0,\dots.X_n]/(f_1,
    \dots,f_r)$. 
    For every field extension $k \hookrightarrow K$ we have
    $$
    X(K)=\left\{x=\left(x_0: \ldots: x_n\right) \in \mathbb{P}^n(K):f_1(x)=\cdots=f_r(x)=0\right\}
    $$    
\end{exam}

\begin{defn}[Galois Actions]
    Assume $K/k$ be a Galois extension and $G=\text{Gal}(K/k)$.
    Let $X$ be a $k$-scheme, 
    we obtain an action of $G$ on 
    $X(K)$ by composition of the morphism $x: \operatorname{Spec} K \rightarrow X$ with ${ }^a \sigma: \operatorname{Spec} K \rightarrow \operatorname{Spec} K$ for $\sigma \in G$.
    Hence, by Proposition\,\ref{proposition: K points of k-scheme}, the Galois group action on 
    $X(K)$ by $\sigma\in G$ is actually a transform of $k$-algebra homeomorphism through composition
    \begin{equation*}
        l\in \text{Hom}_k(\kappa(x),K) \mapsto \sigma\circ l\in \text{Hom}_k(\kappa(x),K). 
    \end{equation*}
    Denote the $K$-points which is stable under a subgroup $H$ of $G$ by $X(K)^H$, we have 
    \begin{equation*}
          X(K)^H=X(K^H).
    \end{equation*}
\end{defn}
\begin{prop}
$k$ be a perfect field. Then $\bar{k}/k$ is a Galois extension. Denote $G=\text{Gal}(\bar{k}/k)$.
Let $X$ be a $k$-scheme locally of finite type. 
There's a one-to-one correspondence between $G$-orbits of 
$X(\bar{k})$ and closed point of $X$.
\end{prop}
\begin{prooff}
    Since the point in $X(\bar{k})$ is the pair $(x,l)$, 
    where $x\in X$ and $l$ be a $k$-algebra homeomorphism 
    from $\kappa(x)$ to $\bar{k}$. By Proposition\,\ref{proposition: locally finite type, closed point},
    for all $(x,l)\in X(\bar{k})$, $x$ is a closed point. Moreover, 
    $G$-action doesn't change $x$, so $(x,l)\mapsto x$ be a map from 
    $G$-orbits of $X(\bar{k})$ to closed point of $X$. 
    By Algebra\,\ref{lemma:extend algebraic extension}, $(x,l)\mapsto x$ is surjective. 
    By Numebr Theory Theorem\,\ref{proposition:res between Galois group}, $(x,l)\rightarrow x$ is injective. 
\end{prooff}

















\begin{prop}
    A gluing datum of schemes consists of the following data:
\begin{enu} 
    \item an index set $I$,
    \item for all $i \in I$ a scheme $U_i$,
    \item for all $i, j \in I$ an open subset $U_{i j} \subseteq U_i$ (we consider $U_{i j}$ as open subscheme of $U_i$ ),
    \item for all $i, j \in I$ an isomorphism $\varphi_{j i}: U_{i j} \rightarrow U_{j i}$ of schemes, such that
    $U_{i i}=U_i$ for all $i \in I$

    and the cocycle condition holds: $\varphi_{k j} \circ \varphi_{j i}=\varphi_{k i}$ on $U_{i j} \cap U_{i k}, i, j, k \in I$.
\end{enu}
\end{prop}
\begin{rema}
    In the cocycle condition we implicitly assume that in particular $\varphi_{j i}\left(U_{i j} \cap U_{i k}\right) \subseteq U_{j k}$, such that the composition is meaningful. 
    
    For $i=j=k$, the cocycle condition implies that $\varphi_{i i}=\operatorname{id}_{U_i}$ and for $i=k$  that $\varphi_{i j}^{-1}=\varphi_{j i}$.
    
    Moreover, $\varphi_{j i}$ is an isomorphism $U_{i j} \cap U_{i k} \rightarrow U_{j i} \cap U_{j k}$. This is because, consider the cocycle conditions 
    $\varphi_{k j} \circ \varphi_{j i}=\varphi_{k i}$ and $\varphi_{k i} \circ \varphi_{i j}=\varphi_{k j}$. We obtain two natural morphisms 
    $\varphi_{ji}: U_{i j} \cap U_{i k} \rightarrow U_{j i} \cap U_{j k}$ and $\varphi_{ij}: U_{j i} \cap U_{j k}\rightarrow U_{i j} \cap U_{i k} $. Then, the claim follows from 
    the fact $\varphi_{ji}^{-1}=\varphi_{ij}$.
\end{rema}
\begin{prop}
    Let $\left(\left(U_i\right)_{i \in I},\left(U_{i j}\right)_{i, j \in I},\left(\varphi_{i j}\right)_{i, j \in I}\right)$ be a gluing datum of schemes.
    Then there exists a scheme $X$ together with morphisms $\psi_i: U_i \rightarrow X$, such that
\begin{enu}   
 \item for all $i$ the map $\psi_i$ is a isomorphism of $U_i$ onto the open subscheme $\psi_i(U_i)$ of $X$.
 % https://q.uiver.app/#q=WzAsMyxbMCwxLCJVX2kiXSxbMSwxLCJcXHBzaV9pKFVfaSkiXSxbMSwwLCJYIl0sWzAsMSwiXFxwc2lfaSIsMl0sWzEsMiwibCIsMl0sWzAsMl1d
\[\begin{tikzcd}
	& X \\
	{U_i} & {\psi_i(U_i)}
	\arrow[from=2-1, to=1-2]
	\arrow["{\psi_i}"', from=2-1, to=2-2]
	\arrow["l"', from=2-2, to=1-2]
\end{tikzcd}\]
 \item $\psi_j \circ \varphi_{j i}=\psi_i$ on $U_{i j}$ for all $i, j$,
 \item $X=\bigcup_i \psi_i\left(U_i\right)$,
 \item $\psi_i\left(U_i\right) \cap \psi_j\left(U_j\right)=\psi_i\left(U_{i j}\right)=\psi_j\left(U_{j i}\right)$ for all $i, j \in I$.
\end{enu} 
    Furthermore, $X$ together with the $\psi_i$ is uniquely determined up to unique isomorphism.
\end{prop}
\begin{prooff}
    Underlying topological space: To define the underlying topological space of $X$, we start with the disjoint union $\coprod_{i \in I} U_i$ of the (underlying topological spaces of the) $U_i$ and define an equivalence relation $\sim$ on it as follows: points $x_i \in U_i, x_j \in U_j, i, j \in I$, are equivalent, if and only if $x_i \in U_{i j}$, $x_j \in U_{j i}$ and $x_j=\varphi_{j i}\left(x_i\right)$. The cocycle condition implies that $\sim$ is in fact an equivalence relation. As a set, define $X$ to be the set of equivalence classes,
    $$
    X:=\coprod_{i \in I} U_i / \sim .
    $$
    The natural maps $\psi_i: U_i \rightarrow X$ are injective and we have $\psi_i\left(U_{i j}\right)=\psi_i\left(U_i\right) \cap \psi_j\left(U_j\right)$ for all $i, j \in I$. We equip $X$ with the quotient topology, i. e. with the finest topology such that all $\psi_i$ are continuous. That means that a subset $U \subseteq X$ is open if and only if for all $i$ the preimage $\psi_i^{-1}(U)$ is open in $U_i$. In particular, the $\psi_i\left(U_i\right)$ and the $\psi_i\left(U_{i j}\right)=\psi_i\left(U_i\right) \cap \psi_j\left(U_j\right)$ are open in $X$. 

    Structure sheaf: Define for $W$ open in $X$, 
    \begin{equation*}
        \mathscr{O}_X(W)=\bbrace{(s_i)_{i\in I}: s_i\in \mathscr{O}_{U_i}(W\cap U_i), \varphi_{ji}(\left.s_i\right|_{W\cap U_{ij}})= \left.s_j\right|_{W\cap U_{ji}}    }
    \end{equation*}
    where $W\cap U_i$ is actually $\psi_i^{-1}(W)$. 

    morphism of sheaves $\psi_i^b$: 
    \begin{equation*}
        \psi_i^b: (\psi_i)_*\mathscr{O}_{X}\rightarrow \mathscr{O}_{U_i}, (s_i)_{i\in I}\mapsto s_i
    \end{equation*}
\end{prooff}
\begin{exam}[line with double origin]
    We denote the line with double origin by $X$. It is obtained by gluing Spec$k[u]$ and $\text{Spec}k[t]$ along the 
    isomorphism $D(u)\simeq \spec{k[u,1/u]}\simeq \spec{k[t,1/t]}=D(t)$. 
    Notice taht $(X,\mathscr{O}_X)$ is affine if the morphism $(f,f^b):(X,\mathscr{O}_X)\rightarrow \spec{\Gamma(X,\mathscr{O}_X)}$ induced by $\text{id}:\Gamma(X,\mathscr{O}_X)\rightarrow \Gamma(X,\mathscr{O}_X)$
    is an isomorphism. 

    An element of $\Gamma\left(X, \mathscr{O}_X\right)$ is the same as giving two polynomials $\sum_n f_n u^n$ and $\sum_m g_m t^m$ such that $\sum_n f_n u^n=\sum_m g_m u^m$ in $k[u, 1 / u]$. Note that this just means that $f_n=g_n$ for all $n$. Hence $\Gamma\left(X, \mathcal{O}_X\right)$ is isomorphic to $k[u]$. If $X$ is affine, then we have isomorphisms
    $$
    \left(X, \mathcal{O}_X\right) \xrightarrow{\left(f, f^{b}\right)} \operatorname{Spec} \Gamma\left(X, \mathscr{O}_X\right) \rightarrow \operatorname{Spec}k[u]
    $$
    
    Now consider the vanishing set $V(u)$ of $X$ where $V(f)$ for some $f$ in global section 
    consists of all those points $x \in X$ such that $f_x=0$ modulo $\mathfrak{m}_x$. 
    and $u$ denotes the global section $u=v$ of $\Gamma\left(X, \mathscr{O}_X\right)$. 
    
    Note that $V(u)$ contains at least two points, the two origins of $X$. 
    But $V(u)$ in Spec $k[u]$ consists of only one point. Hence line with double origin is not affine.
\end{exam}
\begin{exam}[projective space]
    $R$ is a ring and $S=R[X_0,\dots,X_n]$ be a graded ring. Consider the scheme $\bb{P}^n_R= \text{Proj} S$. For $f= x_i,i=1,\dots n$, we have 
    $$  
     S_{(f)}=\bbrace{a/X_i^n\in R[X_0,\dots,X_n]_{X_i}: a\in R[X_0,\dots,X_n]_n}=R\left[\frac{X_0}{X_i},\dots,\frac{X_{i-1}}{X_i},\frac{X_{i+1}}{X_i},\dots, \frac{X_n}{X_i} \right]
    $$
    and for $U_i=D_{+}(f)$, 
    $$
    (U_i, \left. \mathscr{O}_{\bb{P}^n_R}\right|_{U_i})=\left(D_{+}(f),\left.\mathscr{O}\right|_{D_{+}(f)}\right) \simeq \text{Spec} R\left[\frac{X_0}{X_i},\dots,\frac{X_{i-1}}{X_i},\frac{X_{i+1}}{X_i},\dots, \frac{X_n}{X_i} \right]
    $$

    We define a gluing datum with index set $\{0, \ldots, n\}$ as follows: For $0 \leq i, j \leq n$ let $U_{i j}=D_{U_i}\left(\frac{X_j}{X_i}\right) \subseteq U_i$ if $i \neq j$, and $U_{i i}=U_i$. Further, let $\varphi_{i i}=\operatorname{id}_{U_i}$ and for $i \neq j$ let
$$
\varphi_{j i}: U_{i j} \rightarrow U_{j i}
$$
be the isomorphism defined by the equality
$$
R\left[\frac{X_0}{X_i}, \ldots, \frac{\widehat{X_i}}{X_i}, \ldots, \frac{X_n}{X_i}\right]_{\frac{X_j}{X_i}} \longleftarrow R\left[\frac{X_0}{X_j}, \ldots, \frac{\widehat{X_j}}{X_j}, \ldots, \frac{X_n}{X_j}\right]_{\frac{X_i}{X_j}},
$$
(as subrings of $R\left[X_0, \ldots, X_n, X_0^{-1}, \ldots, X_n^{-1}\right]$ ) of the affine schemes $U_{i j}$ and $U_{j i}$. 
\end{exam}
\begin{coro}
    If $R=k$ be a ring, the global section of $\bb{P}^n_k$ is $k$. Hence, $\bb{P}^n_k$ is not affine. 
\end{coro}
\begin{exam}[structure of $\bb{P}^1_\bb{R}$]
    For $U_x, U_y$, there are $\bb{R}$-scheme isomorphisms
    $$
    (U_x, \left. \mathscr{O}_{\bb{P}^1_\bb{R}}\right|_{U_x})\simeq \text{Spec} \bb{R}[y]
    $$
    and
    $$
    (U_y, \left. \mathscr{O}_{\bb{P}^1_\bb{R}}\right|_{U_y})\simeq \text{Spec} \bb{R}[x]
    $$ 
    Hence, 
    $$
      \bb{P}^1_\bb{R}=\bbrace{(x-ay): a\in\bb{R}}\bigcup \bbrace{(y-ax): a\in\bb{R}}\bigcup \bbrace{(ax^2+bxy+cy^2): b^2-4ac<0}
    $$
\end{exam}








\newpage 
\begin{defn}
    \begin{enu}
        \item A scheme is called connected, if the underlying topological space is connected.
        \item A scheme is called quasi-compact, if the underlying topological space is quasi-compact, i. e., if every open covering admits a finite subcovering.
        \item A scheme is called irreducible, if the underlying topological space is irreducible, i. e., if it is non-empty and not equal to the union of two proper closed subsets.     
        \item A morphism $f: X \rightarrow Y$ of schemes is called injective, surjective or bijective, respectively, if the continuous map $X \rightarrow Y$ of the underlying topological spaces has this property.
        \item $f$ is called open, closed, or a homeomorphism, respectively, if the underlying continuous map has this property.
        \item $f$ is called dominant if $f(X)$ is a dense subspace of $Y$.
        \item A scheme $X$ is called locally noetherian, if $X$ admits an affine open cover $X=\bigcup U_i$, such that all the affine coordinate rings $\Gamma\left(U_i, \mathscr{O}_X\right)$ are noetherian. If in addition $X$ is quasi-compact, $X$ is called noetherian.
        \item A scheme $X$ is called reduced, if all local rings $\mathscr{O}_{X, x}, x \in X$, are reduced rings.
        \item An integral scheme is a scheme which is reduced and irreducible.
       
    \end{enu}
\end{defn}

\begin{prop}
    Let $X=\operatorname{Spec} A$ be an affine scheme. Then $X$ is noetherian if and only if $A$ is a noetherian ring.
\end{prop}
\begin{prooff}
    By Nike's Trick, $\operatorname{Spec} A$ can be covered 
    by affine open subschemes of the form $D\left(f_i\right), f_i \in A, i=1, \ldots, n$, such that all $A_{f_i}$ are noetherian rings.

    If $I$ is an ideal of $A$, $I_{f_i}=IA_{f_i}$ is finitely generated ideal in $A_{f_i}$. By Algebra\,\ref{proposition: affine noetherian scheme is spec of noetherian ring},
    $I$ is finitely generated ideal in $A$.
\end{prooff}
\begin{prop}
    $X$ is any noetherian scheme, the underlying topological space of $X$ is noetherian 
\end{prop}
\begin{prooff}
    Since spectrum of a noetherian ring is a noetherian topological space. Then this proposition
    follows from the fact 
    that a topological space covered by finite many noetherian subspace is notherian. 

\end{prooff}
\begin{prop}
    Let $X$ be a (locally) noetherian scheme and $U \subseteq X$ an open subscheme. Then $U$ is (locally) noetherian.
\end{prop}
\begin{prooff}
    In a noetherian topological space, every open subset is quasi-compact.
\end{prooff}
\begin{prop}
    Let $X$ be a scheme. The mapping
    $$
    \begin{aligned}
    & X \longrightarrow\{Z \subseteq X ; Z \text { closed, irreducible }\} \\
    & x \mapsto \overline{\{x\}}
    \end{aligned}
    $$
    is a bijection, i. e. every irreducible closed subset contains a unique generic point.
\end{prop}
\begin{prooff}
    Step 1: If $Z$ is a closed irreducible subset of $X$ and $U$ is an affine open subset of $X$, $Z\cap U$ is irreducible.
    This is because, 
    for $W_1,W_2$ be open subets of $X$ and $Y_j=W_j\cap Z\cap U_i\neq \varnothing,j=1,2$, since $Z$ is irreducible, the intersection  
    of $Y_1$ and $Y_2$ is non-empty. Hence, $Z\cap U_i$ is irreducible.

    Step 2: Since $Z$ is closed, $\overline{Z\cap U}\subset Z$. Since $Z$ is irreducible, $Z\cap U$ is a dense subset of $Z$. 
    Then $\overline{Z\cap U}\cap Z=Z$. 

    Step 3: Since $Z\cap U$ is a irreducible closed subset of $U$, there's $x\in Z\cap U$ such that 
    $\overline{\bbrace{x}}\supset \overline{\bbrace{x}}\cap U= Z\cap U$. Hence, $\bbrace{x}\supset \overline{Z\cap U}=Z$.

    Step 4: To show the uniqueness of $x$, consider $x,y\in X$ such that $\overline{\bbrace{x}}=\overline{\bbrace{y}}=Z$ and 
    $U$ be an open affine subset of $X$ with $U\cap Z\neq \varnothing$. 
    Since there's $z\in \overline{\bbrace{x}}\cap U$, we have $x\in U$. Similarly, $y\in U$. Then, $\overline{\bbrace{x}}\cap U=\overline{\bbrace{y}}\cap U=Z\cap U$, 
    by the uniqueness of affine case, $x=y$.


\end{prooff}
\begin{coro}
    $X$ be a scheme, $Z$ be a irreducible closed subset with generic point $\eta$, then there's a one-to-one order preserving 
    correspondence between prime ideal of $\mathscr{O}_{X,\eta}$ and irreducible closed subset of $X$ contains $Z$. 
    \label{proposition: irreducible closed subset, spec of local ring}
    
\end{coro}
\begin{prooff}
    If $\eta\in U=\spec{A}$ be a affine open neighborhood of $\eta$. Then 
    prime ideal of $\mathscr{O}_{X,\eta}$ corresponds to prime ideal of $A$ which is contained in $\mathfrak{p}_\eta$. 

    Let $Z_1=\overline{\bbrace{\theta_1}}, Z_2=\overline{\bbrace{\theta_2}}$ be two irreducible closed subset of $X$ containing $Z$. 
    Then $\theta_1, \theta_2$ lie in $U$. Hence, $Z_1\cap U=Z_2\cap U$ implies $Z_1=Z_2$. 

    Notice that for a irreducible closed subset $V$ contains $Z$, $V\cap U$ correspondes to a prime ideal of $A$ contained in $\mathfrak{p}_\eta$, 
    and by above proposition, this map is injective. Next step we show the map is surjective. 

    If $W$ be a irreducible closed subet of $U$ with $\eta\in W$, then $\overline{W}$ is a irreducible closed subset of $X$ containing 
    $Z$ and $W=\overline{W}\cap U$. 



\end{prooff}












\begin{prop}
    Let $X$ be a scheme, let $f \in \Gamma\left(X, \mathscr{O}_X\right)$, and define $X_f$ to be the subset of points 
    $x \in X$ such that the stalk $f_x$ of $f$ at $x$ is not contained in the maximal ideal $\mathfrak{m}_x$ 
    of the local ring $\mathscr{O}_x$.

    If $U=\operatorname{Spec} B$ is an open affine subscheme of $X$, and 
    if $\bar{f} \in B=\Gamma\left(U,\left.\mathscr{O}_X\right|_U\right)$ is the restriction of $f$, show that $U \cap X_f=D(\bar{f})$. Conclude that $X_f$ is an open subset of $X$.
    \label{proposition: scheme, X_f}
\end{prop}
\begin{prooff}
    For $\mathfrak{p}\in \spec{B}$, $\mathfrak{p}\in D(\bar{f})$ iff $\bar{f}\notin \mathfrak{p}$ iff $\bar{f}$ viewed as an element in $A_\mathfrak{p}$ does not lie in $\mathfrak{p}A_\mathfrak{p}$. 
\end{prooff}
\begin{prop}
\begin{enu}
    \item A scheme $X$ is reduced if and only if for every open subset $U \subseteq X$ the ring $\Gamma\left(U, \mathscr{O}_X\right)$ is reduced.
    \item A non-empty scheme $X$ is integral if and only if for every open subset $\emptyset \neq U \subseteq X$ the ring $\Gamma\left(U, \mathscr{O}_X\right)$ is an integral domain.
    \item If $X$ is an integral scheme, then for all $x \in X$ the local ring $\mathscr{O}_{X, x}$ is an integral domain. 
    \item An affine scheme $X=\operatorname{Spec} A$ is integral if and only if $A$ is a domain. 
    \item Let $X$ be an integral scheme, and let $\eta \in X$ be its generic point. Then the local ring $\mathscr{O}_{X, \eta}$ is a field.

\end{enu}
\end{prop}
\begin{prooff}
    (1): Trivial. 

    (2): Let $X$ be integral. Because all open subschemes of $X$ are integral, too, it is enough to show that $\Gamma\left(X, \mathscr{O}_X\right)$ is a domain. 
    Take $f, g \in \Gamma\left(X, \mathscr{O}_X\right)$ such that $f g=0$. Then $\varnothing=X_f\cap X_g$ since $f_xg_x\in\mathfrak{m}_x$ for all 
    $x\in X$. By the irreducibility we get $X_f=\varnothing$ or $X_g=\varnothing$. 
    Assume $X_f=\varnothing$. 
    We want to show that $f$ must then be $0$. 
    We can check this locally on $X$, so we may assume that $X$ is affine. 
    Then $f$ lies in the intersection of all prime ideals, i. e. in the nil-radical of the affine coordinate ring of $X$. Since $X$ is reduced, 
    by (1) the nil-radical is the zero ideal.

    If conversely all $\Gamma\left(U, \mathscr{O}_X\right)$ are integral domains, then by (1) $X$ is reduced. If there existed non-empty affine open subsets $U_1, U_2 \subseteq X$ with empty intersection, then the sheaf axioms imply that
    $$
    \Gamma\left(U_1 \cup U_2, \mathscr{O}_X\right)=\Gamma\left(U_1, \mathscr{O}_X\right) \times \Gamma\left(U_2, \mathscr{O}_X\right)
    $$
    But the product on the right hand side obviously contains zero divisors.

    (3): Trivial.

    (4): $A$ is integral domain, then it has a unique minimal prime ideal. Hence, $\spec{A}$ is irreducible. 
    Since $A_\mathfrak{{p}}$ is a subring of $\text{Frac}(A)$, $\spec{A}$ is reduced scheme. 
    Hence, $\spec{A}$ is an integral scheme. By (2), if $\spec{A}$ is an integer scheme, $A$ is an integral domain. 

    (5): If $\eta$ is a generic point of $X$, for all affine open subscheme $U$ such that $\eta\in U$, 
    $\eta$ is a generic point of $U=\spec{A}$. That is, $\eta$ corresponds to $(0)$ in $A$. Then, 
    $\mathscr{O}_{X,x}\simeq A_{(0)}=\text{Frac}(A)$ is a field.
\end{prooff}
\begin{defn}
    Let $X$ be an integral scheme, and let $\eta \in X$ be its generic point. Then the local ring $\mathscr{O}_{X, \eta}$ is a field, which is called the function field of $X$ and denoted by $K(X)$.
\end{defn}
\begin{prop}
    $X$ be an integral scheme with generic point $\eta$. 
    \begin{enu}
    \item Let $U \subseteq V \subseteq X$ be non-empty open subsets. Then the maps
        $$
        \Gamma\left(V, \mathscr{O}_X\right) \xrightarrow{\operatorname{res}_U^V} \Gamma\left(U, \mathscr{O}_X\right) \xrightarrow{f \mapsto f_\eta} K(X)
        $$
    \item For all $x\in X$, there's a canonical injective map $\mathscr{O}_{X,x}\rightarrow \mathscr{O}_{X,\eta}$ gievn by $[s]\mapsto [s]$ 
    and under this map, $\text{Frac}(\mathscr{O}_{X,x})=\mathscr{O}_{X,\eta}$.
    \item For every non-empty open subset $U \subseteq X$ and for every covering $U=\bigcup_i U_i$ by non-empty open subsets $U_i$ we have
    $$
    \Gamma\left(U, \mathscr{O}_X\right)=\bigcap_i \Gamma\left(U_i, \mathscr{O}_X\right)=\bigcap_{x \in U} \mathscr{O}_{X, x},
    $$
    where the intersection takes place in $K(X)$.    
\end{enu}
\end{prop}
\begin{prooff}
    (1): It suffice to show the map $f\mapsto f_{\eta}$ is injective. 
    Since $f_\eta=0$ is equivalent to $\left. f\right|_W=0$ for all $W$ open affine subscheme of $U$,
    we may assume $U$ is an affine open subscheme. Consider the following commutative diagram 
    % https://q.uiver.app/#q=WzAsNixbMCwxLCJcXG1hdGhzY3J7T31fWChVKSJdLFswLDAsIlxcbWF0aHNjcntPfV97WCxcXGV0YX0iXSxbMSwxLCJcXG1hdGhzY3J7T31fe1xcdGV4dHtTcGVjfUF9KFxcdGV4dHtTcGVjfUEpIl0sWzIsMSwiQSJdLFsxLDAsIlxcbWF0aHNjcntPfV97XFx0ZXh0e1NwZWN9QSwoMCl9Il0sWzIsMCwiXFx0ZXh0e0ZyYWN9KEEpIl0sWzAsMV0sWzAsMiwiXFxzaW1lcSJdLFsyLDMsIlxcc2ltZXEiXSxbMiw0XSxbMSw0LCJcXHNpbWVxIl0sWzQsNSwiXFxzaW1lcSJdLFszLDVdXQ==
\[\begin{tikzcd}
	{\mathscr{O}_{X,\eta}} & {\mathscr{O}_{\text{Spec}A,(0)}} & {\text{Frac}(A)} \\
	{\mathscr{O}_X(U)} & {\mathscr{O}_{\text{Spec}A}(\text{Spec}A)} & A
	\arrow["\simeq", from=1-1, to=1-2]
	\arrow["\simeq", from=1-2, to=1-3]
	\arrow[from=2-1, to=1-1]
	\arrow["\simeq", from=2-1, to=2-2]
	\arrow[from=2-2, to=1-2]
	\arrow["\simeq", from=2-2, to=2-3]
	\arrow[from=2-3, to=1-3]
\end{tikzcd}\]
Since $A\rightarrow \text{Frac}(A)$ is injective, we have $f=0$.

(2): By (1) and the following diagram

% https://q.uiver.app/#q=WzAsNixbMCwxLCJcXG1hdGhzY3J7T31fe1gseH0iXSxbMCwwLCJcXG1hdGhzY3J7T31fe1gsXFxldGF9Il0sWzEsMSwiXFxtYXRoc2Nye099X3tcXHRleHR7U3BlY31BLFxcbWF0aGZyYWt7cH19Il0sWzIsMSwiQV9cXG1hdGhmcmFre3B9Il0sWzEsMCwiXFxtYXRoc2Nye099X3tcXHRleHR7U3BlY31BLCgwKX0iXSxbMiwwLCJcXHRleHR7RnJhY30oQSkiXSxbMCwxXSxbMCwyLCJcXHNpbWVxIl0sWzIsMywiXFxzaW1lcSJdLFsyLDRdLFsxLDQsIlxcc2ltZXEiXSxbNCw1LCJcXHNpbWVxIl0sWzMsNV1d
\[\begin{tikzcd}
	{\mathscr{O}_{X,\eta}} & {\mathscr{O}_{\text{Spec}A,(0)}} & {\text{Frac}(A)} \\
	{\mathscr{O}_{X,x}} & {\mathscr{O}_{\text{Spec}A,\mathfrak{p}}} & {A_\mathfrak{p}}
	\arrow["\simeq", from=1-1, to=1-2]
	\arrow["\simeq", from=1-2, to=1-3]
	\arrow[from=2-1, to=1-1]
	\arrow["\simeq", from=2-1, to=2-2]
	\arrow[from=2-2, to=1-2]
	\arrow["\simeq", from=2-2, to=2-3]
	\arrow[from=2-3, to=1-3]
\end{tikzcd}\]

(3): Consider the following commutative diagram
% https://q.uiver.app/#q=WzAsNCxbMCwwLCJcXEdhbW1hKFUsXFxtYXRoc2Nye099X1gpIl0sWzEsMCwiXFxHYW1tYShVX2ksXFxtYXRoc2Nye099X3tYfSkiXSxbMiwxLCJcXEdhbW1hKFVfaVxcY2FwIFVfaixcXG1hdGhzY3J7T31fe1h9KSJdLFsyLDAsIlxcbWF0aHNjcntPfV97WCxcXGV0YX0iXSxbMCwxXSxbMSwyXSxbMiwzXSxbMSwzXV0=
\[\begin{tikzcd}
	{\Gamma(U,\mathscr{O}_X)} & {\Gamma(U_i,\mathscr{O}_{X})} & {\mathscr{O}_{X,\eta}} \\
	&& {\Gamma(U_i\cap U_j,\mathscr{O}_{X})}
	\arrow[from=1-1, to=1-2]
	\arrow[from=1-2, to=1-3]
	\arrow[from=1-2, to=2-3]
	\arrow[from=2-3, to=1-3]
\end{tikzcd}\]
and notice that $\mathscr{O}_X$ is a sheaf.
\end{prooff}

Notice we define locally finite type and finite type $k$-scheme. The morphisms below are all in 
the category $(\text{Sch}/k)$. 
\begin{defn} 
Let $k$ be a field, and let $X \rightarrow \operatorname{Spec} k$ be a $k$-scheme. 
We call $X$ a $k$-scheme locally of finite type or say that $X$ is locally of finite type over $k$, if there is an affine open cover $X=\bigcup_{i \in I} U_i$ such that for all $i$, there's a $k$-algebra $A_i$ such that  
\begin{equation*}
    (U_i,\left.\mathscr{O}_{X}\right|_{U_i})\simeq (\spec{A_i},\mathscr{O}_{\spec{A_i}})
\end{equation*}
as $k$-scheme.
We say that $X$ is of finite type over $k$ if $X$ is locally of finite type and quasi-compact.
\end{defn}
\begin{prop}
    Every (locally) finite type $k$-scheme is (locally) noetherian.
\end{prop}
\begin{prop}
    Let $X$ be a locally noetherian scheme. Prove that the set of irreducible components of $X$ is locally finite ( every point of $X$ has an open neighborhood which meets only finitely many irreducible components of $X$ ).
    \label{proposition: irreducible components, locally finite}

\end{prop}
\begin{prooff}
    Take $x\in U=\spec{A}$ with $A$ noetherian.
    Assume $Z_i,i=1,\dots,n $ be irreducible components of $U$ and 
    $\overline{Z_i}$ is contained in an irreducible component $V_i$ of $X$. 
    Then $Z_i= \overline{Z_i}\cap U\subset V_i\cap U$.
    Since $V_i=\overline{\bbrace{\theta_i}}$, $V_i\cap U$ is the closure of $\theta_i$ in $U$ hence irreducible closed in $U$.
    Since $Z_i$ is maximal, $Z_i=V_i\cap U$. 
     
    Take $Z=\overline{\bbrace{y}}$ be irreducible component of $X$ 
    such that $x\in Z$, then $y\in U$. Hence, 
    $\overline{\bbrace{y}}\cap U \subset \overline{\bbrace{\theta_i}}\cap U$ for some $i$, 
    hence $\bbrace{y}\subset \overline{\bbrace{y}}\cap U \subset \overline{\bbrace{\theta_i}}$. 
    Therefore, we have $Z=V_i$.  

\end{prooff}




\begin{prop}
    Let $X$ be a $k$-scheme locally of finite type and let $U \subseteq X$ be an open affine subset. Then the $k$-algebra $\Gamma\left(U, \mathscr{O}_X\right)$ is a finitely generated $k$-algebra.
    \label{proposition: affine subscheme is finitely-generated k algebra}
\end{prop}
\begin{prooff}
   Let $B=\Gamma\left(U, \mathscr{O}_X\right)$. Since the localization of a finitely generated $k$-algebra with respect to a single element is again finitely generated, 
   we see, by Nike's Trick, 
   that we can cover $U$ by finitely many( since spectrum of a ring is compact ) 
   principal open subsets $D\left(f_i\right), f_1, \ldots, f_n \in B$, such that all localizations $B_{f_i}$ are finitely generated $k$-algebras. The claim now follows from Algebra
   Proposition\,\ref{proposition: local at singe element, finite k-algebra}
\end{prooff}
\begin{prop}
    Let $k$ be a field, let $X$ be a $k$-scheme locally of finite type, and let $x \in X$. Then the following assertions are equivalent.
\begin{enu}
    \item The point $x \in X$ is closed.
    \item The field extension $k \hookrightarrow \kappa(x)$ is finite.
    \item The field extension $k \hookrightarrow \kappa(x)$ is algebraic. 
\end{enu}    
\label{proposition: locally finite type, closed point}
\end{prop}
\begin{prooff}
    (1) implies (2): Take $U$ with $x\in U$ and there's $k$-scheme 
    \begin{equation*}
        (U,\left.\mathscr{O}_{X}\right|_{U})\simeq (\spec{A},\mathscr{O}_{\spec{A}})
    \end{equation*}
    where $A$ be a finitely generated $k$-algebra and $x$ corresponds to a maximal ideal $\mathfrak{m}$ of $A$.
    Consider the follow commutative diagram 
    % https://q.uiver.app/#q=WzAsNyxbMSwzLCJrIl0sWzAsMiwiXFxHYW1tYShVLFxcbWF0aHNjcntPfV9YKSJdLFsyLDIsIkEiXSxbMiwxLCJBX1xcbWF0aGZyYWt7bX0iXSxbMCwxLCJcXG1hdGhzY3J7T31fe1gseH0iXSxbMCwwLCJcXGthcHBhKHgpIl0sWzIsMCwiQS9cXG1hdGhmcmFre219Il0sWzAsMV0sWzAsMl0sWzIsM10sWzEsNF0sWzQsMywiXFxzaW1lcSJdLFsxLDIsIlxcc2ltZXEiXSxbNCw1XSxbMyw2XSxbNSw2LCJcXHNpbWVxIl1d
\[\begin{tikzcd}
	{\kappa(x)} && {A/\mathfrak{m}} \\
	{\mathscr{O}_{X,x}} && {A_\mathfrak{m}} \\
	{\Gamma(U,\mathscr{O}_X)} && A \\
	& k
	\arrow["\simeq", from=1-1, to=1-3]
	\arrow[from=2-1, to=1-1]
	\arrow["\simeq", from=2-1, to=2-3]
	\arrow[from=2-3, to=1-3]
	\arrow[from=3-1, to=2-1]
	\arrow["\simeq", from=3-1, to=3-3]
	\arrow[from=3-3, to=2-3]
	\arrow[from=4-2, to=3-1]
	\arrow[from=4-2, to=3-3]
\end{tikzcd}\]
Since $A/\mathfrak{m}$ is a field and finite generated $k$-algebra, by Algebra\,\ref{theorem:finite generated,field}, $\kappa(x)$ is a finte extension of $k$. 

(3) implies (1): Again take $U$ with $x\in U$ and there's $k$-scheme 
\begin{equation*}
    (U,\left.\mathscr{O}_{X}\right|_{U})\simeq (\spec{A},\mathscr{O}_{\spec{A}})
\end{equation*}
where $A$ be a finitely generated $k$-algebra and $x$ corresponds to a prime ideal $\mathfrak{p}$ of $A$.
% https://q.uiver.app/#q=WzAsOSxbMSwzLCJrIl0sWzAsMiwiXFxHYW1tYShVLFxcbWF0aHNjcntPfV9YKSJdLFsyLDIsIkEiXSxbMiwxLCJBX1xcbWF0aGZyYWt7cH0iXSxbMCwxLCJcXG1hdGhzY3J7T31fe1gseH0iXSxbMCwwLCJcXGthcHBhKHgpIl0sWzIsMCwiQV9cXG1hdGhmcmFre3B9L1xcbWF0aGZyYWt7cH1BX1xcbWF0aGZyYWt7cH0iXSxbMywxLCJBL1xcbWF0aGZyYWt7cH0iXSxbMywwLCJcXHRleHR7RnJhY31BL1xcbWF0aGZyYWt7cH0iXSxbMCwxXSxbMCwyXSxbMSw0XSxbNCwzLCJcXHNpbWVxIl0sWzEsMiwiXFxzaW1lcSJdLFs0LDVdLFszLDZdLFs1LDYsIlxcc2ltZXEiXSxbMiwzXSxbMiw3XSxbNyw4XSxbNiw4LCJcXHNpbWVxIl1d
\[\begin{tikzcd}
	{\kappa(x)} && {A_\mathfrak{p}/\mathfrak{p}A_\mathfrak{p}} & {\text{Frac}A/\mathfrak{p}} \\
	{\mathscr{O}_{X,x}} && {A_\mathfrak{p}} & {A/\mathfrak{p}} \\
	{\Gamma(U,\mathscr{O}_X)} && A \\
	& k
	\arrow["\simeq", from=1-1, to=1-3]
	\arrow["\simeq", from=1-3, to=1-4]
	\arrow[from=2-1, to=1-1]
	\arrow["\simeq", from=2-1, to=2-3]
	\arrow[from=2-3, to=1-3]
	\arrow[from=2-4, to=1-4]
	\arrow[from=3-1, to=2-1]
	\arrow["\simeq", from=3-1, to=3-3]
	\arrow[from=3-3, to=2-3]
	\arrow[from=3-3, to=2-4]
	\arrow[from=4-2, to=3-1]
	\arrow[from=4-2, to=3-3]
\end{tikzcd}\]
Since $\kappa(x)$ is algebraic over $k$, $A/\mathfrak{p}$ is integral over $k$. Hence $\mathfrak{p}$ is a closed point in $U$. 
Consider all such $U$, we have $x$ is closed in $X$. 







\end{prooff}
\begin{coro}
    Let $k$ be algebraically closed and let $X$ be a $k$-scheme locally of finite type. Then
    $$
    \{x \in X ; x \text { closed }\}=\{x \in X ; k=\kappa(x)\}=\operatorname{Hom}_k(\operatorname{Spec} k, X),
    $$
\end{coro}
\begin{prooff}
    Field extension $k\rightarrow \kappa(x)$ is an isomorphism if and only if there's 
    $k$-algebra homomorphism $\kappa(x)\rightarrow k$. And if there's  $k$-algebra homomorphism $\kappa(x)\rightarrow k$, it is obviously unique. 
\end{prooff}
\begin{exam}
    $\bb{P}^n_k$ is an integral, finite type scheme over $k$.
\end{exam}
\begin{prooff}
    reduced: $\bb{P}^n_k$ is reduced since for all $x\in \bb{P}^n_k$, we may find $i\in\bbrace{0,\dots,n}$ such that 
    $x\in U_i=D_+(x_i)$. Then $\mathscr{O}_{\bb{P}^n_k,x}$ is a localization of a polynomial ring at a prime ideal, hence reduced. 

    irreducible: $D_+(f)\cap D_+(g)=D_+(fg)$ and notice that for all $h\in k[x_0,\dots,x_n]_+$, $D_+(h)$ is non-empty. 

    locally finite type: trivial 

    quasi-compact: $\bb{P}^n_k$  is a finite union of compact open subset $U_i$.
\end{prooff}
\begin{exam}
    For $X=\spec{\bb{Q}[x,y]/(x^n+y^n-1)}$ be a $\bb{Q}$-scheme, then 
    \begin{equation*}
        X(\bb{R})=\text{Hom}_{\spec{\bb{Q}}}(X,\spec{\bb{R}})=\text{Hom}_{\bb{Q}}(\bb{Q}[x,y]/(x^n+y^n-1),\bb{R})=\bbrace{(x,y)\in\bb{R}^2:x^n+y^n=1}
    \end{equation*}
    and 
    \begin{equation*}
        X(\bb{Q})=\text{Hom}_{\spec{\bb{Q}}}(X,\spec{\bb{Q}})=\text{Hom}_{\bb{Q}}(\bb{Q}[x,y]/(x^n+y^n-1),\bb{Q})=\bbrace{(x,y)\in\bb{Q}^2:x^n+y^n=1}
    \end{equation*}
    Moreover, since closed point corresponds to the maximal ideal of $\bb{Q}[x,y]/(x^n+y^n-1)$ and $X(\bb{Q})$ be those maximal ideals $\mathfrak{m}$
    of $\bb{Q}[x,y]$ which contain $x^n+y^n-1$ and have a $\bb{Q}$-algebra isomorphism $\bb{Q}[x,y]/\mathfrak{m}\rightarrow \bb{Q}$. Therefore, $\mathfrak{m}$ is 
    of the form $(x-x_0,y-y_0)$ where $(x_0,y_0)$ be a solution of $x^n+y^n=1$.
\end{exam}

\section{Immersions}
\begin{defn}
    A morphism $j: Y \rightarrow X$ of schemes is called an open immersion, if the underlying continuous map is a homeomorphism of $Y$ with an open subset $U$ of $X$, and the sheaf homomorphism $\mathscr{O}_X \rightarrow j_* \mathscr{O}_Y$ induces an isomorphism $\mathscr{O}_{X \mid U} \cong\left(j_* \mathscr{O}_Y\right)_{\mid U}$ (of sheaves on $U)$.
\end{defn}
\begin{rema}
    There's a natural one-to-one correspondence between open immersion and open subscheme. 
\end{rema}
\begin{defn}
    Given a scheme $\left(X, \mathscr{O}_X\right)$, we call a subsheaf $\mathscr{J} \subseteq \mathscr{O}_X$ a sheaf of ideals, if for every open subset $U \subseteq X$ the sections $\Gamma(U, \mathscr{J})$ are an ideal in $\Gamma\left(U, \mathscr{O}_X\right)$. The quotient sheaf 
    $\mathscr{O}_X / \mathscr{J}$ is defined as the sheafification of 
    the presheaf $U \mapsto \mathscr{O}_X(U) / \mathscr{J}(U)$. 
    It is a sheaf of rings. 
    The canonical projection $\mathscr{O}_X \rightarrow \mathscr{O}_X / \mathscr{J}$ is surjective. 
\end{defn}
\begin{defn}
    Let $X$ be a scheme.

    (1) $A$ closed subscheme of $X$ is given by a closed subset $Z \subseteq X$ with inclusion map $i:Z\rightarrow X$ and an ideal sheaf $\mathscr{J} \subseteq \mathscr{O}_X$ such that $Z=\left\{x \in X :\left(\mathscr{O}_X / \mathscr{J}\right)_x \neq 0\right\}$ and $\left(Z,i^{-1}\mathscr{O}_X/\mathscr{J}\right)$ is a scheme.
    
    (2) A morphism $i: (Z,\mathscr{O}_Z)\rightarrow (X,\mathscr{O}_X)$ of schemes is called a closed immersion, if the underlying continuous map is a homeomorphism between $Z$ and a closed subset of $X$, and the sheaf homomorphism $i^b: \mathscr{O}_X \rightarrow i_* \mathscr{O}_Z$ is surjective.
\end{defn}
\begin{prop}
    $X$ be a scheme and $Z$ be a closed subscheme associated to ideal sheaf $\mathscr{J}$. Then, the morphism of ringed space 
    $(Z,i^{-1}\mathscr{O}_X / \mathscr{J})\rightarrow (X,\mathscr{O}_X)$ induced by the natural projection 
    $\mathscr{O}_X\rightarrow \mathscr{O}_X / \mathscr{J}$ and the 
    isomorphism $\mathscr{O}_X / \mathscr{J}\rightarrow i_*i^{-1}\mathscr{O}_X / \mathscr{J}$ is a morphism of locally ringed space and closed immersion.
\end{prop}
\begin{prooff}
   Step 1: The stalk of 
   the morphism of sheaves $\mathscr{O}_X \rightarrow \mathscr{O}_X/\mathscr{J}$ is a local ring homomorphism. 

   It's clear for the case when $x\notin Z$, since $(\mathscr{O}_X/\mathscr{J})_x=0$. 
   For $x\in Z$, since the stalk of the presheaf $U \mapsto \mathscr{O}_X(U) / \mathscr{J}(U)$ at $x$ is 
   $\mathscr{O}_{X,x}/\mathscr{J}_x$ where $\mathscr{J}_x\neq \mathscr{O}_{X,x}$. And 
   notice that the projection $\mathscr{O}_{X,x}\rightarrow \mathscr{O}_{X,x}/\mathscr{J}_x$ is a local ring homomorphism.

   Step 2: $\mathscr{O}_X\rightarrow \mathscr{O}_X / \mathscr{J}$ is surjective. 

   By taking stalks, it suffice to show $\mathscr{O}_X(U)\rightarrow \mathscr{O}_X(U)/\mathscr{J}(U)$ is surjective for all $U$ open in $X$.
\end{prooff}
\begin{prop}
    If $i: (Z,\mathscr{O}_Z)\rightarrow (X,\mathscr{O}_X)$ is a closed immersion, consider the kernel of the morphism of sheaves $\varphi:\mathscr{O}_X\rightarrow i_{*}\mathscr{O}_Z$. 
    It's clear that $\text{Ker}\varphi$ is an ideal sheaf. By Proposition\,\ref{proposition:sheafification}, the natural morphism 
    \begin{equation*}
        \mathscr{O}_X/\text{Ker}\varphi\rightarrow i_*\mathscr{O}_Z
    \end{equation*}
    is an isomorphism of sheaves. 

    Moreover, since $(Z,\mathscr{O}_Z)$ is a scheme and $Z$ is closed in $X$, 
    $\text{Supp}(i_*\mathscr{O}_Z)=Z$. Hence the support of 
    $\mathscr{O}_X/\text{Ker}\varphi$ is $Z$. Then by Proposition\,\ref{proposition: closed subset, sheaf correspondence}, 
    a closed immersion induces a closed subscheme.
\end{prop}
\begin{theo}[closed subscheme of affine scheme]
    Let $X=\operatorname{Spec} A$ be an affine scheme. $i:(Z,\mathscr{O}_Z)\rightarrow (X,\mathscr{O}_X)$ be a closed immersion. 
    Then the global section map 
    \begin{equation*}
        \varphi: A\rightarrow \Gamma(Z,\mathscr{O}_Z)
    \end{equation*}
    induces a commutative diagram of scheme: 
% https://q.uiver.app/#q=WzAsMyxbMCwwLCJcXHRleHR7U3BlY31BIl0sWzEsMCwiWiJdLFswLDEsIlxcdGV4dHtTcGVjfUEvXFx0ZXh0e2tlcn1cXHZhcnBoaSJdLFsxLDIsIlxccHNpIl0sWzIsMCwiXFxwaSJdLFsxLDAsImkiLDJdXQ==
\[\begin{tikzcd}
	{\text{Spec}A} & Z \\
	{\text{Spec}A/\text{ker}\varphi}
	\arrow["i"', from=1-2, to=1-1]
	\arrow["\psi", from=1-2, to=2-1]
	\arrow["\pi", from=2-1, to=1-1]
\end{tikzcd}\]
Then $\psi$ is an isomorphism of scheme. 
\end{theo}
\begin{prooff}
    Since $i$ is an closed immersion, $\psi$ is closed and injective. Hence, 
    $\psi$ is also a closed immersion.  
    To prove $\psi$ is surjective, it suffices to show the following lemma: 
    \begin{lem}
        Let $X=\operatorname{Spec} A$ be an affine scheme. $i:(Z,\mathscr{O}_Z)\rightarrow (X,\mathscr{O}_X)$ be a closed immersion such that the induced map 
        on global section $ \varphi: A\rightarrow \Gamma(Z,\mathscr{O}_Z) $ is injective. Then, $i$ is surjective. 
    \end{lem}
    \begin{proofff} 
        Assume $X-Z$ is non-empty. 
        Let $s \in A$ with $\varnothing \neq D(s)=X_s\subset X-Z$. Then $Z\subset X-X_s$. Hence, 
        $Z\subset Z\cap(X-X_s)$. For all $x\in Z$, we have the following commutative diagram 
        % https://q.uiver.app/#q=WzAsNCxbMCwwLCJcXEdhbW1hKFosXFxtYXRoc2Nye099X1opIl0sWzAsMSwiIFxcbWF0aHNjcntPfV97Wix4fT1pXyooIFxcbWF0aHNjcntPfV97Wn0pX3giXSxbMSwwLCJBIl0sWzEsMSwiXFxtYXRoc2Nye099X3tYLHh9Il0sWzIsMF0sWzIsM10sWzMsMV0sWzAsMV0sWzEsMF1d
\[\begin{tikzcd}
	{\Gamma(Z,\mathscr{O}_Z)} & A \\
	{ \mathscr{O}_{Z,x}=i_*( \mathscr{O}_{Z})_x} & {\mathscr{O}_{X,x}}
	\arrow[from=1-1, to=2-1]
	\arrow[from=1-2, to=1-1]
	\arrow[from=1-2, to=2-2]
	\arrow[from=2-1, to=1-1]
	\arrow[from=2-2, to=2-1]
\end{tikzcd}\]
Hence, $Z\subset Z\cap (X-X_s)\subset Z-Z_{\varphi(s)}$. 
If $U \subseteq Z$ is open, such that $\left(U, \mathscr{O}_{Z \mid U}\right)\simeq\spec{B}$ is affine. By 
Proposition\,\ref{proposition: scheme, X_f}, $U\subset \spec{B}-D(\left. \varphi(s)\right|_U)$. Hence, 
$\left.\varphi(s)\right|_U$ is nilpotent. Moreover, since $Z$ can be covered by finite many affine open subscheme, 
there's some sufficiently large $N$ such that $\varphi(s)^N$ is nilpotent. Hence, $s^N=0$. 
It contradicts to $\varnothing \neq X_s$. 
    \end{proofff}
To show that $\psi$ is an isomorphism of scheme. We still need the following lemma 
\begin{lem}
    Let $X=\operatorname{Spec} A$ be an affine scheme. $(i,i^b):(Z,\mathscr{O}_Z)\rightarrow (X,\mathscr{O}_X)$ be a closed immersion such that the induced map 
    on global section $ \varphi: A\rightarrow \Gamma(Z,\mathscr{O}_Z) $ is injective. Then, $i^b$ is injective.
\end{lem}
\begin{proofff}
    For $x \in X, \mathscr{O}_{X, x}=A_{\mathfrak{p}_x}$, and we see that it is enough to show that every element of $\operatorname{Ker}\left(\mathscr{O}_{X, x} \rightarrow \mathscr{O}_{Z, x}\right)$ of the form $g/1$ is 0 in $\mathscr{O}_{X, x}$. Given $g$, we cover $Z=U \cup \bigcup_{i \in I} U_i$ by finitely many open subsets $U, U_i$, such that:
    (1) The schemes $\left(U, \mathscr{O}_{Z \mid U}\right)$ and $\left(U_i, \mathscr{O}_{Z \mid U_i}\right)$ for all $i$ are affine.
    (2) We have $x \in U$ and $\varphi(g)_{\mid U}=0$.
    
    Choose $s \in A$ with $x \in D(s) \subseteq U$. If we can show that 
    $\varphi\left(s^N g\right)=0$ for some $N$, then $s^N g=0$ because $\varphi$ is injective, and it follows that $g/1=0$ in $\mathscr{O}_{X, x}$, as desired, since $s$ is a unit in $\mathscr{O}_{X, x}$. Since $\varphi(g)_{\mid U}=0$ by assumption, we have $\varphi(s g)_{\mid U}=0$. Now $I$ is finite, so we can search a suitable $N$ for each $U_i$ separately. 
    Because 
    $$
    D_{U_i}\left(\varphi(s)_{\mid U_i}\right)= Z_{\varphi(s)} \cap U_i \subset D(s)\cap U_i$$
    , we obtain $\varphi(g)_{\mid D_{U_i}\left(\varphi(s)_{\mid U_i}\right)}=0$.
    In other words, the image of $\varphi(g)$ in the localization $\Gamma\left(U_i, \mathscr{O}_Z\right)_{\left.\varphi(s)\right|_{U_i}}$ is $0$.
\end{proofff}
\end{prooff}
\begin{defn}[immersion]
    (1) Let $X$ be a scheme. A subscheme of $X$ is a scheme $\left(Y, \mathscr{O}_Y\right)$, such that $Y \subseteq X$ is a locally closed subset, and such that $Y$ is a closed subscheme of the open subscheme $U \subseteq X$, where $U$ is the largest open subset of $X$ which contains $Y$ and in which $Y$ is closed. 
    We then have a natural morphism of schemes $Y \rightarrow X$.
    
    (2) An immersion $i: Y \rightarrow X$ is a morphism of schemes whose underlying continuous map is a homeomorphism of $Y$ onto a locally closed subset of $X$, and such that for all $y \in Y$ the ring homomorphism $i_y^{\sharp}: \mathscr{O}_{X, i(y)} \rightarrow \mathscr{O}_{Y, y}$ between the local rings is surjective.
    
    It's easy to check there's one-to-one correspondence between immersion and open subscheme. 
\end{defn}
\begin{defn}
    Let $k$ be a field.
\begin{enu}
    \item A $k$-scheme $X$ is called projective, if there exist $n \geq 0$ and a closed immersion $X \hookrightarrow \mathbb{P}_k^n$.
    \item  A $k$-scheme $X$ is called quasi-projective, if there exist $n \geq 0$ and an immersion $X \hookrightarrow \mathbb{P}_k^n$.
\end{enu}
\end{defn}
% \begin{prop}
%     Let $\mathbf{P}$ be the property of a morphism of schemes being an "open immersion" (resp. a "closed immersion", resp. an "immersion").
% \begin{enu} 
%     \item The property $\mathbf{P}$ is local on the target, i.e.: If $f: Z \rightarrow X$ is a morphism of schemes, and $X=\bigcup_i U_i$ is an open covering, then $f$ has $\mathbf{P}$ if and only if for all $i$ the restriction $f^{-1}\left(U_i\right) \rightarrow U_i$ of $f$ satisfies $\mathbf{P}$.
%     \item The composition of two morphisms having property $\mathbf{P}$ has again property $\mathbf{P}$.
% \end{enu}
% \end{prop}
\begin{defn}
    We say that a proposition is local on the target if for every morphism $f: X \rightarrow Y$ of schemes and for every open covering $Y=\bigcup_{j \in J} V_j$ the morphism $f$ possesses the proposition  if and only if $f_{\mid f^{-1}\left(V_j\right)}: f^{-1}\left(V_j\right) \rightarrow V_j$ possesses the proposition for all $j \in J$.
\end{defn}



\begin{prop}
    Open immersion, closed immersion, immersion are stable under base change and composition. 
\end{prop}
\begin{prop}
    Open immersion, closed immersion, immersion are local on target.
\end{prop}


\begin{prop}
    Every affine $k$-scheme $X$ of finite type is quasi-projective:
    Indeed, let $X=\operatorname{Spec} A$, where $A \cong k\left[T_1, \ldots, T_n\right] / \mathfrak{a}$. Therefore there exists a closed immersion $i: X \rightarrow \mathbb{A}_k^n$. 
    Moreover, projective space $\mathbb{P}_k^n$ is covered by open subschemes which are isomorphic to $\mathbb{A}_k^n$. 
    Hence, the composition $j \circ i$ is then an immersion $X \rightarrow \mathbb{P}_k^n$.
\end{prop}
\begin{exam}
    Consider $X=\text{Proj} \bb{C}[x,y,z]/(zy^2-(4x^3-a_2xz^2-a_3z^3))$  
    with $a_2^3-27a_2^2\neq 0$ as a $\bb{C}$- scheme. Firstly, the natural morphism 
    $X\rightarrow \bb{P}^2_\bb{C}$ is a closed immersion. 
    Moreover, consider the $\bb{C}$-points $X(\bb{C})$ of $X$. We have 
    \begin{equation*}
        X(\bb{C})=\bbrace{\infty}\cup \bbrace{(x,y)\in\bb{C}^2:y^2=4x^3-a_2x-a_3}
    \end{equation*}
    where $\infty$ represents the point $(x,z)$ in Proj $\bb{C}[x,y,z]/(zy^2-(4x^3-a_2xz^2-a_3z^3))$.

    Now we show that $X$ is an integral, projective , $\bb{C}$-finite type scheme. 

    irreducible: show that $D_+(z)\cap D_+(f)\neq \varnothing$ for all $D_+(f)\neq \varnothing$.

    reduced: It suffice to show $$\spec{\bb{C}[x,y]/(y^2-(4x^3-a_2x-a_3))}$$ is integral and 
    $$\spec{\bb{C}[x,z]/(z-(4x^3-a_2xz^2-a_3z^3))}$$ is integral. 

    $\bb{C}$-finite type: trivial.
\end{exam}
\begin{defn}[reduced subscheme of a scheme]
    % Let $X$ be a scheme. Denote by $\mathscr{N}:=\mathscr{N}_X \subset \mathscr{O}_X$ the sheaf of ideals which is the sheaf associated to the presheaf
    % $$
    % U \mapsto \operatorname{nil}\left(\Gamma\left(U, \mathscr{O}_X\right)\right), \quad U \subseteq X \text { open }
    % $$
    % % Firstly, we show that for all affine open subscheme $U$ of $X$, $\mathscr{N}_X(U)=\operatorname{nil}\left(\Gamma\left(U, \mathscr{O}_X\right)\right)$. 
    % % By Remark\,\ref{proposition: subsheaf, sheafification}, it suffice to show 
    % % $$
    % % U \mapsto \operatorname{nil}\left(\Gamma\left(U, \mathscr{O}_X\right)\right), \quad U \subseteq X \text { open }
    % % $$ 
    % % is a sheaf if $X$ is affine. 
    % % Assume $X=\spec{A}$,  
    % For $x\in X$, there's $U$ open in $X$ such that $(U,\left.\mathscr{O}_X\right|_U)\simeq (\spec{A},\mathscr{O}_{\spec{A}})$ and 
    % the point $x$ correspondes to prime ideal $\mathfrak{p}$.
    % Notice that 
    % \begin{equation*}
    %     (\mathscr{O}_X/\mathscr{N})_x\simeq \mathscr{O}_{X,x}/\operatorname{nil}(\mathscr{O}_{X,x})\simeq A_{\mathfrak{p}}/\text{nil}(A_{\mathfrak{p}})\simeq (A/\text{nil}(A))_\mathfrak{p}\neq 0, 
    % \end{equation*}
    % we have $(X,\mathscr{O}_X/\mathscr{N})$ is a closed subscheme of $X$ with the same underlyinhg topological space. 
    % More precisely, for all $x\in X$, there's $U$ open in $X$ such that 
    % $(U,\left. \mathscr{O}_X/\mathscr{N}\right|_U)\simeq (\spec{A/\text{nil}(A)})$. 
    Let $X$ be a scheme. Let $T \subset X$ be a closed subset. There exists a closed subscheme 
    $Z \subset X$ with the following properties: 
    \begin{enu} 
    \item the underlying topological space of $Z$ is equal to $T$, 
    \item $Z$ is reduced.
    \end{enu}
    If $T=X$, we usually denote the resulting closed subscheme   
    by $X_{red}$. 
    \label{definition: reduced subscheme}
\end{defn}
\begin{prooff}
    Let $\mathcal{I} \subset \mathcal{O}_X$ be the sub presheaf defined by the rule
    $$
    \mathcal{I}(U)=\left\{f \in \mathcal{O}_X(U) \mid f(t)=0 \text { for all } t \in T \cap U\right\}
    $$
    Here we use $f(t)$ to indicate the image of $f$ in the residue field $\kappa(t)$ of $X$ at $t$. 
    Because of the local nature of the condition it is clear that $\mathcal{I}$ is a sheaf of ideals. 
    It's easy to check the 
    stalk of $\mathscr{O}_X/\mathcal{I}$ at $x\notin Z$ vanishes. 
    And for $x\in Z$, there's open subscheme $x\in U=\spec{R}$ where $x$ correspondes to prime ideal $\mathfrak{p}$.

    Let $I$ be the unique radical ideal correspondes 
    to closed subset $Z\cap U$. 
    It's easy to check 
    \begin{equation*}
        (\mathcal{O}_X/\mathcal{I})_x\simeq R_\mathfrak{p}/ I_\mathfrak{p}=(R/I)_\mathfrak{p}
    \end{equation*} 
    Hence, $(Z\cap U,\left. i^{-1}(\mathscr{O}_X/\mathcal{I})\right|_{U\cap Z})\simeq \spec{A/I}$. 
    So, $(Z, i^{-1}(\mathscr{O}_X/\mathcal{I}))$ is a reduced, closed subscheme of $X$ with under lying topological space $T$. 

\end{prooff}
\begin{coro}
    $X$ be a scheme. $Z$ be a closed, quasi-compact subset. Then there's a closed point in $Z$. 
    \label{corollary: quasi-compact, closed subset contains closed point}
\end{coro}
\begin{prooff}
    By Proposition\,\ref{definition: reduced subscheme}, it suffice to show for a quasi-compact scheme $X$, there's a closed point in $X$. 
    Assume $X$ is covered by affine open subscheme $U_i,i=1,\dots,n $ and $U_i $ is not contained in 
    $$ 
      \bigcup_{j\neq i} U_j 
    $$
    Since $U_1$ is affine, there's $p$ closed in $U_1$. Notice that 
    $$ 
      U_1 \cap (\bigcup_{j\neq 1} U_j)^c=\varnothing
    $$
    , we have $p$ closed in $X$.

\end{prooff}







\begin{prop}[category of schemes to the category of reduced schemes]
    For every morphism of schemes $f: X \rightarrow Y$ there exists a unique morphism of schemes $f_{\mathrm{red}}: X_{\mathrm{red}} \rightarrow Y_{\text {red }}$ such that
    commutes, where $i_X$ and $i_Y$ are the canonical inclusion morphisms.
    % https://q.uiver.app/#q=WzAsNCxbMCwwLCJYX3tcXHRleHR7cmVkfX0iXSxbMSwwLCJYIl0sWzEsMSwiWSJdLFswLDEsIllfe1xcdGV4dHtyZWR9fSJdLFswLDFdLFsxLDIsImYiXSxbMCwzXSxbMywyXV0=
\[\begin{tikzcd}
	{X_{\text{red}}} & X \\
	{Y_{\text{red}}} & Y
	\arrow[from=1-1, to=1-2]
	\arrow[from=1-1, to=2-1]
	\arrow["f", from=1-2, to=2-2]
	\arrow[from=2-1, to=2-2]
\end{tikzcd}\]
    If $g: Y \rightarrow Z$ is a second morphism of schemes, we have $(g \circ f)_{\mathrm{red}}=g_{\mathrm{red}} \circ f_{\mathrm{red}}$.


\end{prop}


















\newpage 
\section{Fibered Products}
\begin{prop}
    Let $S$ be a scheme and let $X$ and $Y$ be two $S$-schemes. Then the fiber product $X \times_S Y$ exists in the category of schemes. 
\end{prop}
\begin{prop}
    Let $f: X \rightarrow S$ and $g: Y \rightarrow S$ be morphisms of schemes with the same target. Let $(X \times_S Y, p, q)$ be the fibre product. Suppose that $U \subset S, V \subset X, W \subset Y$ are open subschemes such that $f(V) \subset U$ and $g(W) \subset U$. Then the canonical morphism $V \times_U W \rightarrow X \times_S Y$ is an open immersion 
    which identifies $V \times_U W$ with $p^{-1}(V) \cap q^{-1}(W)$.
    \label{proposition: open subscheme of fiber product}
\end{prop}
\begin{coro}
    Let $k$ be a field and let $X$ and $Y$ be $k$-schemes (locally) of finite type. Then $X \times_k Y$ is (locally) of finite type over $k$.
\end{coro}
\begin{exam}
    Let $A \leftarrow R \rightarrow B$ be homomorphisms of rings, let $S=\operatorname{Spec}(R)$, $X=\operatorname{Spec}(A)$, and $Y=\operatorname{Spec}(B)$. Set $Z=\operatorname{Spec}\left(A \otimes_R B\right)$ and let $p: Z \rightarrow X$ and $q: Z \rightarrow Y$ be the morphisms of schemes corresponding to the ring homomorphisms
    $$
    \begin{array}{ll}
    \alpha: A \rightarrow A \otimes_R B, & a \mapsto a \otimes 1 \\
    \beta: B \rightarrow A \otimes_R B, & b \mapsto 1 \otimes b
    \end{array}
    $$
    Then $(Z, p, q)$ is a fiber product of $X$ and $Y$ over $S$ in the category of schemes.
\end{exam}
% \begin{prop}
%     Let $S$ be a scheme, let $X$ and $Y$ be $S$-schemes, let $S=\bigcup_i S_i$ be an open covering and denote by $X_i\left(\right.$ resp. $\left.Y_i\right)$ the inverse image of $S_i$ in $X$ (resp. in $Y$ ). For all $i$ let $X_i=\bigcup_{j \in J_i} X_{i j}$ and $Y_i=\bigcup_{k \in K_i} Y_{i k}$ be open coverings. Then
%     $$
%     X \times_S Y=\bigcup_i \bigcup_{j \in J_i, k \in K_i} X_{i j} \times_{S_i} Y_{i k}
%     $$
%     is an open covering of $X \times_S Y$ if we view all 
%     $ X_{i j} \times_{S_i} Y_{i k}$ as open subschemes of $ X \times_S Y$.
% \end{prop}
% \begin{exam}
%     Let $R$ be a ring, and $\mathbb{A}_R^n=\operatorname{Spec}\left(R\left[T_1, \ldots, T_n\right]\right)$ be the affine space over $R$. For integers $n, m \geq 0$ one has $R\left[T_1, \ldots, T_n\right] \otimes_R R\left[T_{n+1}, \ldots, T_{n+m}\right] \cong R\left[T_1, \ldots, T_{n+m}\right]$ and 
%     therefore the description of fiber products for affine schemes that
%     $$
%     \mathbb{A}_R^n \times_R \mathbb{A}_R^m \cong \mathbb{A}_R^{n+m}
%     $$    
% \end{exam}
\begin{defn}[Relative Frobenius]
    Let $p$ be a prime number and let $S$ be a scheme over $\mathbb{F}_p$.
    We denote by $\mathrm{Frob}_S: S \rightarrow S$ the absolute Frobenius of $S: \mathrm{Frob}_S$ is the identity on the underlying topological spaces and $\mathrm{Frob}_S^b$ is the map $x \mapsto x^p$ on $\Gamma\left(U, \mathscr{O}_S\right)$ for all open subsets $U$ of $S$. 
    
    Now let $f: X \rightarrow S$ be an $S$-scheme. Note that $\operatorname{Frob}_X$ is in general not an $S$-morphism. Instead of the absolute Frobenius we therefore introduce a relative variant. 
% https://q.uiver.app/#q=WzAsNSxbMiwxLCJYIl0sWzIsMiwiUyJdLFsxLDIsIlMiXSxbMSwxLCJYXnsocCl9Il0sWzAsMCwiWCJdLFswLDEsImYiXSxbMiwxLCJcXHRleHR7RnJvYn1fUyIsMl0sWzMsMF0sWzMsMl0sWzQsMCwiXFx0ZXh0e0Zyb2J9X1giXSxbNCwyLCJmIiwyXSxbNCwzLCJGX3tYL1N9IiwxLHsiY29sb3VyIjpbMjQwLDYwLDYwXSwic3R5bGUiOnsiYm9keSI6eyJuYW1lIjoiZGFzaGVkIn19fSxbMjQwLDYwLDYwLDFdXV0=
\[\begin{tikzcd}
	X \\
	& {X^{(p)}} & X \\
	& S & S
	\arrow["{F_{X/S}}"{description}, color={rgb,255:red,92;green,92;blue,214}, dashed, from=1-1, to=2-2]
	\arrow["{\text{Frob}_X}", from=1-1, to=2-3]
	\arrow["f"', from=1-1, to=3-2]
	\arrow[from=2-2, to=2-3]
	\arrow[from=2-2, to=3-2]
	\arrow["f", from=2-3, to=3-3]
	\arrow["{\text{Frob}_S}"', from=3-2, to=3-3]
\end{tikzcd}\]
Let $X^{(p)}$ be the fiber product of $S\rarr{\text{Frob}_S}S$ and $X\rightarrow S$, then $F_{X/S}$ is called relative Frobenius of $X$ over $S$. 

\end{defn}
\begin{exam}
    Let $\bb{F}_q=\bb{F}_{p^n}$ be a finite field over $\bb{F}_p$. 
    If $f=\sum a_{\alpha}x^{\alpha}\in\bb{F}_q[x_1,\dots,x_n]$, 
    define 
    $f^{(p)}=\sum a_{\alpha}^p x^{\alpha}$. Assume $X$ is a scheme, 
    consider the following commutative diagram 
    % https://q.uiver.app/#q=WzAsNSxbMiwyLCJcXG1hdGhiYntGfV9xIl0sWzEsMiwiXFxtYXRoYmJ7Rn1fcSJdLFsyLDEsIlxcbWF0aGJie0Z9X3FbeF8xLFxcZG90cyx4X25dLyhmX2opIl0sWzEsMSwiXFxtYXRoYmJ7Rn1fcVt4XzEsXFxkb3RzLHhfbl0vKGZfal57KHApfSkiXSxbMCwwLCJcXEdhbW1hKFgsXFxtYXRoc2Nye099X1gpIl0sWzAsMSwieF5wXFxsZWZ0YXJyb3cgeCJdLFswLDJdLFsxLDMsIlxcdGV4dHtpZH0iXSxbMiwzLCJmXnsocCl9IFxcbGVmdGFycm93IGYiXSxbMyw0LCJoIiwxLHsiY29sb3VyIjpbMjQwLDYwLDYwXSwic3R5bGUiOnsiYm9keSI6eyJuYW1lIjoiZGFzaGVkIn19fSxbMjQwLDYwLDYwLDFdXSxbMiw0LCJmIiwyXSxbMSw0LCJnIiwwLHsiY3VydmUiOi0yfV1d
\[\begin{tikzcd}
	{\Gamma(X,\mathscr{O}_X)} \\
	& {\mathbb{F}_q[x_1,\dots,x_n]/(f_j^{(p)})} & {\mathbb{F}_q[x_1,\dots,x_n]/(f_j)} \\
	& {\mathbb{F}_q} & {\mathbb{F}_q}
	\arrow["h"{description}, color={rgb,255:red,92;green,92;blue,214}, dashed, from=2-2, to=1-1]
	\arrow["f"', from=2-3, to=1-1]
	\arrow["{f^{(p)} \leftarrow f}", from=2-3, to=2-2]
	\arrow["g", curve={height=-12pt}, from=3-2, to=1-1]
	\arrow["{\text{id}}", from=3-2, to=2-2]
	\arrow[from=3-3, to=2-3]
	\arrow["{x^p\leftarrow x}", from=3-3, to=3-2]
\end{tikzcd}\]
where $h$ is defined by $h(\alpha_i x_i)=g(\alpha_i)f(x_i)$. This shows that 
$\spec{\mathbb{F}_q[x_1,\dots,x_n]/(f_j^{(p)})}$ is the fiber product of 
$\spec{\bb{F}_q}\rarr{\text{Frob}}\spec{\bb{F}_q}$ and $\spec{\bb{F}_q[x_1,\dots,x_n]/(f_j)}\rightarrow \spec{\bb{F}_q}$. 

In particular, if $X=\spec{\bb{F}_q[x_1,\dots,x_n]/(f_j)}$, $f=\text{Frob}_X$ and $g=\text{id}$, 
then $h$ is a $\bb{F}_p$-algebra homomorphism such that $h(x_i)=x_i^p$. 
% https://q.uiver.app/#q=WzAsNSxbMiwyLCJcXG1hdGhiYntGfV9xIl0sWzEsMiwiXFxtYXRoYmJ7Rn1fcSJdLFsyLDEsIlxcbWF0aGJie0Z9X3FbeF8xLFxcZG90cyx4X25dLyhmX2opIl0sWzEsMSwiXFxtYXRoYmJ7Rn1fcVt4XzEsXFxkb3RzLHhfbl0vKGZfal57KHApfSkiXSxbMCwwLCJcXG1hdGhiYntGfV9xW3hfMSxcXGRvdHMseF9uXS8oZl9qKSJdLFswLDEsInhecFxcbGVmdGFycm93IHgiXSxbMCwyLCJcXHRleHR7aWR9IiwyXSxbMSwzLCJcXHRleHR7aWR9Il0sWzIsMywiZl57KHApfSBcXGxlZnRhcnJvdyBmIl0sWzMsNCwieF9pXFxyaWdodGFycm93IHhfaV5wIiwwLHsiY29sb3VyIjpbMjQwLDYwLDYwXSwic3R5bGUiOnsiYm9keSI6eyJuYW1lIjoiZGFzaGVkIn19fSxbMjQwLDYwLDYwLDFdXSxbMiw0LCJmXFxyaWdodGFycm93IGZecCIsMl0sWzEsNCwiXFx0ZXh0e2lkfSIsMCx7ImN1cnZlIjotMn1dXQ==
\[\begin{tikzcd}
	{\mathbb{F}_q[x_1,\dots,x_n]/(f_j)} \\
	& {\mathbb{F}_q[x_1,\dots,x_n]/(f_j^{(p)})} & {\mathbb{F}_q[x_1,\dots,x_n]/(f_j)} \\
	& {\mathbb{F}_q} & {\mathbb{F}_q}
	\arrow["{x_i\rightarrow x_i^p}", color={rgb,255:red,92;green,92;blue,214}, dashed, from=2-2, to=1-1]
	\arrow["{f\rightarrow f^p}"', from=2-3, to=1-1]
	\arrow["{f^{(p)} \leftarrow f}", from=2-3, to=2-2]
	\arrow["{\text{id}}", curve={height=-12pt}, from=3-2, to=1-1]
	\arrow["{\text{id}}", from=3-2, to=2-2]
	\arrow["{\text{id}}"', from=3-3, to=2-3]
	\arrow["{x^p\leftarrow x}", from=3-3, to=3-2]
\end{tikzcd}\]
That is, although $f\rightarrow f^p$ can factor through two $\bb{F}_p$-algebra homomorphisms.


\end{exam}
\begin{exam}
    Since fiber product exists in category of scheme, 
    consider a morphism $\spec{\bb{C}}\rightarrow \spec{\bb{R}}$ 
    and a $\bb{R}$-scheme $Y$, 
    we have 
    \begin{equation*}
        \text{Hom}_{\spec{\bb{R}}}(\spec{\bb{C}},Y)=\text{Hom}_{\spec{\bb{C}}}(\spec{\bb{C}},Y\times_\bb{R}\spec{\bb{C}})
    \end{equation*}
\end{exam}
\begin{exam}
    If field $K$ is a extension of $k$, 
    consider a morphism $\spec{\bb{K}}\rightarrow \spec{\bb{k}}$ and $k$-schemes $X,Y$, we have 
    \begin{equation*}
        \text{Hom}_{k}(\spec{K},X\times_k Y)=
        \text{Hom}_{k}(\spec{K},X)\times \text{Hom}_{k}(\spec{K},Y)
    \end{equation*}
\end{exam}





\begin{prop}
    Let $S$ be a scheme, $X$ and $Y$ two $S$-schemes, and let 
    $f: X^{\prime} \rightarrow X$ be a morphism of $S$-schemes. Let $g$ be the morphism 
    induced by universal property of fiber product 
% https://q.uiver.app/#q=WzAsNyxbMSwxLCJaPVhcXHRpbWVzX1MgWSJdLFsyLDEsIlkiXSxbMSwyLCJYIl0sWzIsMiwiUyJdLFswLDEsIlheXFxwcmltZSJdLFsxLDAsIlkiXSxbMCwwLCJaXlxccHJpbWUgPVheXFxwcmltZVxcdGltZXNfUyBZIl0sWzAsMSwicSJdLFswLDIsInAiXSxbMiwzXSxbMSwzXSxbNCwyLCJmIl0sWzUsMV0sWzYsNF0sWzYsNV0sWzYsMCwiZyIsMCx7ImNvbG91ciI6WzI0MCw2MCw2MF0sInN0eWxlIjp7ImJvZHkiOnsibmFtZSI6ImRhc2hlZCJ9fX0sWzI0MCw2MCw2MCwxXV1d
\[\begin{tikzcd}
	{Z^\prime =X^\prime\times_S Y} & Y \\
	{X^\prime} & {Z=X\times_S Y} & Y \\
	& X & S
	\arrow[from=1-1, to=1-2]
	\arrow[from=1-1, to=2-1]
	\arrow["g", color={rgb,255:red,92;green,92;blue,214}, dashed, from=1-1, to=2-2]
	\arrow[from=1-2, to=2-3]
	\arrow["f", from=2-1, to=3-2]
	\arrow["q", from=2-2, to=2-3]
	\arrow["p", from=2-2, to=3-2]
	\arrow[from=2-3, to=3-3]
	\arrow[from=3-2, to=3-3]
\end{tikzcd}\]
Then all squares in the following diagram are cartesian 
% https://q.uiver.app/#q=WzAsNixbMSwwLCJaPVhcXHRpbWVzX1MgWSJdLFsyLDAsIlkiXSxbMSwxLCJYIl0sWzIsMSwiUyJdLFswLDEsIlheXFxwcmltZSJdLFswLDAsIlpeXFxwcmltZSA9WF5cXHByaW1lXFx0aW1lc19TIFkiXSxbMCwxLCJxIl0sWzAsMiwicCJdLFsyLDNdLFsxLDNdLFs0LDIsImYiXSxbNSw0LCJwXlxccHJpbWUiXSxbNSwwLCJnIiwwLHsiY29sb3VyIjpbMjQwLDYwLDYwXSwic3R5bGUiOnsiYm9keSI6eyJuYW1lIjoiZGFzaGVkIn19fSxbMjQwLDYwLDYwLDFdXV0=
\[\begin{tikzcd}
	{Z^\prime =X^\prime\times_S Y} & {Z=X\times_S Y} & Y \\
	{X^\prime} & X & S
	\arrow["g", color={rgb,255:red,92;green,92;blue,214}, dashed, from=1-1, to=1-2]
	\arrow["{p^\prime}", from=1-1, to=2-1]
	\arrow["q", from=1-2, to=1-3]
	\arrow["p", from=1-2, to=2-2]
	\arrow[from=1-3, to=2-3]
	\arrow["f", from=2-1, to=2-2]
	\arrow[from=2-2, to=2-3]
\end{tikzcd}\]
In addition, assume that $f: X^{\prime} \rightarrow X$ can be written as the composition of scheme morphisms which satisfy the following condition: each morphism 
is a homeomorphism onto its image and also satisfies one of the assumptions (1), (2):
\begin{enu}
\item For each point $x^{\prime} \in X^{\prime}$, the homomorphism $f_{x^{\prime}}^{\sharp}: \mathscr{O}_{X, f\left(x^{\prime}\right)} \rightarrow \mathscr{O}_{X^{\prime}, x^{\prime}}$
is surjective, and there exists an open affine neighborhood $V$ of $f\left(x^{\prime}\right)$ such that $f^{-1}(V)$ is quasi-compact.

\item For each point $x^{\prime} \in X^{\prime}$, the homomorphism $f_{x^{\prime}}^{\sharp}: \mathscr{O}_{X, f\left(x^{\prime}\right)} \rightarrow \mathscr{O}_{X^{\prime}, x^{\prime}}$ is bijective.
\end{enu}
Then, the morphism $g$ is a homeomorphism of $Z^{\prime}$ onto
$$
   g\left(Z^{\prime}\right)=p^{-1}\left(f\left(X^{\prime}\right)\right)
$$
Besides, for all $z\p\in Z\p$, 
consider following diagram 
% https://q.uiver.app/#q=WzAsNCxbMCwxLCJcXG1hdGhzY3J7T31fe1heXFxwcmltZSxwXlxccHJpbWUoel5cXHByaW1lKX0iXSxbMSwxLCJcXG1hdGhzY3J7T31fe1gscChnKHpeXFxwcmltZSkpfSJdLFsxLDAsIlxcbWF0aHNjcntPfV97WixnKHpeXFxwcmltZSl9Il0sWzAsMCwiXFxtYXRoc2Nye099X3taXlxccHJpbWUsel5cXHByaW1lfSJdLFsxLDAsImZfe3Bee1xccHJpbWV9XFxsZWZ0KHpee1xccHJpbWV9XFxyaWdodCl9XntcXHNoYXJwfSJdLFsyLDMsImdfe3pee1xccHJpbWV9fV57XFxzaGFycH0iXSxbMSwyLCJwX3tnXFxsZWZ0KHpee1xccHJpbWV9XFxyaWdodCl9XntcXHNoYXJwfSIsMl0sWzAsM11d
\[\begin{tikzcd}
	{\mathscr{O}_{Z^\prime,z^\prime}} & {\mathscr{O}_{Z,g(z^\prime)}} \\
	{\mathscr{O}_{X^\prime,p^\prime(z^\prime)}} & {\mathscr{O}_{X,p(g(z^\prime))}}
	\arrow["{g_{z^{\prime}}^{\sharp}}", from=1-2, to=1-1]
	\arrow[from=2-1, to=1-1]
	\arrow["{p_{g\left(z^{\prime}\right)}^{\sharp}}"', from=2-2, to=1-2]
	\arrow["{f_{p^{\prime}\left(z^{\prime}\right)}^{\sharp}}", from=2-2, to=2-1]
\end{tikzcd}\]
induced by the “left square” of above diagram.
We have the homomorphism $g_{z^{\prime}}^{\sharp}$, is surjective and its kernel is generated by the image 
of the kernel of $f_{p^{\prime}\left(z^{\prime}\right)}^{\sharp}$ under $p_{g\left(z^{\prime}\right)}^{\sharp}$.

\label{proposition: immersion stable under base change}

\end{prop}
\begin{exam}
    The following $f$ satisfying above assumption
\begin{enu}
\item $f$ is an immersion of schemes 
\item $f$ is the canonical morphism $\operatorname{Spec} \mathscr{O}_{X, x} \rightarrow X$ for some point $x \in X$.
\item $f$ is the canonical morphism Spec $\kappa(x) \rightarrow X$ for some point $x \in X$.
\end{enu}
\end{exam}
\begin{prooff}
    
\end{prooff}




\begin{defn}[fibers of morphism]
Consider the natural morphism $\text{Spec}\,\kappa(s)\rightarrow S$ and a morphism $f:X\rightarrow S$.  
We define $X_s=X\times_S\text{Spec}\,\kappa(s)$ 
be the fiber of $f:X\rightarrow S$ in $s$. 
% https://q.uiver.app/#q=WzAsNyxbMSwxLCJYIl0sWzIsMSwiWCJdLFsxLDIsIlMiXSxbMiwyLCJTIl0sWzAsMSwiXFx0ZXh0e1NwZWN9XFwsXFxrYXBwYShzKSJdLFsxLDAsIlgiXSxbMCwwLCJYXFx0aW1lc19TXFx0ZXh0e1NwZWN9XFwsXFxrYXBwYShzKSJdLFswLDEsIlxcdGV4dHtpZH0iXSxbMCwyLCJmIl0sWzIsM10sWzEsM10sWzQsMiwiMFxcbWFwc3RvIHMiLDJdLFs1LDEsIlxcdGV4dHtpZH0iXSxbNiw0XSxbNiw1XSxbNiwwLCJnIiwwLHsiY29sb3VyIjpbMjQwLDYwLDYwXSwic3R5bGUiOnsiYm9keSI6eyJuYW1lIjoiZGFzaGVkIn19fSxbMjQwLDYwLDYwLDFdXV0=
\[\begin{tikzcd}
	{X\times_S\text{Spec}\,\kappa(s)} & X \\
	{\text{Spec}\,\kappa(s)} & X & X \\
	& S & S
	\arrow[from=1-1, to=1-2]
	\arrow[from=1-1, to=2-1]
	\arrow["g", color={rgb,255:red,92;green,92;blue,214}, dashed, from=1-1, to=2-2]
	\arrow["{\text{id}}", from=1-2, to=2-3]
	\arrow["{0\mapsto s}"', from=2-1, to=3-2]
	\arrow["{\text{id}}", from=2-2, to=2-3]
	\arrow["f", from=2-2, to=3-2]
	\arrow[from=2-3, to=3-3]
	\arrow[from=3-2, to=3-3]
\end{tikzcd}\]
By Proposition\,\ref{proposition: immersion stable under base change}, the underlying topological space of $X_s$ is $f^{-1}(s)$.
\label{definition: fiber of morphism}
\end{defn}
\begin{exam}
    Consider a integral $k$-scheme of finite type be 
    $$
    X=\operatorname{Spec} k[U, T, S] /(U T-S)
    $$
    Let $f: X \rightarrow \mathbb{A}_k^1=\text{Spec}k[S]$ be the natural morphism, then $\text{Spec}A_s$ be the fiber of $f$ in $(S-s)$ where
    $$
    A_s=k[U, T, S] /(U T-S) \otimes_{k[S]} k[S] /(S-s)=k[U, T,S] /(U T-S,S-s)=k[U, T] /(UT-s)
    $$

    Hence, $X_s(k)=\bbrace{(x,y)\in k^2:xy=s}$
\end{exam}

\begin{defn}[inverse image of $Z$ under $f$]
    Let $f: X \rightarrow Y$ be a morphism of schemes and let $i: Z \rightarrow Y$ be an immersion. 
    Proposition\,\ref{proposition: immersion stable under base change} shows that the base change 
    $i_{(X)}: Z \times_Y X \rightarrow X$ is surjective on stalks 
    and a homeomorphism of $Z \times_Y X$ onto the locally closed subspace $f^{-1}(Z)$. 
    % https://q.uiver.app/#q=WzAsNyxbMSwxLCJYIl0sWzIsMSwiWCJdLFsxLDIsIlkiXSxbMiwyLCJZIl0sWzAsMSwiWiJdLFsxLDAsIlgiXSxbMCwwLCJaXFx0aW1lc19TIFkiXSxbMCwxXSxbMCwyLCJmIiwyXSxbMSwzLCJmIl0sWzQsMiwiaSIsMl0sWzIsM10sWzUsMSwiXFx0ZXh0e2lkfSJdLFs2LDAsImlfeyhYKX0iLDAseyJjb2xvdXIiOlsyNDAsNjAsNjBdLCJzdHlsZSI6eyJib2R5Ijp7Im5hbWUiOiJkYXNoZWQifX19LFsyNDAsNjAsNjAsMV1dLFs2LDVdLFs2LDRdXQ==
\[\begin{tikzcd}
	{Z\times_S Y} & X \\
	Z & X & X \\
	& Y & Y
	\arrow[from=1-1, to=1-2]
	\arrow[from=1-1, to=2-1]
	\arrow["{i_{(X)}}", color={rgb,255:red,92;green,92;blue,214}, dashed, from=1-1, to=2-2]
	\arrow["{\text{id}}", from=1-2, to=2-3]
	\arrow["i"', from=2-1, to=3-2]
	\arrow[from=2-2, to=2-3]
	\arrow["f"', from=2-2, to=3-2]
	\arrow["f", from=2-3, to=3-3]
	\arrow[from=3-2, to=3-3]
\end{tikzcd}\]
    Therefore $i_{(X)}$ is an immersion.
\end{defn}
\begin{prop}
    In above definition, if $Z$ is closed subscheme of $Y$, then $f^{-1}(Z)$ is closed which implies 
    $i_{(X)}$ is a closed immersion. By the second result of Proposition\,\ref{proposition: fiber of morphism}, 
    if $i$ is open immersion, $i_{(X)}$ is also open immersion.
\end{prop}

\begin{defn}[intersection of subscheme]
    As a special case of the inverse image of a subscheme 
    we can define the intersection of two subschemes: Let $i: Y \rightarrow X$ and 
    $j: Z \rightarrow X$ be two subschemes.
    % https://q.uiver.app/#q=WzAsNyxbMSwxLCJaIl0sWzIsMSwiWiJdLFsxLDIsIlgiXSxbMiwyLCJYIl0sWzAsMSwiWSJdLFsxLDAsIloiXSxbMCwwLCJaXFx0aW1lc19TIFkiXSxbMCwxXSxbMCwyLCJqIiwyXSxbMSwzLCJmIl0sWzQsMiwiaSIsMl0sWzIsM10sWzUsMSwiXFx0ZXh0e2lkfSJdLFs2LDAsImlfeyhaKX09cCIsMCx7ImNvbG91ciI6WzI0MCw2MCw2MF0sInN0eWxlIjp7ImJvZHkiOnsibmFtZSI6ImRhc2hlZCJ9fX0sWzI0MCw2MCw2MCwxXV0sWzYsNSwicCJdLFs2LDQsInEiLDJdXQ==
\[\begin{tikzcd}
	{Z\times_X Y} & Z \\
	Y & Z & Z \\
	& X & X
	\arrow["p", from=1-1, to=1-2]
	\arrow["q"', from=1-1, to=2-1]
	\arrow["{i_{(Z)}=p}", color={rgb,255:red,92;green,92;blue,214}, dashed, from=1-1, to=2-2]
	\arrow["{\text{id}}", from=1-2, to=2-3]
	\arrow["i"', from=2-1, to=3-2]
	\arrow[from=2-2, to=2-3]
	\arrow["j"', from=2-2, to=3-2]
	\arrow["f", from=2-3, to=3-3]
	\arrow[from=3-2, to=3-3]
\end{tikzcd}\]
Then the map $j\circ p$ is an immersion onto the locally closed subset $Y\cap X$. 
    
\end{defn}
\begin{defn}
    For an arbitrary scheme $S$, define 
    $\bb{A}_S^n=\bb{A}_\bb{Z}^n\times_\bb{Z} S$, $\bb{P}_S^n=\bb{P}_\bb{Z}^n\times_\bb{Z} S$. 
\end{defn}
\begin{exam}
    Suppose $I \subset A\left[x_1, \ldots, x_m\right]$ and $J \subset A\left[y_1, \ldots, y_n\right]$ are ideals.
    $$
    A\left[x_1, \ldots, x_m\right] / I \otimes_A A\left[y_1, \ldots, y_n\right] / J \simeq A\left[x_1, \ldots, x_m, y_1, \ldots, y_n\right] /(I, J) .
    $$
    In particular, $\bb{A}^n_k\times_k \bb{A}^m_k=\bb{A}^{m+n}_k$
\end{exam}
\begin{prooff}
    The bi-linear map 
    \begin{align*}
        A\left[x_1, \ldots, x_m\right] / I \times A\left[y_1, \ldots, y_n\right] / J&\rightarrow A\left[x_1, \ldots, x_m, y_1, \ldots, y_n\right] /(I, J) \\ 
        (f+I,g+J) & \rightarrow fg+(I,J)
    \end{align*}
    induces an isomorphism 
    $$
    A\left[x_1, \ldots, x_m\right] / I \otimes_A A\left[y_1, \ldots, y_n\right] / J \simeq A\left[x_1, \ldots, x_m, y_1, \ldots, y_n\right] /(I, J) .
    $$
   
\end{prooff}
\begin{exam}
    $\mathbb{C} \otimes_{\mathbb{R}} \mathbb{C}\simeq \bb{C}\times \bb{C}$ as $\bb{R}$-algebra.
\end{exam}
\begin{prooff}
    $$
        \begin{aligned}
        \mathbb{C} \otimes_{\mathbb{R}} \mathbb{C} & \cong \mathbb{C} \otimes_{\mathbb{R}}\left(\mathbb{R}[x] /\left(x^2+1\right)\right) \\
        & \cong\left(\mathbb{C} \otimes_{\mathbb{R}} \mathbb{R}[x]\right) /\left(x^2+1\right) \quad \text {Since $\otimes_\bb{R}\bb{C}$ is an exact functor} \\
        & \cong \mathbb{C}[x] /\left(x^2+1\right)  \\
        & \cong \mathbb{C}[x] /((x-i)(x+i)) \\
        & \cong \mathbb{C}[x] /(x-i) \times \mathbb{C}[x] /(x+i) \quad \text { by the Chinese Remainder Theorem } \\
        & \cong \mathbb{C} \times \mathbb{C}
        \end{aligned}
    $$
\end{prooff}















\begin{defn}
    Let (Grp) be the category of groups and $V:(\mathrm{Grp}) \rightarrow$ (Sets) the forgetful functor. Let $S$ be a scheme and let $G$ be an $S$-scheme. The following data for $G$ are equivalent by Yoneda's lemma 
\begin{enu} 
    \item A factorization of the functor $h_G:(\mathrm{Sch} / S)^{\mathrm{opp}} \rightarrow$ (Sets) through the forgetful functor $V:(\mathrm{Grp}) \rightarrow$ (Sets).
    \item For all $S$-schemes $T$ the structure of a group on $G_S(T)$ which is functorial in $T$ (i.e., for all $S$-morphisms $T^{\prime} \rightarrow T$ the associated map $G_S(T) \rightarrow G_S\left(T^{\prime}\right)$ is a homomorphism of groups).
\end{enu}
\end{defn}
\begin{defn}
    A homomorphism of $S$-group schemes $G$ and $H$ is a morphism $G \rightarrow H$ of $S$-schemes such that for all $S$-schemes $T$ the induced map $G(T) \rightarrow H(T)$ is a group homomorphism.
\end{defn}
\begin{exam}
    $S=\operatorname{Spec} \mathbb{Z}$ and $G:=\mathrm{GL}_n$ with $\mathrm{GL}_n(T):=\mathrm{GL}_n\left(\Gamma\left(T, \mathscr{O}_T\right)\right)$, the group of invertible $(n \times n)$-matrices over $\Gamma\left(T, \mathscr{O}_T\right)$, for any scheme $T$ and for a fixed integer $n \geq 1$. 
    The underlying scheme of $\mathrm{GL}_n$ is Spec $A$ with $A=\mathbb{Z}\left[\left(T_{i j}\right)_{1 \leq i, j \leq n}\right]\left[\operatorname{det}^{-1}\right]$, where $\operatorname{det}:=\sum_{\sigma \in S_n} \operatorname{sgn}(\sigma) T_{1 \sigma(1)} \cdots T_{n \sigma(n)}$ is the determinant of the matrix $\left(T_{i j}\right)_{i, j}$. This group scheme is called the general linear group scheme. We call $\mathbb{G}_m:=\mathrm{GL}_1$ the multiplicative group scheme.
\end{exam}
\begin{exam}
    The additive group scheme $\mathbb{G}_{a, S}$ over $S$ is defined by $\mathbb{G}_{a, S}(T)=\Gamma\left(T, \mathscr{O}_T\right)$ for every $S$-scheme $T$. Its underlying $S$-scheme is $\mathbb{A}_S^1$. 
\end{exam}
% \begin{prooff}
    
% \end{prooff}
\newpage 
\section{Dimension of Scheme over $k$}
Even for noetherian schemes the notion of dimension is sometimes 
\blue{quite counter-intuitive}. 
If one restricts oneself to the case of schemes of finite type over a field, 
then the theory of dimension \blue{works mostly as expected}, and is a very useful invariant.
\begin{prop}
    Let $X$ be a topological space.
    \begin{enu}
        \item Let $Y$ be a subspace of $X$. Then $\operatorname{dim} Y \leq \operatorname{dim} X$. If $X$ is irreducible, $\operatorname{dim} X<\infty$, and $Y \subsetneq X$ is a proper closed subset, then $\operatorname{dim} Y<\operatorname{dim} X$.
        \item Let $X=\bigcup_\alpha U_\alpha$ be an open covering. Then
        $$
        \operatorname{dim} X=\sup _\alpha \operatorname{dim} U_\alpha .
        $$
        \item Let $I$ be the set of irreducible components of $X$. Then
        $$
        \operatorname{dim} X=\sup _{Y \in I} \operatorname{dim} Y .
        $$
        \item  Let $X$ be a scheme. Then
        $$
        \operatorname{dim} X=\sup _{x \in X} \operatorname{dim} \mathscr{O}_{X, x}
        $$
    \end{enu}
    \label{proposition: topological space, dimension}
\end{prop}
\begin{exam}
    $\dim \mathbb{A}_k^n=\operatorname{dim} \mathbb{P}_k^n=n$
\end{exam}
\begin{prop}
    Let $i: Y \rightarrow X$ be a closed immersion of schemes, where $X$ is integral. If $\operatorname{dim} X=\operatorname{dim} Y<\infty$, then $i$ is an isomorphism.
\end{prop}
\begin{prooff}
    By Proposition\,\ref{proposition: topological space, dimension}, 
    $i$ is a homeomorphism. 
    Hence, we need to show the map on stalks 
    $\mathscr{O}_{X,i(x)}\rightarrow \mathscr{O}_{Y,x}$ is injective. 
    Since $X$ is integral, take $i(\eta)$ 
    be the generic point of $X$. 
    It's easy to check the follow diagram commute 
    % https://q.uiver.app/#q=WzAsNCxbMCwwLCJcXG1hdGhzY3J7T31fe1gsaSh4KX0iXSxbMSwwLCJcXG1hdGhzY3J7T31fe1kseH0iXSxbMCwxLCJcXG1hdGhzY3J7T31fe1gsaShcXGV0YSl9Il0sWzEsMSwiXFxtYXRoc2Nye099X3tZLFxcZXRhfSJdLFswLDFdLFswLDIsIiIsMix7InN0eWxlIjp7InRhaWwiOnsibmFtZSI6Imhvb2siLCJzaWRlIjoidG9wIn0sImhlYWQiOnsibmFtZSI6ImhhcnBvb24iLCJzaWRlIjoiYm90dG9tIn19fV0sWzEsM10sWzIsMywiIiwyLHsic3R5bGUiOnsidGFpbCI6eyJuYW1lIjoiaG9vayIsInNpZGUiOiJ0b3AifSwiaGVhZCI6eyJuYW1lIjoiaGFycG9vbiIsInNpZGUiOiJ0b3AifX19XV0=
\[\begin{tikzcd}
	{\mathscr{O}_{X,i(x)}} & {\mathscr{O}_{Y,x}} \\
	{\mathscr{O}_{X,i(\eta)}} & {\mathscr{O}_{Y,\eta}}
	\arrow[from=1-1, to=1-2]
	\arrow[hook, harpoon', from=1-1, to=2-1]
	\arrow[from=1-2, to=2-2]
	\arrow[hook, harpoon, from=2-1, to=2-2]
\end{tikzcd}\]
   Since $\mathscr{O}_{X,i(\eta)}$ is a field, $\mathscr{O}_{X,i(\eta)}\rightarrow \mathscr{O}_{Y,\eta}$ is injective. Hence, 
   $\mathscr{O}_{X,i(x)}\rightarrow \mathscr{O}_{Y,x}$ is injective. 
\end{prooff}
\begin{prop}
    Let $X$ be an $k$-scheme locally of finite type with closed subset 
    $Y=\overline{\bbrace{\theta}}$. Then $\operatorname{dim} Y=\operatorname{trdeg}_k \kappa(\theta)$ 
    \label{proposition: transcendental dimension, closed subset, generic point}

\end{prop}
\begin{prooff}
     By Definition\,\ref{definition: reduced subscheme}, 
     Algebra\,\ref{corollary: dimension= transcendental dimension} and 
     Algebra\,\ref{proposition: finite generated integral algebra, same length}. 
\end{prooff}



\begin{prop}
    Let $X$ be an irreducible $k$-scheme locally of finite type with generic point $\eta$.
\begin{enu}
    \item $\operatorname{dim} X=\operatorname{trdeg}_k \kappa(\eta)$.
    \item $\operatorname{dim} U=\operatorname{dim} X$ for any non-empty open subscheme $U$ of $X$. 
    \item Let $x \in X$ be any closed point. Then $\operatorname{dim} \mathscr{O}_{X, x}=\operatorname{dim} X$.
    \item Let $f: Y \rightarrow X$ be a morphism of $k$-schemes of locally 
    finite type such that $f(Y)$ contains the generic point $\eta$ of $X$. 
    Then $\operatorname{dim} Y \geq \operatorname{dim} X$. 
    
\end{enu}
If $X$ is integral, then $\kappa(\eta)$ is simply the function field of $X$.
\label{proposition: dimension, locally finite type}
\end{prop}
\begin{prooff}
    % (1): By Definition\,\ref{definition: reduced subscheme} and Proposition\,\ref{proposition: topological space, dimension} (2) and (4), 
    % we may assume that $X=\operatorname{Spec} A$, where $A$ is an integral finitely generated $k$-algebra. 
    % Then it follows from Algebra\,\ref{corollary: dimension= transcendental dimension}. 
   
    (2): Notice that $U$ is also an irreducible $k$-scheme locally of finite type, the information on stalks are the same.
    So we have $\dim U=\operatorname{trdeg}_k \kappa(\eta)=\dim X$. 

    (3):For all closed $x\in X$, we may find open subset $U$ such that $U=\spec{A}$ where $A$ is an  
    finitely generated $k$-algebra. Since $U$ is irreducible, $\text{nil}(A)$ is a prime ideal.
    Since $x$ is closed, for some maximal ideal $\mathfrak{m}$, 
    $$
       \dim \mathscr{O}_{X,x}=\dim A_{\mathfrak{m}}=\dim (A/\text{nil}(A))_\mathfrak{m} =\dim A/\text{nil}(A) =\dim U=\dim X
    $$
    where the third "$=$" follows from Algebra\,\ref{proposition: finite generated integral algebra, same length}.

    (4): 
By hypothesis there exists $\theta \in Y$ such that $f(\theta)=\eta$. Therefore $f$ induces a $k$-embedding $\kappa(\eta) \hookrightarrow \kappa(\theta)$. Denote by $Z$ the closure of $\theta$.
$$
\operatorname{dim} X=\operatorname{trdeg} \kappa(\eta) \leq \operatorname{trdeg} \kappa(\theta)\leq \operatorname{dim} Y .
$$
% The last inequality is because, for some $\theta\in U$ open in $X$, $U=\spec{A}$ where $A$ be a finitely generated 
% $k$-algebra. Denote the corresponding prime ideal of $x$ by $\mathfrak{p}$, we have 
% \begin{equation*}
%     \operatorname{trdeg} \kappa(\theta)=\operatorname{trdeg} \text{Frac}(A/\mathfrak{p})=\dim A/\mathfrak{p}\le \dim U\le \dim X
% \end{equation*}
\begin{prop}
    Let $X$ be a non-empty $k$-scheme of finite type. The following are equivalent:
    \begin{enu} 
    \item  $\operatorname{dim} X=0$.
    \item  The scheme $X$ is affine, the $k$-vector space $\Gamma\left(X, \mathscr{O}_X\right)$ is finite-dimensional, and $\Gamma\left(X, \mathscr{O}_X\right)=\prod_x \mathscr{O}_{X, x}$.
    \item  The underlying topological space of $X$ is discrete.
    \item  The underlying topological space of $X$ has only finitely many points.
    \end{enu}
    \label{proposition: finite type, dim 0}
\end{prop}



\end{prooff}
\begin{prop}
    Let $f: Y \rightarrow X$ be a morphism of $k$-schemes of locally finite type with finite fibers. Then $\operatorname{dim} Y \leq \operatorname{dim} X$.
\end{prop}
\begin{prooff}
    Let $Z$ be an irreducible component of $Y$ with generic point 
$\theta$ and set $x:=f(\theta)$. 
By Proposition\,\ref{proposition: transcendental dimension, closed subset, generic point} and 
Proposition\,\ref{proposition: topological space, dimension} (2), 
we only need to show 
$\operatorname{trdeg}_k \kappa(\theta) \leq \operatorname{dim} X$. 

Replacing $X$ by an open affine neighborhood $U$ of $x$ and $Y$ 
by an open affine neighborhood of $\theta$ in $f^{-1}(U)$ we may assume that $X=\operatorname{Spec} A$ and $Y=\operatorname{Spec} B$ are affine. 
Then $B$ is a $k$-algebra of finite type and in particular an $A$-algebra of finite type. The fiber $f^{-1}(x)=\operatorname{Spec}\left(B \otimes_A \kappa(x)\right)$ is thus a $\kappa(x)$-scheme of finite type with only finitely many points. 

Notice that the induced morphism on residue field of a immersion is an isomorphism, 
the residue field of  $\spec{B\otimes_A \kappa(x)}$ at 
$\theta$ is the same as the residue field of 
$\spec{B}$ at $\theta$. 

Since the point $\theta$ is closed in $f^{-1}(x)$ by Proposition\,\ref{proposition: finite type, dim 0}  
and therefore $\kappa(\theta)$ is a finite extension of $\kappa(x)$. This shows $\operatorname{trdeg}_k \kappa(\theta)=\operatorname{trdeg}_k \kappa(x)=\operatorname{dim} \overline{\{x\}} \leq \operatorname{dim} X$.

\end{prooff}
\begin{prop}
    Let $X$ be a $k$-scheme locally of finite type and let $x \in X$ be a closed point. 
    Then $\operatorname{dim} \mathscr{O}_{X, x}=\sup _Z \operatorname{dim} Z$, where $Z$ runs through the (finitely many) irreducible components of $X$ containing $x$.
\end{prop}
\begin{prooff}
    Assume $I= \bbrace{Z_1,\dots,Z_r}=\bbrace{\overline{\bbrace{\theta_1}},\dots, \overline{\bbrace{\theta_r}}}$. 
    
    Since $X$ is locally finite type, there's some $x\in U$ open in $X$, 
    $U=\spec{A}$ where $A$ be a finitely generated $k$-algebra. 
    
    If $x$ corresponds to $\mathfrak{m}$, by Proposition\,\ref{proposition: irreducible components, locally finite}, 
    $\theta_i\in Z_i\cap U$ correspond to minimal prime ideal contained in $\mathfrak{m}$.
    Then by Proposition\,\ref{proposition: finite generated integral algebra, same length}, for some $i$, 
    $\dim A_\mathfrak{m}=\dim A/\mathfrak{p}_{\theta_i}$. 

    Then, 
    $$
    \dim \mathscr{O}_{X,x}=\dim A_\mathfrak{m}=\dim A/\mathfrak{p}_{\theta_i}=\text{trdeg}_k \kappa(\theta_i)=\dim Z_i
    $$
\end{prooff}

\begin{defn}[local dimension]
    Let $X$ be a topological space and $x \in X$. The dimension of $X$ at $x$ is
    $$
    \operatorname{dim}_x X=\inf _U \operatorname{dim} U
    $$
    
\end{defn}
\begin{coro}
    Let $X$ be a scheme locally of finite type over a field and let $I$ be the (finite) set of irreducible components of $X$ containing $x$. Then $\operatorname{dim}_x X=\sup _{Z \in I} \operatorname{dim} Z$. If $x \in X$ is a closed point, then $\operatorname{dim}_x X=\operatorname{dim} \mathscr{O}_{X, x}$.
\end{coro}
\begin{prooff}
    Assume $I=\bbrace{Z_1,\dots,Z_n}$. By Proposition\,\ref{proposition: irreducible components, locally finite}, 
    there's a open subset $x\in U$ such that $U$ only meets with fintie many irreducible components of $X$. 
    Since irreducible components is closed, we may assume $U$ only meets with some subset of $I$. 
    
    By definition of local dimension, we may in addition 
    assume $U$ is affine and $\dim U=\dim_x X$. 
    Then by Proposition\,\ref{proposition: affine subscheme is finitely-generated k algebra}, 
    $U=\spec{A}$ with $A$ finitely generated $k$-algebra. 
    By Proposition\,\ref{proposition: irreducible components, locally finite} and , 
    $$
    \operatorname{dim} U=\sup _{Z \in I} \operatorname{dim}(Z \cap U)
    $$
    Notice that for all $i$, $Z_i$ is irreducible, there's a reduced scheme structure on $Z_i$ such that 
    $Z_i$ is locally finite type by Proposition\,\ref{definition: reduced subscheme}.
    By Proposition\,\ref{proposition: dimension, locally finite type}, 
    $Z_i\cap U\neq \varnothing$ implies $\dim Z_i\cap U= \dim Z_i$.  



\end{prooff}
\begin{prop}
    Let $X$ be a topological space.
\begin{enu} 
    \item Let $Z \subseteq X$ be a closed irreducible subset. The codimension $\operatorname{codim}_X Z$ of $Z$ in $X$ is the supremum of the lengths of chains of irreducible closed subsets $Z_0 \supsetneq Z_1 \supsetneq \cdots \supsetneq Z_l$ such that $Z_l=Z$.
    \item Let $Z \subseteq X$ be a closed subset. We say that $Z$ is equi-codimensional (of codimension $r$ ), if all irreducible components of $Z$ have the same codimension in $X$ (equal to $r$ ).
\end{enu}
\end{prop}
\begin{prop}
    For an arbitrary scheme $X$ and a closed irreducible subset $Z$ with generic point $\eta$ we have
    $$
    \operatorname{codim}_X Z=\operatorname{dim} \mathscr{O}_{X, \eta}=\inf _{z \in Z} \operatorname{dim} \mathscr{O}_{X, z}
    $$
     
\end{prop}
\begin{prooff}
    By Proposition\,\ref{proposition: irreducible closed subset, spec of local ring}. 
\end{prooff}
\begin{defn}
    Let $X$ be a scheme and let $Y \subseteq X$ be an arbitrary subset. Then
    $$
    \operatorname{codim}_X(Y):=\inf _{y \in Y} \operatorname{dim} \mathscr{O}_{X, y}
    $$
    is called the codimension of $Y$ in $X$.
\end{defn}
\begin{prop}
    Let $X$ be a scheme.
If $Y$ is a closed subset of $X$, we find
$$
\operatorname{codim}_X Y=\inf _Z \operatorname{codim}_X Z,
$$
where $Z$ runs through the set of irreducible components of $Y$.
\end{prop}
\begin{prooff}
    \begin{equation*}
        \text{codim}_X Y=\inf_{Z\subset Y, Z \text{ irreducible closed }} \text{codim}_X Z=\inf_{Z\subset Y, Z \text{ irreducible components }} \text{codim}_X Z
    \end{equation*}
\end{prooff}





\begin{prop}
    Let $X$ be an irreducible scheme of finite type over a field $k$. Set $d:=\operatorname{dim} X$.
\begin{enu} 
    \item All maximal chains of closed irreducible subsets of $X$ have the same length.
    \item For all closed subsets $Y$ of $X$ we have
$$
\operatorname{dim} Y+\operatorname{codim}_X Y=\operatorname{dim} X
$$
\end{enu} 
\end{prop}
\begin{prooff}
    (1): If $Z_r \subsetneq \cdots \subsetneq Z_0$ is a maximal chain, 
    then $Z_r=\{x\}$ for some closed point $x \in X$ by Proposition\,\ref{corollary: quasi-compact, closed subset contains closed point}. 
    Hence by Proposition\,\ref{proposition: irreducible closed subset, spec of local ring}, 
    $r=\operatorname{dim} \mathscr{O}_{X, x}$. 
    And by Proposition\,\ref{proposition: dimension, locally finite type}, $d=\dim \mathscr{O}_{X,x}$ which is independent of 
    the choice of maximal chain.

    (2):  
    We first assume that $Y$ is irreducible. Then $\operatorname{dim} Y+\operatorname{codim}_X Y$ is the supremum of the lengths of maximal chains of closed irreducible subsets of $X$ having $Y$ as a member. Thus the claim follows from (1).
    General case follows from above proposition. 
\end{prooff}








\newpage 
\section{Separated Morphisms}
\begin{prop}
    Equalizer exists in category $\text{Sch}/S$.
\end{prop}
\begin{prooff}
    Conisder $f,g:X\rightarrow Y$ be two $S$-morphisms and $h:T\rightarrow X$ be $S$-morphisms such that 
    $f\circ h=g\circ h$. 
    
    By universal property of fiber product, there's unique $S$-morphism $(f,g)$, making the following diagram 
    commutes: 
    % https://q.uiver.app/#q=WzAsNSxbMCwwLCJYIl0sWzEsMSwiWVxcdGltZXNfU1kiXSxbMiwxLCJZIl0sWzIsMiwiUyJdLFsxLDIsIlkiXSxbMCwxLCIoZixnKSIsMSx7ImNvbG91ciI6WzI0MCw2MCw2MF0sInN0eWxlIjp7ImJvZHkiOnsibmFtZSI6ImRhc2hlZCJ9fX0sWzI0MCw2MCw2MCwxXV0sWzEsMiwicCJdLFsyLDNdLFs0LDNdLFsxLDQsInAiLDJdLFswLDQsImYiLDJdLFswLDIsImciXV0=
\[\begin{tikzcd}
	X \\
	& {Y\times_SY} & Y \\
	& Y & S
	\arrow["{(f,g)}"{description}, color={rgb,255:red,92;green,92;blue,214}, dashed, from=1-1, to=2-2]
	\arrow["g", from=1-1, to=2-3]
	\arrow["f"', from=1-1, to=3-2]
	\arrow["q", from=2-2, to=2-3]
	\arrow["p"', from=2-2, to=3-2]
	\arrow[from=2-3, to=3-3]
	\arrow[from=3-2, to=3-3]
\end{tikzcd}\]
Consider the following diagram
% https://q.uiver.app/#q=WzAsNixbMSwxLCJYXFx0aW1lc197WFxcdGltZXNfUyBZfVkiXSxbMiwxLCJYIl0sWzEsMiwiWSJdLFsyLDIsIllcXHRpbWVzX3tTfVkiXSxbMywxLCJZIl0sWzAsMCwiVCJdLFswLDEsIlxccGlfWCIsMl0sWzAsMiwiXFxwaV9ZIl0sWzIsMywiXFxEZWx0YV97WS9TfSJdLFsxLDMsIihmLGcpIiwyXSxbMSw0LCJmIiwwLHsib2Zmc2V0IjotMX1dLFszLDQsInEiLDEseyJvZmZzZXQiOjN9XSxbNSwxLCJoIl0sWzUsMCwiXFx0aGV0YSIsMCx7ImNvbG91ciI6WzI0MCw2MCw2MF0sInN0eWxlIjp7ImJvZHkiOnsibmFtZSI6ImRhc2hlZCJ9fX0sWzI0MCw2MCw2MCwxXV0sWzUsMiwiZlxcY2lyYyBoIiwyXSxbMSw0LCJnIiwyLHsib2Zmc2V0IjoxfV0sWzMsNCwicCIsMV1d
\[\begin{tikzcd}
	T \\
	& {X\times_{Y\times_S Y}Y} & X & Y \\
	& Y & {Y\times_{S}Y}
	\arrow["\theta", color={rgb,255:red,92;green,92;blue,214}, dashed, from=1-1, to=2-2]
	\arrow["h", from=1-1, to=2-3]
	\arrow["{f\circ h}"', from=1-1, to=3-2]
	\arrow["{\pi_X}"', from=2-2, to=2-3]
	\arrow["{\pi_Y}", from=2-2, to=3-2]
	\arrow["f", shift left, from=2-3, to=2-4]
	\arrow["g"', shift right, from=2-3, to=2-4]
	\arrow["{(f,g)}"', from=2-3, to=3-3]
	\arrow["{\Delta_{Y/S}}", from=3-2, to=3-3]
	\arrow["q"{description}, shift right=3, from=3-3, to=2-4]
	\arrow["p"{description}, from=3-3, to=2-4]
\end{tikzcd}\]
It's easy to check $f\circ \pi_X=\pi_Y=g\circ \pi_X$. Moreover, $p \circ(f,g)\circ h= p\circ \Delta_{Y/S}\circ f\circ h$ 
and $q \circ(f,g)\circ h= q\circ \Delta_{Y/S}\circ f\circ h$ implies $(f,g)\circ h=\Delta_{Y/S}\circ f\circ h$. Hence, 
there's unique $\theta$ such that above diagram commutes. 
\end{prooff}
\begin{prop}
    Let $S=\operatorname{Spec} R$ be an affine scheme, let $X=\operatorname{Spec} B \rightarrow S$ and $Y=\operatorname{Spec} A \rightarrow S$ be affine $S$-schemes and let $f: X \rightarrow Y$ be an $S$-morphism corresponding to an $R$-algebra morphism $\varphi: A \rightarrow B$. Then the diagonal morphism $\Delta_{X / S}$ and graph morphism $\Gamma_f$ correspond to the following surjective ring homomorphisms.
    $$
    \begin{aligned}
    \Delta_{B / R}: B \otimes_R B & \rightarrow B, & b \otimes b^{\prime} \mapsto b b^{\prime}, \\
    \Gamma_{\varphi}: A \otimes_R B & \rightarrow B, & a \otimes b \mapsto \varphi(a) b .
    \end{aligned}
    $$
    In particular $\Delta_{X / S}$ and $\Gamma_f$ are closed immersions.
    \label{proposition: affine diagonal morphism is closed immersion}
\end{prop}
\begin{prop}
    Let $S$ be a scheme, let $X$ and $Y$ be $S$-schemes, and let $f, g: X \rightarrow Y$ be morphisms of $S$-schemes. Then $\Delta_{X / S}, \Gamma_f$, and the canonical morphism $\operatorname{Eq}(f, g) \rightarrow X$ are immersions.
\end{prop}
\begin{prooff}
    $\Delta_{X/S}$: Assume $S$ is affine. 
    By Proposition\,\ref{proposition: open subscheme of fiber product}, Proposition\,\ref{proposition: affine diagonal morphism is closed immersion}, we may find 
    $U_i\times_S U_i, i\in I$ open in $Y\times_S Y$ such that $U_i$ are affine open subschemes of $Y$ which cover 
    the image of $Y$( notice that $U_i\times_S U_i$ may not cover $Y\times_S Y$ ). Then the diagonal morphism is locally a closed immersion, which implies 
    the image of $Y$ is locally closed. 
    
    $\Gamma_f$: the same as $\Delta_{X/S}$. 

    $\text{Eq}(f,g)$: Since immersion is stable under base change, then it follows from the proof
    of existence of equalizer in category of schemes. 
\end{prooff}
\begin{lem}
    Let $u: X \rightarrow S, v: Y \rightarrow S$ be $S$-objects, let $p: X \times_S Y \rightarrow X$ and $q: X \times_S Y \rightarrow Y$ be the projections, and $f, g: X \rightarrow Y$ two $S$-morphisms.
    $$
    \Delta_{X / S}=\Gamma_{\mathrm{id}_X}, \quad \Gamma_f=\left(\operatorname{can}: \operatorname{Eq}\left(X \times_S Y \underset{f \circ p}{\stackrel{q}{\Longrightarrow}} Y\right) \rightarrow X \times_S Y\right) .
    $$
    \label{lemma: Gamma f is a equalizer}
\end{lem}
\begin{prooff}
    % https://q.uiver.app/#q=WzAsNCxbMCwxLCJUIl0sWzEsMCwiWFxcdGltZXNfUyBZIl0sWzIsMCwiWSJdLFswLDAsIlgiXSxbMCwxLCJoIiwyXSxbMSwyLCJxIiwyLHsib2Zmc2V0IjoxfV0sWzEsMiwiZlxcY2lyYyBwIiwwLHsib2Zmc2V0IjotMX1dLFszLDEsIlxcR2FtbWFfZiJdLFswLDMsInBcXGNpcmMgaCIsMCx7ImNvbG91ciI6WzI0MCw2MCw2MF0sInN0eWxlIjp7ImJvZHkiOnsibmFtZSI6ImRhc2hlZCJ9fX0sWzI0MCw2MCw2MCwxXV1d
\[\begin{tikzcd}
	X & {X\times_S Y} & Y \\
	T
	\arrow["{\Gamma_f}", from=1-1, to=1-2]
	\arrow["q"', shift right, from=1-2, to=1-3]
	\arrow["{f\circ p}", shift left, from=1-2, to=1-3]
	\arrow["{p\circ h}", color={rgb,255:red,92;green,92;blue,214}, dashed, from=2-1, to=1-1]
	\arrow["h"', from=2-1, to=1-2]
\end{tikzcd}\]
\end{prooff}
\begin{defn}
    A morphism of schemes $v: Y \rightarrow S$ is called separated if the following equivalent conditions are satisfied.
\begin{enu}
\item The diagonal morphism $\Delta_{Y / S}$ is a closed immersion.
\item For every $S$-scheme $X$ and for any two $S$-morphisms $f, g: X \rightarrow Y$ the equalizer $\operatorname{Eq}(f, g) \subseteq X$ is a closed subscheme of $X$.
\item For every $S$-scheme $X$ and for any $S$-morphism $f: X \rightarrow Y$ its graph $\Gamma_f$ is a closed immersion.
\end{enu}
\end{defn}
\begin{prooff}
    (1) implies (2): closed immersion stalbe under base change 

    (2) implies (3): By Lemma\,\ref{lemma: Gamma f is a equalizer}. 

    (3) implies (1): Take $X=Y$ and $f=\text{id}$.
\end{prooff}
\begin{prop}
    These are basic examples of separated morphism. 
    \begin{enu} 
    \item  Every monomorphism of schemes (and in particular every immersion) is separated.
    \item The property of being separated is stable under composition, stable under base change, and local on the target.
    % \item If the composition $X \rightarrow Y \rightarrow Z$ of two morphisms is separated, $X \rightarrow Y$ is separated.
    \end{enu}
\end{prop}
\begin{prooff}
    (1): $f:X\rightarrow S$ be a monomorphism, then $X$ is isomorphic to $X\times_S X$ under $\Delta_{X/S}$ 
    since $X$ also satisfies the universal property of fiber product. 
\end{prooff}
\begin{prop}
    Let $S=\operatorname{Spec} R$ be an affine scheme and let $X$ be an $S$-scheme. Then the following assertions are equivalent.
\begin{enu} 
    \item $X$ is separated.
    \item For every two open affine sets $U, V \subseteq X$ the intersection $U \cap V$ is affine and
    $$
    \rho_{U, V}: \Gamma\left(U, \mathscr{O}_X\right) \otimes_R \Gamma\left(V, \mathscr{O}_X\right) \rightarrow \Gamma\left(U \cap V, \mathscr{O}_X\right), \quad(s, t) \mapsto s_{\mid U \cap V} \cdot t_{\mid U \cap V}
    $$
    is surjective.
    \item There exists an open affine covering $X=\bigcup_i U_i$ such that $U_i \cap U_j$ is affine and $\rho_{U_i, U_j}: \Gamma\left(U_i, \mathscr{O}_X\right) \otimes_R \Gamma\left(U_j, \mathscr{O}_X\right) \rightarrow \Gamma\left(U_i \cap U_j, \mathscr{O}_X\right)$ is surjective for all $i, j$.
\end{enu}
\end{prop}
\begin{exam}
    $k$ be a field, $\bb{P}^n_k$ is separated. 
\end{exam}




\newpage 
\section{Quasi-coherent modules}


\chapter{Algebraic Group}




\end{document}