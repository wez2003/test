\documentclass[12pt]{article}
%宏包
\usepackage{amsmath}%数学符号
\usepackage{amssymb}%数学符号
\usepackage{amsthm}%数学符号
\usepackage{geometry}%界面布局
\usepackage{natbib}%bibtex
\usepackage{tikz}%交换图
\usepackage{tikz-cd}%交换图
\usepackage{quiver}%交换图
\usepackage{float}%浮动体固定
\usepackage{caption}%图片标题
\usepackage[colorlinks,linkcolor=blue]{hyperref}%超链接
\usepackage{enumerate}%计数列表
\usepackage{tabularx}%控制列宽
\usepackage{CTEX}
\usepackage{caption}%字体
\ctexset{today=old}
   
%页面设置
\linespread{1.2}
\geometry{a4paper,left=2cm,right=2cm,top=2.5cm,bottom=2cm}
%\geometry{a4paper,left=2cm,right=2cm,top=2.5cm,bottom=2cm}
%环境和宏指令
\newenvironment{prooff}{{\noindent\it\textcolor{cyan!40!black}{Proof}:}\,}{\par \vskip 1cm}
\newenvironment{proofff}{{\noindent\it\textcolor{cyan!40!black}{Proof of the lemma}:}\,}{\qed \par}
\newcommand{\bbrace}[1]{\left\{ #1 \right\} }
\newcommand{\bb}[1]{\mathbb{#1}}
\newcommand{\p}{^{\prime}}
\renewcommand{\mod}[1]{(\text{mod}\,#1)}
\newcommand{\blue}[1]{\textcolor{blue}{#1}}
\newcommand{\spec}[1]{\text{Spec}({#1})}
\newcommand{\rarr}[1]{\xrightarrow{#1}}
\newcommand{\larr}[1]{\xleftarrow{#1}}
\newcommand{\emptyy}{\underline{\quad}}
\newenvironment{enu}{\begin{enumerate}[(1)]}{\end{enumerate}}
%ctrl+点击文本返回代码  选中代码 ctrl+alt+j 为代码查找文本

%定理环境
\theoremstyle{definition}
\newtheorem{defn}{Definition}[section]
\newtheorem{coro}[defn]{Corollary}
\newtheorem{theo}[defn]{Theorem}
\newtheorem{exer}[defn]{Exercise}
\newtheorem{rema}[defn]{Remark}
\newtheorem{lem}[defn]{Lemma}
\newtheorem{prop}[defn]{Proposition}
\newtheorem{nota}[defn]{Notation}
\newtheorem{exam}[defn]{Example}
\title{Personal History Statement}
\author{Erzhuo Wang}
\date{}
\begin{document}
\maketitle
\section{Experiences}
I discovered my passion for Number Theory in high school because of a magic theorem says an odd prime number 
$p \equiv 1\mod{4}$ if and only if it can be written as the sum of two perfect squares. 
After entering campus, I began to learn number theory in a structured way.

During my bachelor's degree,
I read the initial two chapters and final chapter (Algebraic number field, local field and L-functions) of Neukirch's Algebraic Number Theory, 
the first three chapters of Fred Diamond's A First Course in Modular Form (basic concepts in Modular Form, dimension formula)
and Tate's thesis (analytic continuation of Hecke L-functions).  

In my first year on campus, I focused on studying foundational subjects in Mathematics such as Linear Algebra and Analysis.
I attended a seminar Professor Xi organized that year. 
We discussed the first few chapters of GTM 84, covering propositions of Gauss Sum and some results about the number of 
solutions of some special
Diophantus equations over finite field. 

During the second year summer vacation, 
I successfully completed the written test and interview to participate in a summer school held by the Chinese Academy of Sciences. 
At the school, I learned basic concepts in Algebraic Geometry, Algebraic Number Theory, and Representation Theory. 

In the third year, I enrolled in the Analytic Number Theory course taught by Professor Xi. 
I learned about classical ideas about exponential sums, Sieve method and the Prime Number Theorem. 
After that, I began reading Tate's thesis and obtained a comprehensive understanding of Hecke L-functions. 

\section{How These Experiences Contribute to Academic Study}
These experiences have profoundly shaped my academic development in several ways:
\begin{enu} 
\item Solid Foundation: Over three years of focused study in Number Theory have equipped me with a robust understanding of classical results in the field. I am well-prepared to explore advanced topics and engage with classical papers on automorphic forms.

\item Motivation for Further Exploration: The diverse and in-depth exposure to topics such as Modular Forms and Hecke 
L-functions has fueled my passion for Number Theory. I am particularly excited by the rich landscape of problems in automorphic forms that remain to be explored, and I am eager to delve deeper into this area in my future studies.
\end{enu}








\end{document}