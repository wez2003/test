\documentclass{beamer}
%Information
\title{Combination 3: Binomial Coefficents}
\titlegraphic{\hfill\includegraphics[height=1cm]{orange.png}}
\institute{Youth STEM Academy}
\author{Erzhuo Wang}
\date{\today}
   
%Theme
\usetheme[block=fill, sectionpage=none]{metropolis}
\useoutertheme{infolines}
\useinnertheme{metropolis}
\setbeamertemplate{blocks}[rounded][shadow=false]
\setbeamertemplate{items}[ball]
\setbeamertemplate{sections/subsections in toc}[ball]
\setbeamertemplate{headline}{}
\logo{YSA}
\usecolortheme{custom}
%\usetheme{Madrid}
%\usetheme{Heverlee}

%Setting
\usepackage[UTF8,noindent]{ctexcap}
\ctexset{today=old}
\theoremstyle{definition}
\newtheorem{defn}{Definition}[section]
\newtheorem{coro}[defn]{Corollary}
\newtheorem{theo}[defn]{Theorem}
\newtheorem{exer}[defn]{Exercise}
\newtheorem{rema}[defn]{Remark}
\newtheorem{lem}[defn]{Lemma}
\newtheorem{prop}[defn]{Proposition}
\newtheorem{nota}[defn]{Notation}
\newtheorem{exam}[defn]{Example}
\newtheorem{ques}[defn]{Question}

\newenvironment{prooff}{{\noindent\it\textcolor{cyan!40!black}{Proof}:}\,}{\par}
\newenvironment{proofff}{{\noindent\it\textcolor{cyan!40!black}{Proof of the lemma}:}\,}{\qed \par}
\newcommand{\bbrace}[1]{\left\{ #1 \right\} }
\newcommand{\bb}[1]{\mathbb{#1}}
\newcommand{\p}{^{\prime}}
\renewcommand{\mod}[1]{(\text{mod}\,#1)}
\newcommand{\blue}[1]{\textcolor{blue}{#1}}
\newcommand{\spec}[1]{\text{Spec}({#1})}
\newcommand{\rarr}[1]{\xrightarrow{#1}}
\newcommand{\larr}[1]{\xleftarrow{#1}}
\newcommand{\emptyy}{\underline{\quad}}
\newenvironment{enu}{\begin{enumerate}[(1)]}{\end{enumerate}}
%ctrl+点击文本返回代码  选中代码 ctrl+alt+j 为代码查找文本

\begin{document}
\begin{frame}
    \titlepage
\end{frame}
\begin{frame}{Introduction}
    这节课我们主要讲一些组合数的基本性质, 然后做一些组合方面的综合性题目来巩固之前学的内容。


\end{frame}
\begin{frame}{Binomial Theorem(二项式定理)}
    \begin{theo}[Binomial Theorem]
        % 组合数基本性质+ 真题训练  
        The expansion of $(x+y)^n$ for positive integer $n$ is
        \begin{equation*}
            (x+y)^n=\binom{n}{0}x^n+\binom{n}{1}x^{n-1}y^1+\binom{n}{2}x^{n-2}y^2+\dots+\binom{n}{n}x^0y^n
        \end{equation*}
    \end{theo}
    \begin{exam}
        Show that $2^n= \binom{n}{0}+\binom{n}{1}+\binom{n}{2}+\cdots+\binom{n}{n}$
    \end{exam}
\end{frame}
% \begin{frame}{Hundreds digit of $2011^{2011}$(AMC 10B, 2011-23)}
%     What is the hundreds digit of $2011^{2011}$?
%     \pause

%     \begin{prooff}
%         By Binomial Theorem, hundreds digit of $2011^{2011}$ is equal to that of
%         $$
%             \begin{aligned}
%                   & 1+\binom{2011}{1}(10)\left(1^{2010}\right)+\binom{2011}{2}(10)^2\left(1^{2009}\right) \\
%                 = & 1+2011 \cdot 10+2011 \cdot 1005 \cdot 100                                             \\
%                 = & 1+2011 \cdot 100510
%             \end{aligned}
%         $$

%         Finally, the hundreds digit of this number is equal to that of $1+11 \cdot 510=5611$, so the requested hundreds digit is 6 .

%     \end{prooff}
% \end{frame}
\begin{frame}{Properties of Binomial Coefficients}
    If $k$ and $n$ are two integers and $0 \leq k \leq n$, then
    \begin{itemize}
        \item $\binom{n}{k}=\binom{n}{n-k}$
        \item $\binom{n}{k}+\binom{n}{k+1}=\binom{n+1}{k+1}$
        \item $k\binom{n}{k}=n\binom{n-1}{k-1}$
        \item
              \begin{equation*}
                  \binom{k}{k}+\binom{k+1}{k}+\cdots+\binom{n}{k}=\binom{n+1}{k+1}
              \end{equation*}
        \item  \begin{equation*}
                  \binom{n}{0}-\binom{n}{1}+\binom{n}{2}+\cdots+(-1)^n\binom{n}{n}=0
              \end{equation*}
    \end{itemize}
\end{frame}
% \begin{frame}{A Special Sequence}
%     Let $(a_1,\dots,a_{10})$ be a list of $1,2,3,\dots,10$ such that for all $2\le i\le 10$, either $a_i+1$ or $a_i-1$ or both appear somewhere before $a_i$ in
%     the list. How many such lists are there?
% \end{frame}
\begin{frame}{Example}
    \begin{ques}
        What is the coefficient of $x^7$ in the polynomial $(x+3)^{10}$.
    \end{ques}
\end{frame}
\begin{frame}{Exercise}
    \begin{ques}
        Two people flip a fair coin. One flipped it $5$ times while the other $6$ times. What is the probability that the second person
        flipped more tails than the first person?

        两个人抛一枚均匀的硬币。一个人翻了5次,另一个人翻了6次。求第二个人抛到反面的次数比第一个人多的概率?
    \end{ques}
    \begin{ques}
        A box contains exactly five chips, three red and two white. Chips are randomly removed one at a time without
        replacement until all the red chips are drawn or all the white chips are drawn. What is the probability that the last chip drawn is white?
    
        一个盒子里正好有5个筹码,3个红牌,2个白牌。每次随机取出一张筹码,直到所有红牌或白牌都被取出为止。最后一张牌是白色的概率是多少?
    \end{ques}
\end{frame}
\begin{frame}{Exercise}
    \begin{ques}
        Two different numbers are randomly selected from $-2,-1,0,3,4,5$ and multiplied together. What is the probablity
    that the product is $0$?
    从$-2,-1,0,3,4,5$中随机选择两个不同的数字并相乘, 为0的概率是多少?
    \end{ques}
    \begin{ques}
        在海滩上,50个人戴着太阳镜,35个人戴着帽子。有些人既戴太阳镜又戴帽子。
        如果随机选择一个戴帽子的人,
        这个人也戴太阳镜的概率是$\frac{2}{5}$. 如果随机选择一个戴太阳镜的人,这个人也戴帽子的概率是多少?
    \end{ques}
\end{frame}
\begin{frame}
    \begin{ques}[AMC 8, 2019-25]
        Alice has $24$ apples. In how many ways can she sharethem with Becky and Chris so that each of the three people has at least two
        apples?

        爱丽丝有24个苹果。她有多少种方法可以与贝基和克里斯分享这些苹果,使得这样三个人每人至少有两个苹果?
    \end{ques}
    \begin{ques}[AMC8, 2020-23]
        Five different awards are to be given to three students. Each student will receive at least one award. In how many different ways can the awards be distributed?

        (A) 120 (B) 150 (C) 180 (D) 210 (E) 240
    \end{ques}
\end{frame}
\begin{frame}{Challenging Problem}
\begin{ques}
    What is the probability that any $50$ person has the same birthday? (Assuming there are $365$ days in a year)
    
    任取$50$个人, 存在两人生日相同的概率是多少?(假设一年有$365$天)
\end{ques}
\begin{ques}
    A set of teams held a round-robin tournament in which every team played every other team exactly once. Every teamwon 10 games and lost 10 games. How many
    sets of three teams $\bbrace{A,B,C}$ were there in which $A$ beat $B$, $B$ beat $C$, $C$ beat $A$.
\end{ques}

\end{frame}

\end{document}