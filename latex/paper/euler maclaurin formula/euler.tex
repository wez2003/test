\documentclass[a4paper,12pt]{ctexart}
%宏包
\usepackage{amsmath}
\usepackage{amssymb}
\usepackage{amsthm}
\usepackage{geometry}
\usepackage{natbib}%bibtex
\usepackage[dvipsnames]{xcolor}
\usepackage{tcolorbox}
\usepackage{enumerate}
\usepackage{tikz}
\usepackage{tikz-cd}
\usepackage{float}
\usepackage{caption}
\usepackage[colorlinks,linkcolor=blue]{hyperref}
\usepackage{enumerate}

%页面设置
\linespread{1.1}
\geometry{a4paper,left=2cm,right=2cm,top=2.5cm,bottom=2cm}

%环境和宏指令
\newenvironment{prooff}{{\noindent\it\textcolor{black}{Proof}:}\quad}{\par}
\newcommand{\bbrace}[1]{\left\{ #1 \right\} }
\newcommand{\bb}[1]{\mathbb{#1}}
\newcommand{\p}{^{\prime}}
\renewcommand{\mod}[1]{(\text{mod}\,#1)}
%ctrl+点击文本返回代码  选中代码 ctrl+alt+j 为代码查找文本


%定理环境
\theoremstyle{definition}
\newtheorem{defn}{Definition}
\newtheorem{coro}[defn]{Corollary}
\newtheorem{theo}[defn]{Theorem}
\newtheorem{exer}[defn]{Exercise}
\newtheorem{rema}[defn]{Remark}
\newtheorem{lem}[defn]{Lemma}
\newtheorem{prop}[defn]{Proposition}

\title{Euler-Maclaurin 公式及其简单应用}
\date{}
\author{王尔卓}

\begin{document}
\maketitle
\section{Bernoulli数与Bernoulli多项式}
\subsection{递推公式}
考虑函数
\begin{align*}
    f(x)=\begin{cases}
        \dfrac{e^x-1}{x} & \quad x\neq 1            \\
        \quad 1          & \quad                x=1
    \end{cases}
\end{align*}
其在$\mathbb{R}$上有幂级数展开,
\begin{equation*}
    f(x)=\sum_{n=0}^{\infty}\frac{x^n}{(n+1)!}
\end{equation*}
且首项不为0,因此$\dfrac{1}{f(x)}$在$0$的某个邻域内也有幂级数展开,
设为:
\begin{equation*}
    \sum_{n=0}^{\infty}\frac{B_n}{n!}x^n
\end{equation*}
我们将$B_n$称为Bernoulli数,Bernoulli数有如下递推公式:
$B_0=1,B_1=-\frac{1}{2}$,$B_n=\sum_{j=0}^{n}\binom{n}{j}B_j$

\begin{prooff}
    由定义:
    \begin{align*}
        x & =(\sum_{n=0}^{\infty}\frac{B_n}{n!}x^n)(\sum_{n=1}^{\infty}\frac{x^n}{n!})                         \\
          & =\sum_{n=1}^{\infty}\frac{x^n}{n!}+\sum_{n=2}^{\infty}\sum_{i+j=n}\frac{B_i}{i!}\frac{x^{i+j}}{j!} \\
          & =\sum_{n=1}^{\infty}\frac{x^n}{n!}+\sum_{n=2}^{\infty}\sum_{i+j=n}\frac{B_i}{n!}\binom{n}{i}x^n
    \end{align*}
    比较两端系数得到:
    \begin{equation*}
        B_n=\sum_{j=0}^{n}\binom{n}{j}B_j
    \end{equation*}
\end{prooff}
\subsection{奇数点取值}
令$$g(x)=\frac{x}{e^x-1}$$
\begin{align*}
    g(x)-g(-x)=\frac{x}{e^x-1}-\frac{-x}{e^{-x}-1}=-x
\end{align*}
因此$$g(x)+\frac{x}{2}$$为偶函数,进而$$(\frac{1}{f(x)})^{(3)}$$为奇函数,从而

$$B_{2n+1}=0 \quad \forall n\in \mathbb{Z}_{>0} $$

\subsection{Bernoulli多项式}
我们再给出Bernoulli多项式的定义:
\begin{align*}
     & B_n(x)\text{以1为周期}                               \\
     & B_0(x)=1,\forall x \in [0,1)\quad                    \\
     & B^{\prime}_{n+1}(x)=(n+1)B_n(x) \quad \forall n\ge 0 \\
     & \int_0^1  B_n(x)\text{d}x =0\quad  \forall n\ge 1
\end{align*}

则在$t=0$的某个去心邻域内有:
\begin{equation}
    \sum_{n=0}^{\infty}\frac{B_n(x)}{n!}t^n\quad \text{收敛于} \quad \frac{te^{tx}}{e^t-1}\quad \forall x\in[0,1)
\end{equation}
\begin{prooff}
    计算易得:$$B_1(x)=\bbrace{x} -\frac{1}{2}$$ $$ B_2(x)=\bbrace{x}^2-\bbrace{x}+\frac{1}{6}$$
    $$B_3(x)=\bbrace{x}^3-\frac{2}{3}\bbrace{x}^2+\frac{1}{2}\bbrace{x} $$
    我们想归纳证明:
    \begin{equation}
        B_n(x)=\sum_{k=0}^{n}\binom{n}{k}B_k(0)\bbrace{x}^{n-k}
    \end{equation}
    假设对所有$\le n$的数成立,则对$n+1$有
    \begin{equation*}
        B_{n+1}^{\prime}(x)=(n+1)\sum_{k=0}^{n}\binom{n}{k}B_k(0)\bbrace{x}^{n-k}
    \end{equation*}

    对上式在$0$到$x$上积分有:

    \begin{align}
        B_{n+1}(x)-B_{n+1}(0) & =\int _{0}^{x}(n+1)B_n(x)                                                             \\
                              & =\sum_{k=0}^{n}(n+1)\frac{n!}{k!(n-k)!}B_k(0)\bbrace{x}^{n-k+1}\frac{1}{n-k+1} \notag \\
                              & =\sum_{k=0}^nB_k(0)\binom{n+1}{k}\bbrace{x}^{n-k+1} \notag
    \end{align}
    进而:
    \begin{align*}
        B_{n+1}(x)=\sum_{k=0}^{n+1}B_k(0)\binom{n+1}{k}\bbrace{x}^{n-k+1}
    \end{align*}
    其次我们想证明:
    \begin{equation}
        |B_n(0)|\le n!
    \end{equation}
    $(3)$式在$0$到$1$上积分有:

    \begin{equation*}
        0=\sum_{k=0}^{n}\binom{n}{k}B_k(0)\frac{1}{n-k+1}=\sum_{k=0}^n \binom{n+1}{k}B_k(0)\frac{1}{n+1}
    \end{equation*}
    从而:

    \begin{equation*}
        (n+1)B_n(0)=\sum_{k=0}^{n-1}-\binom{n+1}{k}B_k(0)
    \end{equation*}
    我们对$(5)$采用归纳法,不难验证$n=1,2,3$时结论成立,假设对所有$\le n-1$的数成立,则对$n$有
    \begin{align*}
        |(n+1)B_n(0)| & =|\sum_{k=0}^{n-1}-\binom{n+1}{k}B_k(0)|\le 1+\sum_{k=1}^{n-1}\binom{n+1}{k}k! \\
                      & \le 1+(n+1)!\sum_{k=2}^{n}\frac{1}{k!}                                         \\
                      & \le 1+(n+1)!(e-2)\le (n+1)!
    \end{align*}
    进而:
    \begin{equation*}
        |B_n(0)|\le n!
    \end{equation*}
    通过上面的结论我们进一步归纳证明:

    \begin{equation}
        |B_{n}(x)|\le n \times  n!
    \end{equation}
    $n=1,2,3$显然成立,假设对所有$\le n$的数成立,由$(4)$

    \begin{equation*}
        |B_{n+1}(x)|\le (n+1)!+(n+1)n\times n! \le (n+1)(n+1)!
    \end{equation*}
    有了上述对Bernoulli数和Bernoulli多项式的上界估计,我们来证明原命题,
    考虑任意的$|t|\le \dfrac{1}{2}$,由Weierstrass判别法,$$\sum_{n=0}^{\infty}\frac{B_n(x)}{n!}t^n$$在$x\in \mathbb{R}$上一致收敛,因此可以逐项积分,
    设和函数为$f(x)$

    又因为
    \begin{align*}
        \sum_{n=0}^{\infty}\frac{B_n^{\prime}(x)}{n!}t^n=\sum_{n=1}^{\infty}\frac{nB_{n-1}(x)}{n!}t^n=\sum_{n=0}^{\infty}\frac{B_n(x)}{n!}t^{n+1}
    \end{align*}
    因此可以逐项求导,且$$f^{\prime}(x)=tf(x)$$
    解微分方程组可知:

    \begin{equation*}
        f(x)=e^{tx}g(t)
    \end{equation*}

    两端对$x$在$0$到$1$上积分有:

    \begin{equation*}
        g(t)h(t)=\int _0^1f(x)\text{d}x=\int _0^1\sum_{n=0}^{\infty}\frac{B_n(x)}{n!}t^n=\sum_{n=0}^{\infty}\int _0^1\frac{B_n(x)}{n!}t^n=1
    \end{equation*}
    $$h(t)=\begin{cases}
            \dfrac{e^t-1}{t} & \quad t\neq 0 \\
            1                & \quad t=0
        \end{cases}
    $$

    因此\begin{equation}f(x)=\begin{cases}
            \dfrac{te^{tx}}{e^t-1} & \quad \forall t\in [-\frac{1}{2},0)\cup(0,\frac{1}{2}] \\
            1                      & \quad  t=1
        \end{cases}
    \end{equation}
\end{prooff}
在$(7)$中令$x=0$可以得到多项式满足$$B_n(0)=B_n$$
其中$B_n$为Bernoulli数
\section{Euler-Maclaurin公式}
\begin{theo}
$a,b\in \mathbb{Z}$,$f\in C^{k+1}([a,b])$,$k\in \mathbb{Z}_{\ge 0}$,则
\begin{align}
    \sum_{a<n\le b}f(n) & =\int _{a}^{b}f(x)\text{d}x+\sum_{j=1}^{k+1}(-1)^j \frac{B_j}{j!}(f^{(j-1)}(a)-f^{(j-1)}(b))\notag \\
                        & + \frac{(-1)^k}{(k+1)!} \int_a^b B_{k+1}(x)f^{(k+1)}(x)\text{d}x
\end{align}
\end{theo}
\begin{prooff}
    由欧拉求和公式,并不断分部积分

    \begin{align*}
        \sum_{a<n\le b}f(n) & =\int _{a}^{b}f(x)\text{d}x+\int_a^b f^{\prime}(x)B_1(x)\text{d}x +(f(a)-f(b))B_1(0)                                              \\
                            & =\int _{a}^{b}f(x)\text{d}x-\frac{1}{2}\int_a^bB_2(x)f^{(2)}(x)\text{d}x +(f(a)-f(b))B_1(0)                                       \\
                            & +\frac{1}{2}(f^{\prime}(b)-f^{\prime}(a))B_2(0)                                                                                   \\
                            & =\int _{a}^{b}f(x)\text{d}x-\frac{1}{2}(\frac{B_3(b)f^{(2)}(b)-B_3(a)f^{(2)}(a)}{3}-\int_a^b\frac{B_3(x)}{3}f^{(3)}(x)\text{d}x ) \\
                            & +(f(a)-f(b))B_1(0) +\frac{1}{2}(f^{\prime}(b)-f^{\prime}(a))B_2(0)                                                                \\
                            & =\dots =  \int _{a}^{b}f(x)\text{d}x+\sum_{j=1}^{k+1}(-1)^j \frac{B_j}{j!}(f^{(j-1)}(a)-f^{(j-1)}(b))\notag                       \\
                            & + \frac{(-1)^k}{(k+1)!} \int_a^b B_{k+1}(x)f^{(k+1)}(x)\text{d}x
    \end{align*}

\end{prooff}
\section{调和级数部分和估计}
\begin{theo}
    当$N\ge10$时,存在$\theta \in [0,1]$,使得:
\end{theo}
\begin{equation}
    \sum_{n\le N}\frac{1}{n}=\text{log}N+\gamma+\frac{1}{2N}-\frac{1}{2N^2}+\frac{\theta}{60N^4}
\end{equation}
\begin{prooff}
    令$$f(x)=\frac{1}{x}$$,由欧拉麦克劳林公式:
    \begin{align*}
        \sum_{1\le n\le N }\frac{1}{n} & =1+\text{log}N+\frac{1}{2}(\frac{1}{N}-1)-\frac{1}{12}(\frac{1}{N^2}-1)+\frac{1}{120}(\frac{1}{N^4}-1)                                                                                 \\
                                       & +\int_1^N-\frac{1}{x^5}B_4(x)\text{d}x                                                                                                                                                 \\
                                       & =1+\text{log}N+\frac{1}{2}(\frac{1}{N}-1)-\frac{1}{12}(\frac{1}{N^2}-1)+\frac{1}{120}(\frac{1}{N^4}-1)                                                                                 \\
                                       & +\int_1^{+\infty}-\frac{1}{x^5}B_4(x)\text{d}x+\int_N^{+\infty}\frac{1}{x^5}B_4(x)\text{d}x                                                                                            \\
                                       & =1+\text{log}N+\frac{1}{2}(\frac{1}{N}-1)-\frac{1}{12}(\frac{1}{N^2}-1)+\frac{1}{120}(\frac{1}{N^4}-1)                                                                                 \\
                                       & +\int_1^{+\infty}-\frac{1}{x^5}B_4(x)\text{d}x+\int_N^{+\infty}\frac{1}{x^5}(\bbrace{x}^4-2\bbrace{x}^3+\bbrace{x}^2-\frac{1}{30})\text{d}x                                            \\
                                       & =\text{log}N+\frac{1}{2N}-\frac{1}{12N^2}+\frac{69}{120}+\int_1^{+\infty}-\frac{1}{x^5}B_4(x)\text{d}x+\int_N^{+\infty}\frac{1}{x^5}(\bbrace{x}^4-2\bbrace{x}^3+\bbrace{x}^2)\text{d}x
    \end{align*}
    又因为
    \begin{align*}
         & 0\le \int_N^{+\infty}\frac{1}{x^5}(\bbrace{x}^4-2\bbrace{x}^3+\bbrace{x}^2)\text{d}x=\sum_{n=N}^{+\infty}\int_n^{n+1}\frac{\bbrace{x}^2(\bbrace{x}-1)^2}{x^5}\text{d}x \\
         & \le \sum_{n=N}^{+\infty}\frac{1}{30n^5}
    \end{align*}
    考虑当$x\ge 10$时
    \begin{align*}
        g(x) & = x((x+1)^4-x^4)-2(x+1)^4   \\
             & = 2x^4-2x^3-8x^2-7x-2 \ge 0 \\
    \end{align*}
    因此
    \begin{equation}
        \frac{1}{x^5}\le \frac{1}{2}(\frac{1}{x^4}-\frac{1}{(x+1)^4})
    \end{equation}

    由$(10)$,当$N\ge 10$时
    \begin{equation}
        \sum_{n=N}^{+\infty}\frac{1}{30n^5}\le \sum_{n=N}^{+\infty}\frac{1}{60}(\frac{1}{n^4}-\frac{1}{(n+1)^4})\le \frac{1}{60N^4}
    \end{equation}
    因此当$N\ge10$时,存在$\theta \in [0,1]$,使得:
    \begin{equation*}
        \int_N^{+\infty}\frac{1}{x^5}(\bbrace{x}^4-2\bbrace{x}^3+\bbrace{x}^2)\text{d}x=\frac{\theta}{60N^4}
    \end{equation*}
    同时不难看出:
    \begin{equation*}
        \frac{69}{120}+\int_1^{+\infty}-\frac{1}{x^5}B_4(x)\text{d}x=\gamma
    \end{equation*}
    其中$\gamma$为欧拉常数

    因此我们得到:当$N\ge10$时,存在$\theta \in [0,1]$,使得:
    \begin{equation}
        \sum_{n\le N}\frac{1}{n}=\text{log}N+\gamma+\frac{1}{2N}-\frac{1}{12N^2}+\frac{\theta}{60N^4}
    \end{equation}
\end{prooff}
\section{Riemann Zeta函数的延拓}
\subsection{延拓的手段}
\begin{theo}
    $k\in \mathbb{Z}_{>0}$,$\forall s>1$

    \begin{equation}
        \zeta(s)=\frac{1}{s-1}+\frac{1}{2}+\sum_{j=2}^{k+1}\binom{s+j-2}{j-1}\frac{B_j(0)}{j}-\binom{s+k}{k+1}\int_1^{+\infty}\frac{B_{j+1}(x)}{x^{k+s+1}}\text{d}x
    \end{equation}
    并由此说明可以将$\zeta$函数延拓为$\mathbb{R}\textbackslash\left\{1\right\}$上的函数
\end{theo}
\begin{prooff}
    设$$f(x)=\frac{1}{x^s}\quad \forall s>1$$
    对$f(x)$求$k$阶导数:
    \begin{equation*}
        f^{(k)}(x)=(-1)^k\binom{s+k-1}{k}k!\,x^{-(s+k)}\quad \forall k\in \mathbb{Z}_{>0}
    \end{equation*}
    对$f(x)$应用欧拉麦克劳林公式:
    \begin{align*}
        \sum_{n\le N}\frac{1}{n^s} & =1+\sum_{1<n\le N}\frac{1}{n^s}=1+\int_1^N \frac{1}{x^s}\text{d}x +\frac{1}{2}(\frac{1}{N^s}-1)                       \\
                                   & +\sum_{j=2}^{k+1}(-1)^j\frac{B_j(0)}{j!}(j-1)!\binom{s+j-2}{j-1}(-1)^{(j-1)}(\frac{1}{N^s}-1)                         \\
                                   & +\int_1^N (-1)^k\frac{1}{(k+1)!}B_{k+1}(x)(-1)^{(k+1)}(k+1)!\binom{s+k}{k+1}x^{-(x+k+1)}\text{d}x                     \\
                                   & =\frac{1}{s-1}+\frac{1}{2}-\frac{1}{(s-1)N^{s-1}}+\sum_{j=2}^{k+1}\frac{B_j(0)}{j}\binom{s+j-2}{j-1}(1-\frac{1}{N^s}) \\
                                   & -\int_1^N B_{k+1}(x)\binom{s+k}{k+1}x^{-(x+k+1)}\text{d}x
    \end{align*}

    令$n\to \infty $有:
    \begin{equation}
        \zeta(s)=\frac{1}{s-1}+\frac{1}{2}+\sum_{j=2}^{k+1}\binom{s+j-2}{j-1}\frac{B_j(0)}{j}-\binom{s+k}{k+1}\int_1^{+\infty}\frac{B_{j+1}(x)}{x^{k+s+1}}\text{d}x
    \end{equation}
    根据Bernoulli多项式的周期性,由$Dirchlet$判别法,右端$$\int_1^{+\infty}\frac{B_{j+1}(x)}{x^{k+s+1}}\text{d}x$$
    在$s+k>-1$均收敛,由$k$的任意性,可将$\zeta$函数延拓为$\mathbb{R}\textbackslash\left\{1\right\}$上的函数
    \subsection{平凡零点的判定}
    在上述$\zeta$函数的延拓中,试证明全体负偶数点取值为$0$,即
\end{prooff}
$$\zeta(-2n)=0\quad \forall n\in\mathbb{Z}_{>0}$$
\begin{prooff}
    取$k=2n+1$,应用$(1)$式以及$B_{2n+1}=0 \quad \forall n\in \mathbb{Z}_{>0}$
    我们得到:
    \begin{align*}
        \zeta(-2n) & =\frac{1}{-2n-1}+\frac{1}{2}+\sum_{j=2}^{2n+2}\binom{-2n-2+j}{j-1}\frac{B_j}{j}            \\
                   & =-\frac{1}{2n+1}+\frac{1}{2}+\sum_{j=2}^{2n+2}\frac{1}{2n+1}\binom{2n+1}{j}(-1)^{(j-1)}B_j \\
                   & =-\frac{1}{2n+1}+\frac{1}{2}-\frac{1}{2n+1}\sum_{j=2}^{2n+1}\binom{2n+1}{j}B_j             \\
                   & =-\frac{1}{2n+1}+\frac{1}{2}-\frac{1}{2n+1}(B_{2n+1}-B_0-(2n+1)B_1)                        \\
                   & =0
    \end{align*}


\end{prooff}



\end{document}