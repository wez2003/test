\documentclass[a4paper,12pt]{ctexart}
%宏包
\usepackage{amsmath}
\usepackage{amssymb}
\usepackage{amsthm}
\usepackage{geometry}
\usepackage{natbib}%bibtex
\usepackage{enumerate}
\usepackage{tikz}
\usepackage{tikz-cd}
\usepackage{quiver}
\usepackage{float}
\usepackage{caption}
\usepackage[colorlinks,linkcolor=blue]{hyperref}
\usepackage{enumerate}

%页面设置
\linespread{1.2}
\geometry{a4paper,left=2cm,right=2cm,top=2.5cm,bottom=2cm}

%环境和宏指令
\newenvironment{prooff}{{\noindent\it\textcolor{black}{Proof}:}\quad}{\par}
\newcommand{\bbrace}[1]{\left\{ #1 \right\} }
\newcommand{\bb}[1]{\mathbb{#1}}
\newcommand{\p}{^{\prime}}
\renewcommand{\mod}[1]{(\text{mod}\,#1)}
%ctrl+点击文本返回代码  选中代码 ctrl+alt+j 为代码查找文本


%定理环境
\theoremstyle{definition}
\newtheorem{defn}{Definition}
\newtheorem{coro}[defn]{Corollary}
\newtheorem{theo}[defn]{Theorem}
\newtheorem{exer}[defn]{Exercise}
\newtheorem{rema}[defn]{Remark}
\newtheorem{lem}[defn]{Lemma}
\newtheorem{prop}[defn]{Proposition}




\title{Pell方程}
\author{王尔卓}
\date{2022年11月22日}
\begin{document}
\maketitle
% \tableofcontents
% \newpage
\section{解的结构}
\begin{defn}
    \textcolor{blue}{Pell方程}是指形如\begin{equation*}
        x^2-dy^2=\pm 1
    \end{equation*}的不定方程,其中$d>1$为正整数,而形如\begin{equation*}
        x^2-dy^2=N
    \end{equation*}的不定方程一般称为\textcolor{blue}{广义Pell方程}.
\end{defn}
本文完全剖析了Pell方程的解的结构,并就实二次域类数为1的情况刻画了广义Pell方程
解的结构,而给出这种刻画最犀利的工具自然是Dirichlet单位定理,因为其对一个数域
$K$的代数整数环$O_{K}$的单位给出了一种精确的描述.

在本文的处理中,我们先假设$d$无平方因子,并且在最后我们会说明有平方因子的情况下解的结构。

\begin{theo}[Dirichlet单位定理,\cite{antfkq}]
    设$K$为$n$次代数数域,$K$到$\mathbb{C}$有$r_1$个实嵌入,$r_2$对复嵌入,$r_1+2r_2=n$,则$K$的代数整数环$O_{K}$
    的单位构成的乘法群$U_k$可以表为:\begin{equation*}
        U_k=W_k\times V_k\label{dirichlet unit theorem}
    \end{equation*}
    其中$W_k$为数域$K$的单位根群,且为一个有限循环群,$V_k$为秩为$r_1+r_2-1$的自由$Abel$群.
\end{theo}
Dirichlet单位定理的证明是比较复杂的,但其对$O_{K}$单位群$U_k$给出的描述是简单易懂的,该定理告诉我们可以在$U_k$中找到$r=r_1+r_2-1$个元素$u_1,u_2,\dots,u_r$,使得
$U_k$中每个元素$u$可以表示为\begin{equation*}
    u=w\prod_{i=1}^{r}u_i^{a_i}
\end{equation*}
且这种表法在相差一个单位根的意义下唯一,此时将这组单位$\bbrace{u_1,\dots,u_r}$称为\textcolor{blue}{基本单位组},将每个$u_i$称为\textcolor{blue}{基本单位}.

而我们知道,对于代数整数环$O_{K}$中的元素$u$,$u\in U_k$等价于$N_{K/\mathbb{Q}}(u)=\pm 1$,而且一个基本的事实是,对于
实二次域$K=\mathbb{Q}(\sqrt{d})$,其代数整数环\begin{equation*}
    O_k=\begin{cases}
        k_1+k_2\sqrt{d}              & k_1,k_2\in \mathbb{Z} \quad \text{当}d\equiv 2,3\mod{4} \\
        k_1+k_2\dfrac{1+\sqrt{d}}{2} & k_1,k_2\in \mathbb{Z} \quad  \text{当}d\equiv 1 \mod{4}
    \end{cases}
\end{equation*}
对于实二次域$K=\mathbb{Q}(\sqrt{d})$,其只有两个实嵌入,没有复嵌入,且单位根群为$\bbrace{1,-1}$构成的乘法群,因此取其中一个基本单位
$\epsilon=a+b\omega $,则所有单位可以表示为
\begin{equation}
    U_k=\bbrace{\pm \epsilon^n:n\in \mathbb{Z}}\label{group of unit}
\end{equation}
其中\begin{equation*}
    \omega=\begin{cases}
        \sqrt{d}                   & \text{当}d\equiv 2,3\mod{4} \\
        \dfrac{1+\sqrt{d}}{2}\quad & \text{当}d\equiv 1 \mod{4}
    \end{cases}
\end{equation*}

而对于$d\equiv  2,3\mod{4}$的实二次域而言, $k_1+k_2\sqrt{d}$为单位等价于$k_1,k_2$为Pell方程$x^2-dy^2=\pm 1$的解.
不难看出在$U_k$能作为基本单位的只有$\pm \epsilon,\pm\epsilon^{-1}$,这四个数中有且仅有一个写成$k_1+k_2\sqrt{d}$的形式后满足$k_1,k_2>0$,
此后对于$d\equiv 2,3\mod{4}$的情况我们都取这样的元素作为基本单位,并不妨记为$\epsilon$,不难验证这个基本单位$\epsilon>1$.

\begin{theo}
    设$K=\mathbb{Q}(\sqrt{d}),d>0$且无平方因子,$d\equiv  2,3\mod{4},\epsilon=a+b\sqrt{d}$为基本单位,令$\epsilon^n=a_n+b_n\sqrt{d}$,则
    \begin{enumerate}
        \item  当$N_{K/\mathbb{Q}}(\epsilon)= 1$时,Pell方程$x^2-dy^2=-1$无整数解,Pell方程$x^2-dy^2=1$整数解
              解为$\bbrace{(\pm a_n,\pm b_n):n\in \mathbb{Z}_{\ge 0}}$
        \item 当$N_{K/\mathbb{Q}}(\epsilon)=- 1$时,Pell方程$x^2-dy^2=-1$整数解为$\bbrace{(\pm a_{2n+1},\pm b_{2n+1}):n\in \mathbb{Z}_{\ge 0}}$,
              Pell方程$x^2-dy^2=1$整数解为$\bbrace{(\pm a_{2n},\pm b_{2n}):n\in \mathbb{Z}_{\ge 0}}$
    \end{enumerate}
    \label{theo pell 4k+2,3}
\end{theo}
\begin{prooff}
    由于Pell方程的解关于原点对称,我们只需求出所有非负整数解,则其所有解只差一对正负号.注意到由(\ref{group of unit})刻画的单位群$U_k$的结构,方程$x^2-dy^2=\pm 1$的非负整数解为:
    $\bbrace{(a_{n},b_{n}):n\in \mathbb{Z}_{\ge 0}}$.当$N_{K/\mathbb{Q}}(\epsilon)= 1$时,所有$U_k$中元素的范数均为$1$,化作不定方程的语言就是说:
    Pell方程$x^2-dy^2=-1$无非负整数解,Pell方程$x^2-dy^2=1$非负整数解
    为$\bbrace{(a_n,b_n):n\in \mathbb{Z}_{\ge 0}}$,情况1证毕.

    对于情况2,$N_{K/\mathbb{Q}}(\epsilon)=- 1$,此时单位根群中元素的范数按奇偶呈正负交替的形式排列,
    从而Pell方程$x^2-dy^2=-1$非负整数解
    为$\bbrace{(a_{2n},b_{2n}):n\in \mathbb{Z}_{\ge 0}}$,$x^2-dy^2=1$非负整数解
    为$\bbrace{(a_{2n+1},b_{2n+1}):n\in \mathbb{Z}_{\ge 0}}$,从而得到全体解的表达形式.
\end{prooff}

\begin{theo}[解的递推关系]
    $x^2-dy^2=\pm 1$的所有正整数解$\bbrace{(a_n,b_n):n\in \mathbb{Z}_{\ge 1}}$可以由基本单位$\epsilon=a+b\sqrt{d}$按递推关系:
    \begin{equation*}
        \begin{bmatrix}
            a_n \\
            b_n
        \end{bmatrix}=
        \begin{bmatrix}
            a & bd \\
            b & a
        \end{bmatrix}^{n}\begin{bmatrix}
            1 \\
            0
        \end{bmatrix}\label{solution calu}
    \end{equation*}得到,
    且序列$\bbrace{a_n:n\in \mathbb{Z}_{>0}},\bbrace{b_n:n\in \mathbb{Z}_{>0}}$均为严格单调递增的序列.
\end{theo}
\begin{prooff}
    注意到:\begin{equation*}
        a_{n+1}+b_{n+1}\sqrt{d}=(a_{n}+b_{n}\sqrt{d})(a+b\sqrt{d})=(aa_n+bdb_n)+(ba_n+ab_n)\sqrt{d}
    \end{equation*}
    从而
    \begin{equation}
        \begin{bmatrix}
            a_{n+1} \\
            b_{n+1}
        \end{bmatrix}=
        \begin{bmatrix}
            a & bd \\
            b & a
        \end{bmatrix}
        \begin{bmatrix}
            a_n \\
            b_n
        \end{bmatrix}
        \label{a_n,b_n matrix}
    \end{equation}
    则\begin{equation*}
        \begin{bmatrix}
            a_{n+1} \\
            b_{n+1}
        \end{bmatrix}=
        \begin{bmatrix}
            a & bd \\
            b & a
        \end{bmatrix}\begin{bmatrix}
            a_n \\
            b_n
        \end{bmatrix}=\dots = \begin{bmatrix}
            a & bd \\
            b & a
        \end{bmatrix}^{n}\begin{bmatrix}
            a \\
            b
        \end{bmatrix}
    \end{equation*}而序列的递增性可以由递推关系轻松得到,因而一种求取Pell方程基本单位的计算方法是
    依次取$y=1,2,\dots$直到$x^2-dy^2=\pm 1$对$x$有解,此时求出的$x$的正值解就得到了基本单位,也顺便得到了基本单位的范数.
\end{prooff}
下面我们讨论$d\equiv 1\mod{4}$时解的结构,由于此时代数整数环里元素的范数并非与该形式Pell方程一一对应,因此我们需要对命题进行一定转化.

考虑$K=\bb{Q}(\sqrt{d})$的代数整数环
\begin{equation*}
    O_{k}=\bbrace{k_1+k_2\frac{1+\sqrt{d}}{2}:k_1,k_2\in \bb{Z}}
\end{equation*}
中的元素$u=a+b\dfrac{1+\sqrt{d}}{2}$,$u$为单位等价于$N_{K/\mathbb{Q}}(u)=\pm 1$也就是
\begin{equation*}
    (2a+b)^2-db^2=\pm 4
\end{equation*}
容易验证集合\begin{equation*}
    A=\bbrace{(x,y)\in \bb{Z}^2: (2x+y)^2-dy^2=4}
\end{equation*}和集合\begin{equation*}
    B=\bbrace{(x,y)\in \bb{Z}^2: x^2-dy^2=4}
\end{equation*}
通过映射\begin{equation}
    \varphi:A\rightarrow B \quad (x,y)\rightarrow  (2x+y,y)\label{map between A and B}
\end{equation}
建立一一对应.(单射显然,双射只需$\mod{4}$证明右边的解$(x,y)$同奇偶)

显然\begin{equation*}
    C=\bbrace{(x,y)\in \bb{Z}^2: (2x+y)^2-dy^2=-4}
\end{equation*}
与\begin{equation*}
    D=\bbrace{(x,y)\in \bb{Z}^2: x^2-dy^2=-4}
\end{equation*}
也有相同的对应关系.

现在设$\epsilon=a+b\omega$为基本单位,则由(\ref{group of unit})能作为基本单位的只有$\pm \epsilon^{-1},\pm \epsilon$,这四者中
存在唯一个$>1$,从而不妨设$\epsilon=a+b\omega>1,\epsilon^n=a_n+b_n\omega$.

当$N_{K/\mathbb{Q}}(\epsilon)= 1$时,
方程$(2x+y)^2-dy^2=4$全部整数解为
\begin{equation}
    \bbrace{(a_n,b_n):n\in \mathbb{Z}}\cup \bbrace{(-a_n,-b_n):n\in \mathbb{Z}}
\end{equation}
方程$(2x+y)^2-dy^2=-4$无整数解.
由(\ref{map between A and B})我们得到:$x^2-dy^2=-4$无整数解.
为了表示$x^2-dy^2=4$的正整数解,我们引入一组新的序列$$\bbrace{c_n=2a_n+b_n:n\in \mathbb{Z}},\bbrace{d_n=b_n:n\in \mathbb{Z}}$$
则$x^2-dy^2=4$的全体整数解可表示为
\begin{equation}
    \bbrace{(c_n,d_n):n\in \mathbb{Z}}\cup \bbrace{(-c_n,-d_n):n\in \mathbb{Z}}
\end{equation}
当$N_{K/\mathbb{Q}}(\epsilon)=-1$时,类比Theorem~\ref{theo pell 4k+2,3}~我们得到,
方程$(2x+y)^2-dy^2=4$全部整数解为:
\begin{equation}
    \label{exer1} \bbrace{(a_{2n},b_{2n}):n\in \mathbb{Z}}\cup \bbrace{(-a_{2n},-b_{2n}):n\in \mathbb{Z}}
\end{equation}
方程$(2x+y)^2-dy^2=-4$全部整数解为:
\begin{equation}
    \bbrace{(a_{2n-1},b_{2n-1}):n\in \mathbb{Z}}\cup \bbrace{(-a_{2n-1},-b_{2n-1}):n\in \mathbb{Z}}
\end{equation}
从而$x^2-dy^2=4$的全体整数解可表示为
\begin{equation}
    \bbrace{(c_{2n},d_{2n}):n\in \mathbb{Z}}\cup \bbrace{(-c_{2n},-d_{2n}):n\in \mathbb{Z}}
\end{equation}
$x^2-dy^2=-4$的全体整数解可表示为:
\begin{equation*}
    \bbrace{(c_{2n-1},d_{2n-1}):n\in \mathbb{Z}}\cup \bbrace{(-c_{2n-1},-d_{2n-1}):n\in \mathbb{Z}}
\end{equation*}
%递推关系:
% 现在我们考虑一下解之间的递推关系,
有了上述分析,我么可以刻画$x^2-dy^2=\pm 1$的解的结构了,记$\textcolor{blue}{m=c_1,n=d_1}$,则定理表述如下:
\begin{theo}
    当$N_{K/\mathbb{Q}}(\epsilon)= 1$时,Pell方程$x^2-dy^2=-1$无整数解,Pell方程$x^2-dy^2=1$分下列两种情况,
    \begin{enumerate}
        \item 当$m\equiv n\equiv 0\mod{2}$时,全部整数解为
              \begin{equation*}
                  \bbrace{(\frac{1}{2}c_n,\frac{1}{2}d_n):n\in \mathbb{Z}}\cup \bbrace{(-\frac{1}{2}c_n,-\frac{1}{2}d_n):n\in \mathbb{Z}}
              \end{equation*}
        \item 当$m\equiv n\equiv 1\mod{2}$时,全部整数解为
              \begin{equation*}
                  \bbrace{(\frac{1}{2}c_{3n},\frac{1}{2}d_{3n}):n\in \mathbb{Z}}\cup \bbrace{(-\frac{1}{2}c_{3n},-\frac{1}{2}d_{3n}):n\in \mathbb{Z}}
              \end{equation*}
    \end{enumerate}
    当$N_{K/\mathbb{Q}}(\epsilon)= -1$时,Pell方程$x^2-dy^2=-1$分下列两种情况讨论:\begin{enumerate}
        \item 当$m\equiv n\equiv 0\mod{2}$时,全部整数解为
              \begin{equation*}
                  \bbrace{(\frac{1}{2}c_{2n+1},\frac{1}{2}d_{2n+1}):n\in \mathbb{Z}}\cup \bbrace{(-\frac{1}{2}c_{2n+1},-\frac{1}{2}d_{2n+1}):n\in \mathbb{Z}}
              \end{equation*}
        \item 当$m\equiv n\equiv 1\mod{2}$时,全部整数解为
              \begin{equation*}
                  \bbrace{(\frac{1}{2}c_{6n+3},\frac{1}{2}d_{6n+3}):n\in \mathbb{Z}}\cup \bbrace{(-\frac{1}{2}c_{6n+3},-\frac{1}{2}d_{6n+3}):n\in \mathbb{Z}}
              \end{equation*}
    \end{enumerate}
    Pell方程$x^2-dy^2=1$分下列两种情况讨论:
    \begin{enumerate}
        \item 当$m\equiv n\equiv 0\mod{2}$时,全部整数解为
              \begin{equation*}
                  \bbrace{(\frac{1}{2}c_{2n},\frac{1}{2}d_{2n}):n\in \mathbb{Z}}\cup \bbrace{(-\frac{1}{2}c_{2n},-\frac{1}{2}d_{2n}):n\in \mathbb{Z}}
              \end{equation*}
        \item 当$m\equiv n\equiv 1\mod{2}$时,全部整数解为
              \begin{equation*}
                  \bbrace{(\frac{1}{2}c_{6n},\frac{1}{2}d_{6n}):n\in \mathbb{Z}}\cup \bbrace{(-\frac{1}{2}c_{6n},-\frac{1}{2}d_{6n}):n\in \mathbb{Z}}
              \end{equation*}
    \end{enumerate}\label{pell 4k+1}
\end{theo}
\begin{prooff}
    显然这里方程的所有解就是前文分析的$x^2-dy^2=\pm 4$解的两项皆为偶数的部分同时除以$2$,
    所以我们只需判断何时$\bbrace{c_n:n\in \mathbb{Z}},\bbrace{d_n:n\in \mathbb{Z}}$同为奇数,何时同为偶数.
    注意到:
    \begin{equation*}
        \frac{c_{n+1}+d_{n+1}\sqrt{d}}{2}=a_{n+1}+b_{n+1}\frac{1+\sqrt{d}}{2}=(a_{n}+b_{n}\frac{1+\sqrt{d}}{2})(a+b\frac{1+\sqrt{d}}{2})= (\frac{c_{n}+d_{n}\sqrt{d}}{2})(\frac{m+n\sqrt{d}}{2})
    \end{equation*}从而:
    \begin{equation}
        \begin{bmatrix}
            c_{n+1} \\
            d_{n+1}
        \end{bmatrix}=
        \begin{bmatrix}
            m/2 & nd/2 \\
            n/2 & m/2
        \end{bmatrix}\begin{bmatrix}
            c_n \\
            d_n
        \end{bmatrix}\label{c,d transform}
    \end{equation}
    对递推公式消去$\bbrace{d_n}$得到:
    \begin{equation*}
        c_{n+2}=mc_{n+1}-c_n
    \end{equation*}
    该等式可以完全刻画$c_n\mod{2}$的周期性分布.

    最后我们来处理$d$有平方因子的情况,注意到$d$可以写成$d=n_0^2d\p$,其中$d\p$无平方因子,此时我们可以将方程
    \begin{equation*}
        x^2-dy^2=\pm 1,  x^2-dy^2= 1, x^2-dy^2=- 1
    \end{equation*}
    的解与方程
    \begin{equation*}
        x^2-d\p y^2=\pm 1,  x^2-d\p y^2= 1, x^2-d\p y^2=- 1
    \end{equation*}
    满足$n_0|y$的解一一对应.
    \begin{lem}
        考虑初值:
        \begin{equation*}
            x_1=b_1,x_2=b_2,\dots,x_k=b_k
        \end{equation*}
        均给定且为整数的递推数列:
        \begin{equation*}
            x_{n+k}=a_1x_n+a_2x_{n+1}+\dots+a_kx_{n+k-1}
        \end{equation*}
        其中$a_i$也均为整数,则对任意$m$为正整数,$x_n$在$\mod{m}$下取值呈周期性。
    \end{lem}
    \begin{prooff}
        考虑数列$u_n$满足下列条件:
        \begin{enumerate}
            \item $1\le u_n\le m$
            \item  $u_n\equiv x_n\mod{m}$
        \end{enumerate}
        易知这样的$u_n$存在且唯一,再考虑:
        \begin{equation*}
            U_i= (u_{1+i},u_{2+i},\dots,u_{k+i})\in \bb{Z}^k,i=0,\dots ,m^k
        \end{equation*}
        由抽屉原理必有$s,t$使得$U_s=U_t$,从而有:
        \begin{equation*}
            x_{s+1}\equiv x_{t+1}\mod{m},x_{s+2}\equiv x_{t+2}\mod{m},\dots,x_{s+k}\equiv x_{t+k}\mod{m}
        \end{equation*}
        从而$s-t$是该数列$\mod{m}$的一个周期.
    \end{prooff}
    由上述引理,结合(\ref{c,d transform})和(\ref{a_n,b_n matrix}),将第一个变量消去得到第二个变量的递推式,在结合其$\mod{n_0}$的周期性即可得到方程
    \begin{equation*}
        x^2-dy^2=\pm 1,  x^2-dy^2= 1, x^2-dy^2=- 1
    \end{equation*}
    的全部解
\end{prooff}
\section{基本单位的判定}
本节我们总结几个常用的基本单位的判定.
\begin{prop}
    \label{judge}
    \begin{enumerate}
        \item $d\equiv 2,3\mod{4}$时,基本单位$\epsilon=a+b\sqrt{d}$可以通过将$y=1,2,\dots $依次代入$dy^2\pm 1$看其是否为完全平方数,
              如果$dy^2-1$为完全平方数$x^2,x>0$,则此时基本单位为$x+y\sqrt{d}$,范数为$-1$,
              如果$dy^2+1$为完全平方数$x^2,x>0$,则此时基本单位为$x+y\sqrt{d}$,范数为$1$.
        \item $d\equiv 1\mod{4}$时,且$d\neq 5$,基本单位$\epsilon=a+b\omega =\dfrac{m+n\sqrt{d}}{2}>1$,
              可以依次用$y=1,2\dots,$代入$dy^2\pm 4$看其是否为完全平方数,如果$dy^2-4$为完全平方数$x^2,x>0$,则基本单位为$\dfrac{x+y\sqrt{d}}{2}$,范数为$-1$,
              如果$dy^2+4$为完全平方数$x^2,x>0$,则基本单位为$\dfrac{x+y\sqrt{d}}{2}$,范数为$1$.
        \item $d=5$时,基本单位为$\dfrac{1+\sqrt{5}}{2}$
    \end{enumerate}
\end{prop}
\begin{prooff}
    1的证明已经在Theorem~\ref{solution calu}~给出,只证明2,3.

    先证明3,由于$\epsilon >1$,所以$\epsilon$为$\pm\epsilon^{-1},\pm\epsilon$中最大者,因此$m>0,n>0$,而$\dfrac{1+\sqrt{5}}{2}>1$,因此有
    $\epsilon^k=\dfrac{1+\sqrt{5}}{2},k>0$,但是如果$k>1$,则有\begin{equation*}
        \frac{m+n\sqrt{d}}{2}=\epsilon<\epsilon^k=\dfrac{1+\sqrt{5}}{2}
    \end{equation*}
    与$m>0,n>0$矛盾,因此$k=1$.

    已经排除了第三种情况,所以$dy^2+4,dy^2-4$不能同时为完全平方数,注意到$x^2-dy^2=\pm 4$全体正整数解为$\bbrace{(c_n,d_n):n\in \mathbb{Z}_{>0}}$,
    因此只需证明递推公式(\ref{c,d transform})中表示的序列$d_n$是递增的,如果$m>1$递增是显然的,$m=1$时会回归到2情况,证毕.
\end{prooff}
上面一个性质是具体计算的角度,下面的性质都是理论的角度.
\begin{prop}
    设$p\equiv 1\mod{4}$为素数,则$\mathbb{Q}{(\sqrt{p})}$的基本单位的范数为$-1$.
\end{prop}
\begin{prooff}
    由Theorem~\ref{pell 4k+1}~,等价于证明$x^2-dy^2=-1$有整数解.反证法,假设无整数解,
    则基本单位范数为$1$,取$(x_0,y_0)$为$x^2-dy^2=1$的一组正整数解使得$x_0$最小,$\mod{4}$知$x_0\equiv 1\mod{2},y_0\equiv 0\mod{2},$
    这组解为$(\frac{1}{2}c_1,\frac{1}{2}d_1)$或$(\frac{1}{2}c_3,\frac{1}{2}d_3)$,
    则\begin{equation*}
        x_0^2-py_0^2=1\Rightarrow py_0^2=(x_0-1)(x_0+1)\Rightarrow p(\frac{y_0}{2})^2=\frac{x_0-1}{2}\frac{x_0+1}{2}
    \end{equation*}
    从而由$x_1>0,x_2>0$
    \begin{equation*}
        x_1^2=\frac{x_0-1}{2},px_2^2=\frac{x_0+1}{2}
    \end{equation*}
    或者
    \begin{equation*}
        x_1^2=\frac{x_0+1}{2},px_2^2=\frac{x_0-1}{2}
    \end{equation*}
    不论是哪种情况通过左式减右式都会矛盾,前者与方程无解矛盾,后者与最小性矛盾.
\end{prooff}

\begin{prop}
    $d\equiv 3\mod{4}$且无平方因子,则$K=\mathbb{Q}{(\sqrt{d})}$基本单位范数为$1$.
\end{prop}
\begin{prooff}
    由Theorem~\ref{theo pell 4k+2,3}~,等价于证明$x^2-dy^2=-1$无整数解.反证法,假设有解,则两边$\mod{4}$得到矛盾.
\end{prooff}
\begin{prop}
    设$d=t^2+4$无平方因子,$t>0$,则$\dfrac{t+\sqrt{d}}{2}$为实二次域$\mathbb{Q}{(\sqrt{d})}$的基本单位.
\end{prop}
\begin{prooff}
    $t=1$时为Proposition~\ref{judge}~的情况3,$t>1$时,由Proposition~\ref{judge}~的情况2,
    可知$\dfrac{t+\sqrt{d}}{2}$为实二次域的基本单位,且基本单位的范数为$-1$.
\end{prooff}
\begin{prop}
    设$d=t^2-4$无平方因子,$t\ge 5$,则$\dfrac{t+\sqrt{d}}{2}$为实二次域$\mathbb{Q}{(\sqrt{d})}$的基本单位.
\end{prop}
\begin{prooff}
    由Proposition~\ref{judge}~的情况2,
    可知$\dfrac{t+\sqrt{d}}{2}$为实二次域的基本单位,且基本单位的范数为$1$.
\end{prooff}
\section{广义Pell方程}
未完待续......
% 本小节将讨论广义Pell方程的部分情况,即形如\begin{equation*}
%     x^2-dy^2=N
% \end{equation*}的不定方程,其中$d>1$且无平方因子,$N$为任意整数.

% 首先我们假设$K=\mathbb{Q}(\sqrt{d}),d\equiv 2,3\mod{4},d>1$且无平方因子,且类数$h(K)=1$,即$K$的代数整数环$O_k$为主理想整环.
% 实二次域麻烦之处在于不定方程$x^2-dy^2=N$不一定能和找到一个理想范数为$N$等价,这不同于虚二次域,原因是单位的范数有可能取到$-1$,
% 因此我们需要做一些特殊的处理.

% 一个代数数论中基本的结论是,素数$p$生成的理想在$O_k$中分解如下图:
% https://q.uiver.app/#q=WzAsMTUsWzAsMiwicE9fayJdLFsyLDEsInBcXGdlIDMiXSxbMywwLCIoXFxmcmFje2R9e3B9KT0wIl0sWzMsMSwiKFxcZnJhY3tkfXtwfSk9MSJdLFszLDIsIihcXGZyYWN7ZH17cH0pPS0xIl0sWzQsMCwicE9faz1QXjIiXSxbNCwxLCJwT19rPVBfMVBfMiJdLFs0LDIsInDmg6/mgKciXSxbMiwzLCJwPTIiXSxbMywzLCJcXHF1YWQgZFxcZXF1aXYgMlxcLFxcdGV4dHtvcn1cXCwzKFxcdGV4dHttb2R9NCkiXSxbMyw0LCJkXFxlcXVpdiAxKFxcdGV4dHttb2R9OCkiXSxbMyw1LCJkXFxlcXVpdiA1KFxcdGV4dHttb2R9OCkiXSxbNCwzLCJwT19rPVBeMiJdLFs0LDQsInBPX2s9UF8xUF8yIl0sWzQsNSwicOaDr+aApyJdLFswLDEsIiIsMCx7ImN1cnZlIjotM31dLFsxLDIsIiIsMCx7Im9mZnNldCI6LTEsImN1cnZlIjotMn1dLFsxLDNdLFsxLDQsIiIsMCx7ImN1cnZlIjoyfV0sWzAsOCwiIiwyLHsiY3VydmUiOjN9XSxbOCw5XSxbOCwxMCwiIiwyLHsib2Zmc2V0IjoyLCJjdXJ2ZSI6MX1dLFs4LDExLCIiLDIseyJvZmZzZXQiOjEsImN1cnZlIjozfV1d
\begin{tikzcd}
    &&& {(\dfrac{d}{p})=0} & {pO_k=P^2} \\
    && {p\ge 3} & {(\dfrac{d}{p})=1} & {pO_k=P_1P_2} \\
    {pO_k} &&& {(\dfrac{d}{p})=-1} & {\text{inertia}} \\
    && {p=2} & {\quad d\equiv 2\,\text{or}\,3(\text{mod}4)} & {pO_k=P^2} \\
    &&& {d\equiv 1(\text{mod}8)} & {pO_k=P_1P_2} \\
    &&& {d\equiv 5(\text{mod}8)} & {\text{inertia}}
    \arrow[curve={height=-18pt}, from=3-1, to=2-3]
    \arrow[shift left, curve={height=-12pt}, from=2-3, to=1-4]
    \arrow[from=2-3, to=2-4]
    \arrow[curve={height=12pt}, from=2-3, to=3-4]
    \arrow[curve={height=18pt}, from=3-1, to=4-3]
    \arrow[from=4-3, to=4-4]
    \arrow[shift right=2, curve={height=6pt}, from=4-3, to=5-4]
    \arrow[shift right, curve={height=18pt}, from=4-3, to=6-4]
\end{tikzcd}

% 又因为如果理想$A$范数$N$,则$N\in A$,从而$A|(N)$,因此任何范数为$N$的理想
% 一定是$N$生成的理想的因子,根据上图,将$N$在$\bb{Z}$上进行素因数分解后,如果素因子在$O_k$
% 上生成的理想惯性,则必须取偶数个,如果分裂为两个不同的理想,则取法总数为要乘以对应素数的素因子幂次加$1$,
% 如果素数分歧为两个相同理想,只需将这些理想平分,因此首先我们能求出所有范数$N$的理想.

% 先找到这些素理想的生成元,所有生成元只差一个单位,通过$N$的正负判断这个生成元该乘以范数为何的单位,从而得到$x^2-dy^2=N$的所有解.
% 但由于一个基本单位范数为$1$的二次域并不存在范数为$-1$的单位,因此即使可以找到范数为$N$的理想,我们也不一定能保证方程$x^2-dy^2=N$有解,
% 但是$x^2-dy^2=\pm N$的解通过上述手段是可以很轻松的求出的.


% 而对于$d\equiv 1\mod{4},h(K)=1$的情况,我们可以用上述方法求出不定方程\begin{equation*}
%     x^2+xy+\frac{1-d}{4}y^2=N
% \end{equation*}
% 的所有解,此时我们可以将这个不定方程像前面一样和$x^2-dy^2=4N$的解一一对应起来,此时$x^2-dy^2=N$的解
% 正是将$x^2-dy^2=4N$所有两项均为偶数的解摘出来除以$2$,此时的周期性具体计算时可以归结为二阶线性递推数列本身的周期性,
% 这个命题将被总结在本文的最后,利用该方法我们可以从理论上获得$d$有平凡因子时$Pell$方程的全部解.
\section{例题分析}
% \begin{exer}
%     设$A=\begin{bmatrix}
%             2 & 1 \\
%             1 & 1
%         \end{bmatrix},B$与$A$可交换,且元素均为正整数且行列式为$1$,求证存在$k\in \bb{Z}_{>0}$使得$B=A^k$
% \end{exer}
% \begin{prooff}
%     设$B=\begin{bmatrix}
%             a & b \\
%             c & d
%         \end{bmatrix}$,$A,B$可交换知:\begin{equation*}
%         b=c ,d=a-b
%     \end{equation*}
%     结合题中所给$\det B=1$,所以我们只需要求出$a^2-ab-b^2=1$的正整数解,并刻画出递推关系即可解决该题.

%     先求其所有整数解,这等价于求$(2a-b)^2-5b^2=4$的整数解,考虑式(\ref{exer1}),整数解即为\begin{equation*}
%         \bbrace{(a_{2n},-b_{2n}):n\in \bb{Z}}\cup \bbrace{(-a_{2n},b_{2n}):n\in \bb{Z}}
%     \end{equation*}
%     注意到$\bb{Q}(\sqrt{5})$的基本单位为$\dfrac{1+\sqrt{5}}{2}$,且
%     \begin{equation*}
%         a_{n+1}+b_{n+1}\omega= (a_n+b_n\omega)(a+b\omega)=aa_n+b\frac{d-1}{4}b_n+(bb_n+ab_n+ba_n)\omega
%     \end{equation*}
%     从而\begin{equation*}
%         \begin{bmatrix}
%             a_{n+1} \\
%             b_{n+1}
%         \end{bmatrix}=
%         \begin{bmatrix}
%             a & b\dfrac{d-1}{4} \\
%             b & a+b
%         \end{bmatrix}\begin{bmatrix}
%             a_n \\
%             b_n
%         \end{bmatrix}=\begin{bmatrix}
%             0 & 1 \\
%             1 & 1
%         \end{bmatrix}\begin{bmatrix}
%             a_n \\
%             b_n
%         \end{bmatrix}
%     \end{equation*}
%     从而\begin{equation*}
%         \begin{bmatrix}
%             a_{2n+2} \\
%             -b_{2n+2}
%         \end{bmatrix}=
%         =\begin{bmatrix}
%             1  & -1 \\
%             -1 & 2
%         \end{bmatrix}\begin{bmatrix}
%             a_{2n} \\
%             -b_{2n}
%         \end{bmatrix}\quad n\in \mathbb{Z}
%     \end{equation*}
%     其中$a_0=1,-b_0=0$,现在我们只需求出这两个序列中均为正整数或者均为负整数的部分,注意到$\begin{bmatrix}
%             1  & -1 \\
%             -1 & 2
%         \end{bmatrix}^{-1}=\begin{bmatrix}
%             2 & 1 \\
%             1 & 1
%         \end{bmatrix}$,我们列出$a_{2n},-b_{2n}$前几项并
%     进行简单的归纳,我们可以得到所有正整数解为:
%     \begin{equation*}
%         \begin{bmatrix}
%             2 & 1 \\
%             1 & 1
%         \end{bmatrix}^{n}\begin{bmatrix}
%             1 \\
%             0
%         \end{bmatrix}\quad n\in \bb{Z}_{>0}
%     \end{equation*}
%     从而存在$k\in \mathbb{Z}_{>0}$使得
%     \begin{equation*}
%         \begin{bmatrix}
%             a \\
%             b
%         \end{bmatrix}=\begin{bmatrix}
%             2 & 1 \\
%             1 & 1
%         \end{bmatrix}^{k}\begin{bmatrix}
%             1 \\
%             0
%         \end{bmatrix}
%     \end{equation*}
%     由于$A^{k}$为对称矩阵,所以\begin{equation*}
%         \begin{bmatrix}
%             2 & 1 \\
%             1 & 1
%         \end{bmatrix}^{k}=\begin{bmatrix}
%             a & b \\
%             b & q
%         \end{bmatrix}=\begin{bmatrix}
%             a & b   \\
%             b & a-b
%         \end{bmatrix}=B
%     \end{equation*}
%     倒数第二个等号是因为$\det A^{k}=1$且$a^2-ab-b^2=1$
% \end{prooff}

\begin{exer}
    求所有三边长为连续自然数的三角形,其面积为正整数.
\end{exer}
\begin{prooff}
    设三边长分别为$n-1,n,n+1$,其面积由海伦公式为\begin{equation*}
        m=\frac{\sqrt{3n(n-2)(n+2)}}{4}
    \end{equation*}
    所以只需求不定方程\begin{equation*}
        16m^2=3(n-2)(n+2)n^2
    \end{equation*}的正整数解.

    注意到$n\equiv 0\mod{2}$,设$n=2x$,则原方程化为\begin{equation*}
        3(x-1)(x+1)x^2=m^2
    \end{equation*}
    注意到$x^2|m^2$,令$m=xk$有
    \begin{equation*}
        3(x-1)(x+1)=k^2
    \end{equation*}
    注意到$3|k$再令$k=3t$得到\begin{equation*}
        3(x-1)(x+1)=9t^2
    \end{equation*}
    即$x^2-3t^2=1$,由Theorem~\ref{solution calu}~,所有正整数解可以表示为:
    \begin{equation*}
        \begin{bmatrix}
            x \\
            t
        \end{bmatrix}=
        \begin{bmatrix}
            2 & 3 \\
            1 & 2
        \end{bmatrix}^{n}\begin{bmatrix}
            1 \\
            0
        \end{bmatrix}\quad n\in \bb{Z}_{>0}
    \end{equation*}
    从而原方程的解为$(3xt,2x)$
\end{prooff}
% \begin{exer}[\cite{mingtirenjiangzuo}]
\begin{exer}
    求$5^a-3^b=2$的所有正整数解$(a,b)$.
\end{exer}
\begin{prooff}
    注意到$(a,b)=(1,1)$是一组解,下面证明$a,b$都大于$1$时该方程无解.

    反证法,假设有解$(a,b)$,分别$\mod{3},\mod{4}$知$a,b$均为奇数,
    一个很难注意到的等式是\begin{equation*}
        15(3^{\frac{b-1}{2}}5^{\frac{a-1}{2}})^{2}=(3^b+1)^2-1
    \end{equation*}

    考虑Pell方程$x^2-15y^2=1$,其所有正整数解由由Theorem~\ref{solution calu}~可表示为
    \begin{equation*}
        \begin{bmatrix}
            x_n \\
            y_n
        \end{bmatrix}=
        \begin{bmatrix}
            4 & 15 \\
            1 & 4
        \end{bmatrix}^{n}\begin{bmatrix}
            1 \\
            0
        \end{bmatrix}\quad n\in \bb{Z}_{>0}
    \end{equation*}
    只需证明$y_n$中的素因子不能只出现$3,5$,由递推式$\mod{3}$知:\begin{equation*}
        y_{n+1}\equiv y_n+1\mod{3}
    \end{equation*}
    从而$3|y_n\Leftrightarrow 3|n$,另一方面由于\begin{equation*}
        y_{n+1}=x_n+4y_n=4x_{n-1}+15y_{n-1}+4y_n=4y_n-16y_{n-1}+15y_{n-1}+4y_n=8y_n-y_{n-1}
    \end{equation*}
    将递推式$\mod{7}$知\begin{equation*}
        y_{n+1}\equiv y_n-y_{n-1}\mod{7}
    \end{equation*}
    因为$y_1\equiv y_2\equiv 1\mod{7}$,所以$7|y_n\Leftrightarrow 3|n$,因此$3|n,\Leftrightarrow 21|y_n$,这说明$y_n$的素因子不能只出现$3,5$.
\end{prooff}













% \begin{exer}
%     方程
%     \begin{equation*}
%         x^4-2y^2=17
%     \end{equation*}
%     在$\bb{Q}$上无解,但在所有$\bb{Q}_{p}$其中$p\le \infty$上有解.
% \end{exer}
% \begin{prooff}
%     在$\bb{R}=\bb{Q}_{\infty}$上该方程有解显然,考虑在$\bb{Q}_{2}$上方程的解.
%     \begin{lem}[Hensel,\cite{p-adicnumber}]
%         考虑多项式$f(x)\in \bb{Z}_{p}[x]$和$p$进整数$\alpha_1$满足:
%         \begin{equation*}
%             |f(\alpha_1)|_{p}<|f^{\p}(\alpha_1)|_{p}^{2}
%         \end{equation*}
%         则存在$\alpha\in \bb{Z}_{p}$和正整数$n$使得
%         \begin{equation*}
%             \alpha_1\equiv \alpha\mod{p^n\bb{Z}_p}
%         \end{equation*}
%         且满足$f(\alpha)=0$.
%     \end{lem}
%     根据引理,我们考虑$f(x)=x^4-17\in \bb{Z}_2[x]$和$\alpha_1=3$,容易验证$\alpha_1$满足引理所需条件,从而有
%     $\alpha\in \bb{Z}_{p}$使得$f(\alpha)=0$,因此$x=\alpha,y=0$是原方程在$\bb{Q}_2$上的解.

%     再考虑$\bb{Q}_p$其中$p$为奇素数,分两种情况讨论,当$(\frac{2}{p})=1$时,不难证明有$x_0$使得:
%     \begin{equation*}
%         x_0^2\equiv 32\mod{p}
%     \end{equation*}
%     从而得到$\alpha\in \bb{Z}_p$使得$\alpha^2-32=0$,从而$x=3,y=\alpha$为$\bb{Q}_{p}$的上方程的解.
%     当$(\frac{-2}{p})=1$时,不难证明有$y_0$使得:
%     \begin{equation*}
%         2y_0^2\equiv -1\mod{p}
%     \end{equation*}
%     从而得到$\beta\in \bb{Z}_p$使得$2\beta^2+1=0$,从而$x=2,y=\beta$为$\bb{Q}_{p}$的上方程的解.
%     综合上面两种情况,我们证明了$p\equiv 3\mod{4}$时局部解一定存在,当$p\equiv 1\mod{4}$时,$(\frac{-1}{p})=1$(不难看出$p=17=2\times 8+1$也属于上面两种情况,因此我们后续不讨论$p=17$的情况),从而有
%     \begin{equation*}
%         y_0^2\equiv -1\mod{p}
%     \end{equation*}
%     如果我们能证明方程
%     \begin{equation*}
%         x^4-17z^4=-2\mod{p}
%     \end{equation*}
%     有解$(x_0,z_0)$,此时如果$z_0\equiv 0\mod{p}$则回到前面的情况,因此我们取$(x_1,y_1)=(\dfrac{x_0}{z_0},\dfrac{y_0}{z_0^2})$(除法按取逆元理解),得到了:
%     \begin{equation*}
%         x_1^4-17=y_1^2\mod{p}
%     \end{equation*}
%     不难验证此时$p\nmid y_1^2$,因此由Hensel引理得到一个局部解.
%     因此对于局部解的情况最终只需证明
%     \begin{equation*}
%         x^4-17z^4=-2\mod{p}
%     \end{equation*}
%     有解,这需要一些雅可比和相关的知识.

%     我们称有限域乘法群$\bb{F}_p^*$到$\bb{C}^*$的同态$\chi$为一个特征标,平凡的特征标称为主特征,一般记作$\varepsilon$,非主特征补充在$0$处的定义为$0$,主特征补充为$1$.

%     \begin{prop}
%         当素数$p\nmid  a$时,方程$x^n-a\equiv 0 \mod{p}$的解数为
%         \begin{equation*}
%             \sum_{\chi^n=\varepsilon}\chi(a)
%         \end{equation*}
%         当素数$p|a$时,方程$x^n-a\equiv 0 \mod{p}$的解数为$1$.
%     \end{prop}
%     \begin{prooff}
%         后者显然,前者只需考虑特征标的生成元,计算得到两者均为$(n,p-1)$.
%     \end{prooff}
%     \begin{prop}
%         定义雅克比和为
%         \begin{equation*}
%             J(\chi,\lambda)=\sum_{a+b=1}\chi(a)\lambda(b)\quad a,b\in \bb{F}_p
%         \end{equation*}
%         \begin{enumerate}
%             \item $\chi$为非主特征,则$J(\chi,\chi^{-1})=-\chi(-1)$
%             \item $J(\varepsilon,\varepsilon)=p$
%             \item $J(\varepsilon,\chi)=0,\chi\neq \varepsilon$
%             \item  $|J(\lambda,\chi)|=\sqrt{p},\lambda,\chi,\lambda\chi\neq \varepsilon$
%         \end{enumerate}
%     \end{prop}
%     \begin{prooff}
%         详见\citep[page93]{gtm84}
%     \end{prooff}
%     记$\mod{p}$的所有四次特征为$\varepsilon,\chi_i,i=1,2,3$,$N()$表示该方程解数,
%     则由上述两条性质我们得知
%     \begin{equation*}
%         x^4-17y^4=-2\mod{p}
%     \end{equation*}
%     的解数为
%     \begin{align*}
%             & \sum_{a-17b\equiv -2\mod{p}}N(x^4=a)N(y^4=b)                                                                                \\
%         =   & \sum_{a-17b\equiv -2\mod{p}}(\varepsilon+\chi_1+\chi_2+\chi_3)(a) (\varepsilon+\chi_1+\chi_2+\chi_3)(b)                     \\
%         =   & \sum_{a+b\equiv 1\mod{p}}(\varepsilon+\chi_1+\chi_2+\chi_3)(\frac{a}{-2}) (\varepsilon+\chi_1+\chi_2+\chi_3)(\frac{b}{-34}) \\
%         \ge & p-3-6\sqrt{p}
%     \end{align*}
%     考虑所有形如$8k+5$的素数,当$p\ge 53$时,$p-3-6\sqrt{p}>0$,因此只需证明$p=5,13,29,37$的情况,但此时只需在方程
%     \begin{equation*}
%         x^4-2y^2=17
%     \end{equation*}
%     中考虑,我们通过C++简单计算知这四种情况均成立.

%     因此我们证明了局部解的存在性,下面证明全局解不存在.



%     如果该方程在$\bb{Q}$上有解,显然我们能得到方程$x^4-17z^4=2y^2$的一组整数解,因此我们只需证明方程
%     \begin{equation*}
%         x^4-17z^4=2y^2
%     \end{equation*}
%     无整数解,反证法,假设有整数解$(x,y,z)=(a,b,c)$不妨设$a,c$互素,且均为奇数.
%     \begin{lem}
%         实二次域$K=\bb{Q}(\sqrt{17})$类数为$1$.
%     \end{lem}
%     \begin{prooff}
%         注意到$K$的Minkowski常数为$\dfrac{\sqrt{17}}{2}<3$,所以$K$的理想类群可以由
%         \begin{equation*}
%             2\mathcal{O}_K=P_1P_2
%         \end{equation*}
%         中的$[P_1],[P_2]$生成,又因为$[P_2]=[P_1]^{-1}$,所以理想类群平凡,即类数为$1$.
%     \end{prooff}
%     根据(\ref{judge})我们知道$K$基本单位为$\epsilon=4+\sqrt{17}$且范数为$-1$,注意到
%     \begin{equation*}
%         \frac{a^2-c^2\sqrt{17}}{2} \frac{a^2+c^2\sqrt{17}}{2}=2(\frac{b}{2})^2
%     \end{equation*}
%     我们想证明 $s=\dfrac{a^2-c^2\sqrt{17}}{2},t=\dfrac{a^2+c^2\sqrt{17}}{2}$两者互素,只需证明他们生成的理想为$(1)$,
%     首先显然有$a^2\in (s,t),17c^2\in (s,t)$由因为$(a,c)=1$所以不难证明$(a^2,17c^2)=1$,
%     从而$s,t$互素,由代数整数环的唯一分解性和$N_{K/\bb{Q}}(t)=2(\dfrac{b}{2})^2$得到
%     \begin{equation*}
%         t=\frac{a^2+c^2\sqrt{17}}{2}=\frac{5\pm \sqrt{17}}{2}u\mu^2
%     \end{equation*}
%     其中$u$为单位,$\mu\in \mathcal{O}_K$.
%     注意到无论那种情况,左边作用$\text{Gal}(K/\bb{Q})$的元素不改变正负,因此右边也不改变正负,从而单位$u=\epsilon^{2n}$,我们不妨将两项合并起来有
%     \begin{equation*}
%         \frac{a^2+c^2\sqrt{17}}{2}=\frac{5\pm \sqrt{17}}{2}\mu^2=\frac{5\pm \sqrt{17}}{2}(\frac{\alpha +\beta\sqrt{17}}{2})^2
%     \end{equation*}
%     其中$\alpha\equiv\beta\mod{2}$,从而由线性无关性进行比较系数得到
%     \begin{equation*}
%         4a^2=5(\alpha^2+17\beta)\pm 34\alpha \beta
%     \end{equation*}
%     \mod{17}得到矛盾,证毕.
% \end{prooff}
% \begin{exer}[三元有理系数二次型]
%     设$a,b,c$是三个两两互素且无平凡因子的整数,不定方程
%     \begin{equation*}
%         ax^2+by^2+cz^2=0
%     \end{equation*}
%     在$\bb{Q}$有非平凡解(指的$xyz\neq 0$的解)当且仅当其系数$a,b,c$满足下列四个条件:
%     \begin{enumerate}
%         \item $a,b,c$不能均为非负或者非正.
%         \item 对于奇素数$p|a$,有$(\dfrac{-bc}{p})=1$,与之对称的两个条件也满足.
%         \item 若$a,b,c$均为奇数,则有其中两者之和$\equiv 0\mod{4}$
%         \item 如果其中有一个为偶数(不妨设为$a$),则$8|b+c$或$8|a+b+c$
%     \end{enumerate}
% \end{exer}
% \begin{prooff}
%     事实上任意三元有理系数二次型有无非平凡解的判断都可以通过初等的手段转化为系数无平方因子且两两互素的情况.
%     \begin{lem}[Hasse-Minkowski]
%         \begin{equation*}
%             F(x_1,x_2,\dots,x_n)\in \bb{Q}[x_1,x_2,\dots,x_n]
%         \end{equation*}
%         为一个有理系数二次型,方程
%         \begin{equation*}
%             F(x_1,x_2,\dots,x_n)=0
%         \end{equation*}
%         在$\bb{Q}$上有非平凡解当且仅当其在所有$\bb{Q}_p$(其中$p\le \infty$)上有非平凡解.
%     \end{lem}
%     \begin{prooff}
%         详见\cite{gtm7}
%     \end{prooff}
%     因此原命题等价于探索对所有$\bb{Q}_p$原方程有非平凡解的充要条件,我们先证明对奇素数$p\nmid abc$原方程在$\bb{Q}_p$上有解.
%     我们先证明方程
%     \begin{equation*}
%         ax^2+by^2+cz^2\equiv 0\mod{p}
%     \end{equation*}
%     一定有解.
%     \begin{lem}
%         定义
%         \begin{equation*}
%             J_0(\chi_1,\chi_2,\dots,\chi_l)=\sum_{t_1+\dots+t_l=0}\chi_1(t_1)\chi_2(t_2)\dots\chi_l(t_l)
%         \end{equation*}
%         有以下性质:
%         \begin{enumerate}
%             \item $J_0(\epsilon,\epsilon,\dots,\epsilon)=p^{l-1}$
%             \item $J_0(\chi_1,\chi_2,\dots,\chi_l)=0$,当$\chi_1,\dots,\chi_l$中存在平凡特征但不全为平凡特征.
%             \item $J_0(\chi_1,\chi_2,\dots,\chi_l)=0$,当$\chi_1,\dots,\chi_l$全为非平凡特征,且它们乘积为非平凡特征.
%             \item $|J_0(\chi_1,\chi_2,\dots,\chi_l)|=(p-1)p^{\frac{l-2}{2}}$,当$\chi_1,\dots,\chi_l$全为非平凡特征,且它们乘积为平凡特征.
%         \end{enumerate}
%     \end{lem}
%     \begin{prooff}
%         详见\citep[page93]{gtm84}
%     \end{prooff}
%     从而方程解数为
%     \begin{align*}
%           & \sum_{au+bv+cw\equiv 0\mod{p}}N(x^2=u)N(x^2=v)N(x^2=w)                                           \\
%         = & \sum_{au+bv+cw\equiv 0\mod{p}}  (1+(\frac{u}{p}))(1+(\frac{v}{p}))(1+(\frac{w}{p}))              \\
%         = & \sum_{u+v+w\equiv 0\mod{p}}(1+(\frac{a^{-1}u}{p}))(1+(\frac{b^{-1}v}{p}))(1+(\frac{c^{-1}w}{p})) \\
%         = & p^2
%     \end{align*}
%     因此一定存在一组非平凡的解$(x_0,y_0,z_0)$,不妨设其中$p\nmid x_0$,再对多项式
%     \begin{equation*}
%         f(x)= ax^2+(by_0^2+cz_0^2)
%     \end{equation*}
%     用Hensel引理即可得到$\bb{Q}_p$上的一个解,我们接下来处理$p=2$的情况,这需要一个引理.
%     \begin{lem}
%         \label{lem:squareinZ_2}
%         $b\in \bb{Z}_2^{\times}$为平方元当且仅当$b\equiv 1\mod{8\bb{Z}_2}$.
%     \end{lem}
%     \begin{prooff}
%         详见\citep[page73]{p-adicnumber}
%     \end{prooff}
%     当$a,b,c$全为奇数时,假如
%     \begin{equation*}
%         ax^2+by^2+cz^2=0
%     \end{equation*}
%     在$\bb{Q}_2$上有非平凡解$(x_0,y_0,z_0)$,不妨设$\max\bbrace{|x_0|_2,|y_0|_2,|z_0|_2}=1$,否则两边同时乘以$2$的幂进行调整,
%     $\mod{2\bb{Z}_2}$知$x_0,y_0,z_0$中恰有两个为$\bb{Z}_2^{\times}$中单位,不妨设为$y_0,z_0$从而由Lemma~\ref{lem:squareinZ_2}~并 $\mod{4\bb{Z}_2}$知
%     \begin{equation*}
%         b+c\equiv 0\mod{4}
%     \end{equation*}
%     反之,如果系数$a,b,c$满足均为奇数且
%     \begin{equation*}
%         b+c\equiv 0\mod{4}
%     \end{equation*}
%     我们想构造一组非平凡的解,令$y^2=1+8k_1,z^2=1+8k_2$,$k_1,k_2\in \bb{Z}_2$则有$b+c+8(k_1b+k_2c)=-ax^2$
%     当$b+c\equiv 0\mod{8}$时,可取$k_1,k_2$使得$b+c+8(k_1b+k_2c)=-16a$,则$x=4$为解,
%     当$b+c\equiv 4\mod{8}$时,可取$k_1,k_2$使得$b+c+8(k_1b+k_2c)=-4a$,则$x=2$为解,从而当系数全为奇数时,
%     $\bb{Q}_2$上有非平凡解充要条件是两者之和$\equiv 0\mod{4}$.

%     下面讨论$\bb{Q}_2$上有一个系数为偶数的情况,不妨设$a$为偶数,
%     若$\bb{Q}_2$上有非平凡解$(x_0,y_0,z_0)$,
%     通过和上面类似的思路可知只有三者均为单位或者只有$x_0$不是单位两者情况,分别$\mod{8\bb{Z}_2}$知
%     $8|a+b+c$或$8|b+c$,另一方面要找非平凡解按照系数都为奇数时的情况操作即可.

%     最后是$p|a$时$\bb{Q}_p$的情况,
%     \begin{equation*}
%         ax^2+by^2+cz^2=0
%     \end{equation*}
%     在$\bb{Q}_p$上有非平凡解$(x_0,y_0,z_0)$,不妨设$\max\bbrace{|x_0|_p,|y_0|_p,|z_0|_p}=1$,
%     否则两边同时乘以$2$的幂进行调整,分类讨论知$y_0,z_0$此时均为 $ \bb{Z}_p^{\times}$中单位,从而有$p\nmid y_1,z_1\in \bb{Z}$使得
%     \begin{equation*}
%         by_1^2\equiv -cz_1^2\mod{p}
%     \end{equation*}
%     因此有$(\dfrac{-bc}{p})=1$.
%     另一方面要证明这个条件是充要的,由于$(\dfrac{-bc}{p})=1$,从而有
%     \begin{equation*}
%         by_1^2\equiv -c\mod{p}
%     \end{equation*}
%     考虑多项式$f(y)=by^2+c$,由Hensel引理存在$y_2\equiv y_1\mod{p}$使得$f(y_2)=0$,从而$(0,y_2,1)$是一组非平凡解,
%     综合上面所有情况,原命题得证.

% \end{prooff}
% \begin{exer}
%     对任意正整数$M>0$,存在$N>0$,使得任意$n\ge N$有
%     \begin{equation*}
%         v_2(2+\frac{2^2}{2}+\frac{2^3}{3}+\dots+\frac{2^n}{n})\ge M
%     \end{equation*}
%     其中$v_2(n)$表示有理数$n$写成既约分数后分子$2$的幂次减去分母$2$的幂次.
% \end{exer}

% \begin{prooff}
%     这看上去是一个初等数论问题,实际上要用$p$进幂级数的方法来解决.
%     我们在$\bb{Q}_p$上定义幂级数
%     \begin{equation*}
%         \log_p(1+x)=\sum_{n=1}^{\infty}(-1)^{n+1}\frac{x^n}{n}
%     \end{equation*}
%     如果我们证明了$p=2$时该幂级数在$x=-2$处收敛于$0$即可得到该命题.
%     首先我们记该幂级数收敛半径为$\rho$,则收敛半径满足
%     \begin{equation*}
%         f(x)=\frac{1}{\rho}=\limsup \sqrt[n]{ \left| \frac{1}{n}\right|_p}=1
%     \end{equation*}
%     且$|x|_p=1$时候,由于$\left|\dfrac{1}{n}\right|_p$不趋于$0$,
%     因此该幂级数收敛范围为$|x|_p<1$,即为$p\bb{Z}_p$.如果我们证明了对于$a,b\in 1+p\bb{Z}_p$,
%     \begin{equation*}
%         \log_p(a)+\log_p(b)=\log_p(ab)
%     \end{equation*}
%     则取$p=2,a=b=-1$即可证明原命题.
%     首先我们对该幂级数求导得到
%     \begin{equation*}
%         f\p(x)=\sum_{n=1}^{\infty}(-1)^{n-1}x^{n-1}=\frac{1}{x+1}
%     \end{equation*}
%     其次我们取定$y\in p\bb{Z}_p$,定义
%     \begin{equation*}
%         g(x)=f(y+(1+y)x)
%     \end{equation*}
%     由于$\left| y+(1+y)x\right|_p<1\Longleftrightarrow |x|_p<1$,因此$f,g$有相同的收敛范围,由\citep[Proposition 4.4.2]{p-adicnumber},$g$可以在$0$处展开为幂级数,
%     由链式求导法则
%     \begin{equation*}
%         g\p(x)=(1+y)\frac{1}{1+(y+(1+y)x)}=\frac{1}{1+x}
%     \end{equation*}
%     由\citep[Corollary 4.4.5]{p-adicnumber},$f$和$g$只差常数,取$x=0$知
%     \begin{equation*}
%         g(0)=c+f(0)
%     \end{equation*}
%     则常数$c=f(y)$,从而有
%     \begin{equation*}
%         \log_p(a+1)+\log_p(b+1)=\log_p((a+1)(b+1))
%     \end{equation*}
%     进行一次变量替换即可得证.
% \end{prooff}

\bibliography{pell}
\bibliographystyle{plain}%终端输入bibtex+文件名再ctrl+s两次,plain(数字起头) alpha(字母起头),cd 进入下一级别 cd..返回上一级


\end{document}