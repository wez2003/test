\documentclass[UTF8]{ctexart}
\usepackage{amsmath}
\usepackage{amssymb}
\usepackage{amsthm}
\usepackage{geometry}
\newtheorem{lemma}{引理}[section]
\newtheorem{theorem}{定理}[section]
\title{RSA密码系统}
\author{濯}
\begin{document}
\maketitle
\tableofcontents
\newpage
\section{RSA背景}
RSA是1977年由罗纳德·李维斯特、阿迪·萨莫尔和伦纳德·阿德曼一起提出的。
当时他们三人都在麻省理工学院工作。RSA就是他们三人姓氏开头字母拼在一起组成的。

RSA基于$Euler$定理和大数的素因子分解非常困难。

\section{加密原理}
\begin{lemma}
    当$m$的素因数分解为\,$p_{1}p_{2} \dots p_{r}$时\\
    \begin{equation*}
        a^{\varphi (m)+1}\equiv a\,(\text{mod}\,m)
    \end{equation*}
\end{lemma}
\begin{proof}[证明:]
    只需证明\,$p_{i}\,|\,a^{\varphi (m)+1}-a \qquad i=1,2,\dots,r$\\
    注意到当\, $p_{i} \, | \,a$ 时 结论显然成立\\
    只需证明$(a,p_{i})=1$的情况,即证$a^{\varphi (m)}\equiv 1\,(\text{mod}\,p_{i})$\\
    由费马小定理 $a^{p_{i}-1}\equiv 1\,(\text{mod}\,p_{i})$
    \quad 同时$\varphi (m)=\prod\limits_{i=1}^{r}(p_{i}-1)$\\
    因此$a^{\varphi (m)}\equiv 1\,(\text{mod}\,p_{i})$\quad 进而$a^{\varphi (m)+1}\equiv a\,(\text{mod}\,m)$
\end{proof}
现在我们来具体说明RSA的实现原理。\\
设$n=pq$,$pq$是两个不同的大素数,$\alpha$ 是大于$p$和$q$的素数,则存在$\beta $使得
\begin{equation*}
    \alpha \beta \equiv 1\,(\text{mod}\,\varphi (n))
\end{equation*}
这样对于任意整数$A$,$0\leq A< n$,必有唯一整数$B$满足
\begin{equation*}
    B \equiv A^{\alpha }\,(\text{mod}\,n) \quad 0\le B<n
\end{equation*}
进而由引理得到
\begin{equation*}
    B^{\beta }\equiv A^{\alpha \beta}\equiv A\,(\text{mod}\,n) \quad 0\le A<n
\end{equation*}
\section{加密原理具体解释}
现在小王公开声明了$n$和$\alpha$,并且他已经知道了$n$的质因数分解,小卓想给小王传递数据$A$但是害怕别人知道传递的是什么,于是他将$A^{\alpha}$对$n$取模后得到$B$,将$B$传给小王,
小王用$B$计算出$\beta$ 进而得到小卓真实想传递的数据$A$。

由于其他人不知道$n$的质因数分解,于是无法从小卓传递的信息中获取真实想要传递的数据$A$。而正是$n$质因数分解的困难性保证了RSA的加密安全性。
\section{加密实践}
通过\emph{线性筛}\footnote{又称作欧拉筛}小王得到了两个数量级在$10^{7}$的素数$pq$以及$\alpha $\\
其中$n=pq=2500085400729307$,\,$\alpha =50000897$\\
现在假如你是小卓,你想向小王传递$2003211$这个数字,请问你该对小王输入什么?

\end{document}