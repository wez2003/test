\documentclass[a4paper,12pt]{ctexart}
\usepackage{amsmath}
\usepackage{amssymb}
\usepackage{amsthm}
\usepackage{geometry}
\usepackage[dvipsnames]{xcolor}
\usepackage{tcolorbox}
\usepackage{enumerate}
\linespread{1.1}
\newenvironment{prooff}{{\noindent\it\textcolor{cyan!40!black}{Proof}:}\quad}{\par}
\usepackage[colorlinks,linkcolor=cyan!40!black]{hyperref}
\usepackage{enumerate}
\geometry{a4paper,left=2cm,right=2cm,top=2.5cm,bottom=2cm}
\usepackage{tikz}
\usepackage{float}
\usepackage{caption}
\usepackage{color}
\newtheorem{defn}{Definition}
\newtheorem{coro}[defn]{Corollary}
\newtheorem{theo}[defn]{Theorem}
\newtheorem{exer}[defn]{Exercise}
\newtheorem{rema}[defn]{Remark}
\newtheorem{lem}[defn]{Lemma}
\newtheorem{prop}[defn]{Proposition}


\newcommand{\bbrace}[1]{\left\{ #1 \right\} }
\newcommand{\bb}[1]{\mathbb{#1}}
\newcommand{\p}{^{\prime}}
\renewcommand{\mod}[1]{(\text{mod}\,#1)}


\title{几个级数的计算}
\author{王尔卓 \\ 
强基数学2101}
\date{}
\begin{document}
\maketitle
\section{命题及其推论}
我们已经通过傅里叶分析的方法证明了:
\begin{lem}
    \label{lem:cot}
    \begin{equation*}
        \cot \alpha \pi =\frac{1}{\pi}(\frac{1}{\alpha}+\sum_{n=0}^{\infty}(\frac{1}{\alpha+n})+\frac{1}{\alpha-n})),\quad \alpha \notin \bb{Z}
    \end{equation*}
\end{lem}
接下来我们用上述引理证明几个级数的表达式.
\begin{exer}
    证明在$0$的充分小的去心邻域内有:
    \begin{equation*}
        \cot \alpha =\frac{1}{\alpha}-2\sum_{k=1}^{\infty}\frac{\zeta (2k)}{\pi ^{2k}}\alpha^{2k-1}
    \end{equation*}
\end{exer}
\begin{prooff}
    由引理在$0$充分小的去心邻域内有:
    \begin{align*}
        \cot \alpha & =\frac{1}{\pi}(\frac{\pi}{\alpha}+\sum_{n=1}^{\infty}(\frac{1}{\frac{\alpha}{\pi}+n}+\frac{1}{\frac{\alpha}{\pi}-n}))
        =\frac{1}{\alpha}+\frac{2\alpha}{\pi^2}\sum_{n=1}^{\infty} \frac{1}{n^2}\frac{1}{(\frac{\alpha}{n\pi})^2-1}                          \\
                    & =\frac{1}{\alpha}-\frac{2\alpha}{\pi^2}\sum_{n=1}^{\infty} \frac{1}{n^2}\sum_{k=0}^{\infty} (\frac{\alpha}{n\pi})^{2k}
        =\frac{1}{\alpha}-\frac{2\alpha}{\pi^2}\sum_{k=0}^{\infty} \sum_{n=1}^{\infty} \frac{1}{n^2}(\frac{\alpha}{n\pi})^{2k}               \\
                    & =\frac{1}{\alpha}-\sum_{k=0}^{\infty}\frac{2\alpha}{\pi^2}(\frac{\alpha}{\pi})^{2k}\zeta(2k+2)
        =\frac{1}{\alpha}-2\sum_{k=1}^{\infty}\frac{\alpha^{2k-1}}{\pi^{2k}}\zeta(2k)                                                        \\
    \end{align*}
    证毕.
\end{prooff}
\begin{coro}
    \label{coro:zeta and bernoulli}
    \begin{equation*}
        \zeta(2k)=\frac{(-1)^{k+1}B_{2k}(2\pi)^{2k}}{2(2k)!},\quad \forall k\in \bb{Z}_{>0}
    \end{equation*}
\end{coro}
\begin{prooff}
    考虑函数$f(z)=\cot z-\dfrac{1}{z}$,由于$\cot z$在$0$处为留数为$1$的一阶极点,因此$f(z)$可以看作$0$处局部的全纯函数,由全纯函数的唯一性定理,在$0$局部有:
    \begin{equation*}
        \cot z -\frac{1}{z}=-2\sum_{k=1}^{\infty}\frac{\zeta (2k)}{\pi ^{2k}}z^{2k-1}
    \end{equation*}
    又因为:
    \begin{align*}
        \cot z -\frac{1}{z} & =i\frac{e^{2iz}+1}{e^{2iz}-1}-\frac{1}{z}=i(1+\frac{2}{e^{2iz}-1})-\frac{1}{z}
        =i(1-i\frac{2iz}{z(e^{2iz}-1)})-\frac{1}{z}                                                          \\
                            & =i(1-i\frac{1}{z}\sum_{n=0}^{\infty}\frac{B_n}{n!}(2iz)^n)-\frac{1}{z}
        =i+\sum_{n=1}^{\infty}\frac{B_{n}}{n!}(2i)^{n}z^{n-1}
    \end{align*}
    对比系数得到:
    \begin{equation*}
        \frac{B_{2k}}{2k!}(2i)^{2k}=-2\frac{\zeta(2k)}{\pi^{2k}}
    \end{equation*}
    从而有:
    \begin{equation*}
        \zeta(2k)=\frac{(-1)^{k+1}B_{2k}(2\pi)^{2k}}{2(2k)!},\quad \forall k\in \bb{Z}_{>0}
    \end{equation*}
\end{prooff}
\begin{coro}[Bernoulli数绝对值的渐进公式]
    \begin{equation*}
        |B_{2n}|\sim 4\sqrt{\pi n}(\frac{n}{\pi e})^{2n}  \quad  \text{当} n\to \infty
    \end{equation*}
\end{coro}
\begin{prooff}
    注意到对实数$s>1$:
    \begin{equation*}
        1\le \zeta(s)\le 1+\sum_{n=2}^{\infty}\frac{1}{n^s}\le 1+\int_{1}^{\infty}\frac{1}{x^s}\text{d}x=1+\mathcal{O}(\frac{1}{s})
    \end{equation*}
    因此$\zeta(2n)\to 1$,再由斯特林公式以及Corollary~\ref{coro:zeta and bernoulli}~得到:
    \begin{equation*}
        |B_{2n}|\sim \frac{2(2n)!}{(2\pi)^{2n}}\sim 4\sqrt{\pi n}(\frac{n}{\pi e})^{2n}
    \end{equation*}
    证毕!
\end{prooff}

\begin{exer}
    \begin{equation*}
        \sin \pi x=\pi x\prod_{n=1}^{\infty}(1-\frac{x^2}{n^2})\quad \forall x\in \bb{R}
    \end{equation*}
\end{exer}
\begin{prooff}
    考虑整函数$\sin \pi z$的Hadamard分解:
    \begin{equation*}
        \sin \pi z=e^{P(z)}z\prod_{n=-\infty,n\neq 0}^{\infty}(1-\frac{z}{n})e^{\frac{z}{n}}=e^{P(z)}z\prod_{n=1}^{\infty}(1-\frac{z^2}{n^2})
    \end{equation*}
    其中$\deg P(z)\le 1$,设为$P(z)=az+b$,则由于$\sin(\pi z)=-\sin (-\pi z)$,对任意$z\notin \bb{Z}$,有$e^{P(z)}=e^{P(-z)}$
    从而得到:
    \begin{equation*}
        a=0
    \end{equation*}
    从而:
    \begin{equation*}
        \sin \pi z=e^{b}z\prod_{n=1}^{\infty}(1-\frac{z^2}{n^2})
    \end{equation*}
    将$z$除至左侧取$z\to 1$的极限得到$e^b=\pi $,从而:
    \begin{equation*}
        \sin \pi z=\pi z\prod_{n=1}^{\infty}(1-\frac{z^2}{n^2})
    \end{equation*}
\end{prooff}

\begin{exer}
    \begin{equation*}
        \frac{\pi^2}{\sin^{2}\pi x}=\sum_{n\in \bb{Z}}\frac{1}{(x+n)^2}\quad x\notin  \bb{Z}
    \end{equation*}
\end{exer}
\begin{prooff}
    用傅里叶分析的方法是直接对Lemma~\ref{lem:cot}~求导,接下来我们用复变的方法将这个等式推广为
    \begin{equation*}
        \sum_{n\in \bb{Z}}\frac{1}{(\tau +n)^2}= \frac{\pi^2}{\sin^{2}\pi \tau}\quad \tau\in \bb{C}-\bb{Z}
    \end{equation*}
    首先我们考虑$Im(\tau)>0 $的情况,此时考虑函数$f(z)=\dfrac{1}{(\tau+x)^k},k\ge 2$,在上半平面和下半平面分别构造半圆形曲线,
    由留数定理计算$f$的傅里叶变换得:
    \begin{align*}
        \int_{-\infty}^{\infty}f(z)e^{-2\pi ix\xi }\text{d}x=
        \begin{cases}
            \dfrac{(-2\pi i\xi)^{k-1}}{(k-1)!}2\pi ie^{2\pi i\tau \xi} & \quad \xi >0     \\
            0                                                          & \quad  \xi \le 0
        \end{cases}
    \end{align*}
    由Poisson求和公式得到:
    \begin{equation*}
        \sum_{n\in \bb{Z}}\frac{1}{(\tau+n)^k}=\sum_{n=1}^{\infty}\dfrac{(-2\pi i n)^{k-1}}{(k-1)!}2\pi ie^{2\pi i\tau n}
    \end{equation*}
    令$k=2$得到:
    \begin{equation*}
        \sum_{n\in \bb{Z} }\frac{1}{(\tau+n)^2}=\sum_{n=1}^{\infty}4\pi^2ne^{2\pi i\tau n}=\frac{\pi^2}{\sin^2 (\pi \tau )}
    \end{equation*}
    最后一个等号是对函数
    \begin{equation*}
        g(z)=\sum_{n=1}^{\infty}e^{2\pi i \tau n }=\frac{e^{2\pi i\tau} }{1-e^{2\pi i \tau}}
    \end{equation*}
    求导得到的,从而$Im(\tau)>0$的情况得证,$Im(\tau)<0$的情况同理可得.
    从而我们得到了$\tau \in \bb{C}-\bb{Z}$时结论成立.
\end{prooff}
\section{总结}
通过上面几个级数的计算,我们可以看出,傅里叶分析中得到的公式
可以通过复变函数的工具进行推广,得到一些形式优美的结论。



\end{document}